\documentclass{xam}
\xamclass{ត្រៀមបាក់ឌុប}
\prepareby{ស៊ុំ សំអុន}
\phonenumber{០៩៦៩៤០៥៨៤០}
\SetWatermarkText{}
\begin{document}
  \xam{១}
  \begin{enumerate}
    \item (១៥ ពិន្ទុ) គណនាលីមីត
    \begin{Enumerate}(3)
      \item $ \lim\limits_{x\to 1}\dfrac{1-x^2}{x^3-x^2+x-1} $
      \item $ \lim\limits_{x\to 0}\dfrac{\sin 3x}{-x} $
      \item $ \lim\limits_{x\to 0}\dfrac{\sqrt{2+x}-\sqrt{2-x}}{\sin x} $~។
    \end{Enumerate}
    \item (១០ ពិន្ទុ) ក្នុងថ្នាក់រៀនមួយមានសិស្សពូកែ $ 10 $ នាក់ ដែលក្នុងនោះ $ 4 $ នាក់ជាសិស្សស្រីនិង $ 6 $ នាក់ជាសិស្សប្រុស។ គេរៀបចំសិស្សជាក្រុមក្នុងមួយក្រុមមានសិស្ស $ 4 $ នាក់ដោយចៃដន្យ យកទៅប្រកួតជាមួយក្រុមសិស្សថ្នាក់ដ៏ទៃ។ រកប្រូបាបនៃព្រឹត្តិការណ៍ខាងក្រោម៖
    \begin{enumerate}
      \item $ A: $`` ក្រុមសិស្សដែលជ្រើសរើសបានសុទ្ធតែស្រី ''។
      \item $ B: $`` ក្រុមសិស្សដែលជ្រើសរើសបានសុទ្ធតែប្រុស ''។
      \item $ C: $`` ក្រុមសិស្សដែលជ្រើសរើសបាន $ 50\% $ ជាសិស្សប្រុស ''។
    \end{enumerate}
    \item (១៥ ពិន្ទុ) គេមានចំនួនកុំផ្លិច $ z_1=1+i\sqrt{3} $ និង $ z_2=6\left(\cos\dfrac{\pi}{4}+i\sin\dfrac{\pi}{4}\right) $
    \begin{enumerate}
      \item សរសេរ $ z_1 $ ជាទម្រង់ត្រីកោណមាត្រ។
      \item រកម៉ូឌុលនិងអាគុយម៉ង់ $ z_1^3 $~។
      \item សរសេរផលគុណ $ z_1\times z_2 $ ជាទម្រង់ពីជគណិត។
    \end{enumerate}
    \item (២៥ ពិន្ទុ)
    \begin{enumerate}
      \item ក្នុងលំហប្រដាប់ដោយតម្រុយ $ (O,\vec{i},\vec{j},\vec{k}) $ គេមានចំណុច $ A(-2,1,0),B(0,1,1),C(1,2,2) $ និង $ D(0,3,-4) $។
      \begin{enumerate}
        \item រកវ៉ិចទ័រ $ \overrightarrow{AB},\overrightarrow{AC},\overrightarrow{AD},\overrightarrow{BC},\overrightarrow{CD} $~។
        \item គណនាប្រវែង $ AB,AC,AD,BD $ និង $ CD $~។ ទាញបង្ហាញថាត្រីកោណ $ ABD $ និង $ ACD $ កែងត្រង់ $ A $~។
      \end{enumerate}
      \item គេមានសមីការ $ 9y^2-16x^2=144 $~។ បង្ហាញថាសមីការនេះជាសមីការអ៊ីពែបូល។
      \item[] រកកូអរដោនេរបស់កំពូលទាំងពីរនិងកំណុំទាំងពីរនៃអ៊ីពែបូល។
      \item[] រកសមីការអាស៊ីមតូតរបស់អ៊ីពែបូលនេះ និងសង់អ៊ីពែបូលនេះ។
    \end{enumerate}
    \item (១៥ ពិន្ទុ) គណនាអាំងតេក្រាល $ I=\displaystyle\int_{1}^{3}\left(x-2+3x^3\right)\mathrm{d}x $; $ J=\displaystyle\int_{0}^{\frac{\pi}{4}}\left(\sin 2x-\cos x\right)\mathrm{d}x $; \\
    $ K=\displaystyle\int_{0}^{1}\frac{x^3+(x+1)^2}{x^2+1}\mathrm{d}x $។ ដើម្បីគណនា $ K $ យើងត្រូវបង្ហាញថា $ \dfrac{x^3+(x+1)^2}{x^2+1}=x+1+\dfrac{x}{x^2+1} $។
    %
    \item (១០ ពិន្ទុ)
    \begin{enumerate}
      \item ដោះស្រាយសមីការឌីផេរ៉ង់ស្យែល $ (E): y''-3y'+2y=0 $~។
      \item រកចម្លើយពិសេសមួយនៃសមីការឌីផេរ៉ង់ស្យែល $ (E) $ ដែល $ y(0)=1 $ និង $ y'(1)=e^2 $~។
    \end{enumerate}
    %
    \item (៣៥ ពិន្ទុ) គេមានអនុគមន៍ $ f $ កំណត់លើ $ \mathbb{R} $ ដោយ $ f(x)=x+\dfrac{1-3e^x}{1+e^x} $~។ គេតាងដោយ $ C $ ក្រាបរបស់វានៅក្នុងប្លង់ប្រដាប់ដោយតម្រុយអរតូណរម៉ាល់ $ (0,\vec{i},\vec{j}) $~។
    \begin{enumerate}
      \item បង្ហាញថា $ f(x)=x+1-\dfrac{4e^x}{1+e^x} $ និងគណនាលីមីតនៃ $ f $ ត្រង់ $ -\infty $~។ ស្រាយបំភ្លឺថាបន្ទាត់ $ d_1 $ ដែលមានសមីការ $ y=x+1 $ អាស៊ីមតូតទៅនឹងក្រាប $ C $ ត្រង់ $ -\infty $~។ សិក្សាទីតាំងនៃក្រាប $ C $ ធៀបនឹងបន្ទាត់ $ d_1 $~។
      \item គណនាលីមីត $ f $ ត្រង់ $ +\infty $~។ ស្រាយបំភ្លឺថាបន្ទាត់ $ d_2 $ ដែលមានសមីការ $ y=x-3 $ អាស៊ីមតូតទៅនឹងក្រាប $ C $ ត្រង់ $ +\infty $~។\\
      សិក្សាទីតាំងក្រាប $ C $ ធៀបនឹងបន្ទាត់ $ d_2 $~។
      \item 
      \begin{enumerate}
        \item គណនាដេរីវេ $ f'(x) $ និងបង្ហាញថាគ្រប់ចំនួនពិត $ x,\; f'(x)=\left(\dfrac{e^x-1}{e^x+1}\right)^2 $~។
        \item សិក្សាអថេរភាពនៃ $ f $ រួចសង់តារាងអថេរភាពនៃ $ f $~។ សង់ក្រាប $ C $ និងអាស៊ីមតូត $ d_1 $ និង $ d_2 $ របស់វា។
      \end{enumerate}
    \end{enumerate}
  \end{enumerate}
\end{document}