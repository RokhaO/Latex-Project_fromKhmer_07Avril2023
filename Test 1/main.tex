% !TEX encoding = UTF-8 Unicode
% J.Roussel
% MAJ : 2014-06-03
%-------------------------------------------
\documentclass[11pt]{article}
\usepackage[usenames, dvipsnames]{xcolor}
\usepackage[utf8x]{inputenc}
\usepackage[T1]{fontenc}
\usepackage{tikz}
\usepackage{pgfplots}
\pgfplotsset{compat=1.9}
\usetikzlibrary{calc,arrows,fadings,decorations.pathreplacing,decorations.markings,patterns,shapes.geometric}
\usepackage{fp}
\usepackage{verbatim}
\usepackage[active,tightpage]{preview}
\PreviewEnvironment{tikzpicture}
\setlength\PreviewBorder{5pt}



%------ styles tikz ---------------
\colorlet{darkblue}{blue!50!black} 
\tikzset{>=stealth,inner sep=0pt, outer sep=2pt,}
\tikzset{tiret/.style={gray,dashed}}
\tikzset{doublefleche/.style={|<->|,>=stealth,thin}}
\tikzset{titre/.style={inner sep=0pt, outer sep=0pt,above right,text justified,fill=orange!50}}
\tikzset{bloc/.style={rounded corners=4pt,color=white,ball color=purple,smooth}}
\tikzset{force/.style={->,ultra thick,rounded corners=4pt,color=blue,smooth,line cap=round}}
\tikzset{vecteur/.style={->,thick,color=black,smooth}}
\tikzset{verre/.style={draw=SkyBlue,fill=SkyBlue!30}}
\tikzset{axis/.style={thin,gray}}
\tikzset{figure/.style={thick,color=#1,fill=purple, opacity=0.5}}
\tikzset{ressort/.style={very thick,black,smooth}}
\tikzset{eau/.style={draw=black,fill=blue,opacity=0.5}}
%-----------------------------------

%-------- patatoide ----------
\newcommand{\patate}[1][fill=white] 
{\draw [#1][preaction={fill=white}] (0,0) .. controls +(0.1,0.2) and +(0.3,0.3) .. (1,0) .. controls +(-0.3,-0.3) and +(-0.05,0.15) .. (1,-1) .. controls +(0.2,-0.6) and +(0.1,-0.2) .. (0,-1) .. controls +(-0.1,0.15) and +(-0.2,-0.4) .. (0,0);}
%------ aimant -------
\newcommand{\aimant}[1][ultra thick]{
\draw[fill, color=red,#1] (0,0.2) rectangle(1,-0.2) node[color=white,midway]{\tiny S};
\draw[fill, color=black,#1] (1,0.2) rectangle(2,-0.2) node[color=white,midway]{\tiny N};}
\title{Figures TikZ du cours sur l'énergie mécanique}
\begin{document}
	
	%============================================================
	% 	profil énergétique du pendule simple
	%============================================================
	
	\begin{tikzpicture} [scale=0.8]
	\begin{scope}
	\coordinate (O) at (0, 0);
	\coordinate (M) at (-55:5);
	\draw[thick] (O)--(M); 
	\draw[thick,->] (0,-1.2) arc(-90:-55:1.2) ;
	\draw (-85:1.5) node[right]{$\theta(t)$} (-57:2.5) node[above right]{$\ell$};
	\draw[vecteur] (M)--++({cos(55)},{-sin(55)}) node[below right ]{$\overrightarrow{u_{r}}$};
	\draw[vecteur] (M)--++({sin(55)},{cos(55)}) node[above right ]{$\overrightarrow{u_{\theta}}$};
	\draw[force] (M)--(-55:3) node[below left]{$\overrightarrow{T}$};
	\draw[force] (M)--++(0,-2) node[right]{$\overrightarrow{P}=m\overrightarrow{g}$};
	\draw[bloc] (M) circle(0.15) node[black,right=5pt]{M($\ell$,$\theta$)};
	\draw [ultra thick,gray] (-1,0)--(1,0); 
	\draw [thin,gray]  (0,0.5) --(0,-6);   
	\draw[vecteur] (-1,-1)--++(0,-1) node[midway,right]{$\overrightarrow{g}$};
	\end{scope}
	\begin{scope}[xshift=9cm,yshift=-3cm]
	\def \xLabel {$\theta$}; % legende sur x
	\def \yLabel {$E_{\rm p}$}; % legende sur y
	\def \xmin{-3.5}; 
	\def \xmax{3.5};
	\def \ymin{-2.5};
	\def \ymax{3};
	\def \stepGrid{0.5cm}; % pas de la grille
	\def \titleWidth {7cm}; % largeur de la vignette de titre
	\draw[->] (\xmin,0) -- (\xmax,0) node[above=2pt] {\xLabel};
	\draw[->] (0,\ymin) -- (0,\ymax) node[below right] {\yLabel};
	\draw[style=help lines,step=\stepGrid,color=gray,opacity=0.5] (\xmin,\ymin) grid (\xmax,\ymax);% a commenter si pas de grille
	\foreach \x/\xtext in {-pi/-\pi,pi/\pi}
	\draw[shift={(\x,0)}] (0pt,2pt) -- (0pt,-2pt) node[below] {$\xtext$};
	\foreach \x/\xtext in {{-pi/3}/-\theta_{\rm max},{pi/3}/\theta_{\rm max}}
	\draw[shift={(\x,0)}] (0pt,2pt) -- (0pt,-2pt) node[above=4pt] {$\xtext$};
	\foreach \y/\ytext in {-2/-mg\ell,2/mg\ell}
	\draw[shift={(0,\y)}] (-2pt,0pt) -- (2pt,0pt) node[right] {$\ytext$};
	\draw[thick,color=purple, variable=\x, samples at={\xmin,-3.4,...,\xmax}]
	plot ({\x}, {-2*cos(\x r)}) ;% fonction à tracer
	\fill[fill=gray!20,draw=black,opacity=0.5] (pi/3,0) -- (pi/3,-1) -- plot[domain=pi/3:-pi/3] ({\x}, {-2*cos(\x r)})  -- (-pi/3,0) -- cycle;
	\draw[tiret] (-3,2.5)--++(6,0) node[pos=0.2,fill=white]{$E_{m2}$};
	\draw[tiret] (-3,-1)--++(6,0) node[pos=0.2,fill=white]{$E_{m1}$};
	\end{scope}
	\end{tikzpicture} 
	
	
	
	%============================================================
	% 	Travail d'une force	
	%============================================================
	\begin{tikzpicture} [scale=0.8] 
	\coordinate (A) at (0, 0);
	\coordinate (M) at (3,1);
	\coordinate (B) at (6,-2);  
	\draw (A)  node[below=0.2em]{A};
	\draw (M)  node[above=0.2em]{M};
	\draw (B)  node[below=0.2em]{B};
	\draw[force] (M)--++(-1,-1.2) node[pos=0.8,above left]{$\overrightarrow{f}$};
	\draw[vecteur] (M)--++(0.25,-0.4) node[above right]{$\overrightarrow{ \mathrm{d}\ell}$};
	\draw[color=gray] (A) node{\small $\bullet$} .. controls +(1,1) and +(-0.5,0.5) .. (M) node{\small $\bullet$} .. controls +(0.5,-0.5) and +(-2,-1) .. (B) node{\small $\bullet$};
	\end{tikzpicture}
	%-------------------
	
	
	
	%--------- TIKZ ------------
	\begin{tikzpicture} [scale=0.8] 
	\coordinate (A) at (0, 0);
	\coordinate (M) at (3,1);
	\coordinate (B) at (6,-2);  
	\draw[bloc,rotate=-30] (M) ellipse(0.6 and 0.3);
	\draw (A)  node[below=0.2em]{A};
	\draw (M)  node[above=0.2em]{G};
	\draw (B)  node[below=0.2em]{B};
	\draw[force] (M)--++(0,-1.5) node[below]{$\overrightarrow{P}$};
	\draw[color=gray] (A) node{\small $\bullet$} .. controls +(1,1) and +(-0.5,0.5) .. (M) node{\small $\bullet$} .. controls +(0.5,-0.5) and +(-2,-1) .. (B) node{\small $\bullet$};
	\draw[vecteur] (5,1)--++(0,-0.6) node[pos=0.5,right]{$\overrightarrow{g}$};
	\draw[tiret] (-1,0)--++(7,0);
	\draw[tiret] (-1,-2)--++(7,0);
	\draw[<->] (-1,0)--(-1,-2) node[midway,fill=white]{$h$};
	\end{tikzpicture}
	%----------------------------
	
	
	
	%--------- TIKZ ------------
	\begin{tikzpicture} [scale=0.8]
	\begin{scope}[rotate=-30] 
	\draw[vecteur,rotate=20] (2,1) --++(0,-0.6) node[pos=0.5,right]{$\overrightarrow{g}$};
	\draw[color=black,fill=gray,pattern=north east lines] (-2,0)node{\small $\bullet$}.. controls +(1,0) and +(-0.5,0) ..(2,0) .. controls +(0.5,0) and +(-1,-1) .. (4,1)node{\small $\bullet$} --++(0.6,-1.2)--++(-4,-2)--cycle;
	\draw (-2,0)  node[above=0.2em]{A};
	\draw (4,1)  node[above=0.2em]{B};
	\fill[shift={(0.5,0.25)},rotate=-10] (0,0)ellipse(0.12 and 0.25) (0,0.35)circle(0.1);
	\fill (0,0)--(0.5,0) .. controls +(0.2,0) and +(0,-0.2) ..(1,0.25) --++(-1,0)--cycle;
	\draw[force] (0.4,0) --++(-1,0) node[above right]{$\overrightarrow{T}$};
	\draw[force] (0.4,0) --++(0,2) node[above right]{$\overrightarrow{N}$};
	\end{scope}
	\end{tikzpicture}
	
	
	%============================================================
	% 	Date : 25/02/2011  
	%	Exemple de profil énergétique à un puits. Suivant la valeur de l’énergie mécanique, les valeurs de $r$ sont bornées ou pas.
	%============================================================
	\begin{tikzpicture}[scale=1]
	\def \xLabel {$x$}; % legende sur x
	\def \yLabel {$E_{\rm p}$}; % legende sur y
	\def \xmin{0}; 
	\def \xmax{7};
	\def \ymin{-2};
	\def \ymax{3};
	\draw[tiret] (7,0.6)--++(-7,0) node[left]{$E_{m}$};
	\draw[thick] (0.8,0.6)--++(2.05,0) node[pos=0.5, above]{\small \'etats li\'es};
	\draw[thick] (4.5,0.6)--++(2.5,0) node[pos=0.5, above]{\small \'etats non li\'es};
	\draw[thick,color=purple, variable=\x, samples at={0.7,0.75,...,6.5}] plot ({\x}, {5/(\x*\x)-(10*sin(1.1*\x r)*exp(-0.4*\x))/(\x)}) ;
	\draw[dashed] (0.8,0.6)--(0.8,0) node[below,fill=white]{\small $x_1$};
	\draw[dashed] (2.85,0.6)--(2.85,0) node[below,fill=white]{\small $x_2$};
	\draw[dashed] (4.5,0.6)--(4.5,0) node[below,fill=white]{\small $x_3$};
	\draw[->] (\xmin,0) -- (\xmax,0) node[right=2pt] {\xLabel};
	\draw[->] (0,\ymin) -- (0,\ymax) node[below left] {\yLabel};
	\end{tikzpicture}
	
	
	\begin{tikzpicture} [scale=3]
	\coordinate (M) at (0.2,-0.1);
	\coordinate (N) at (0.8,-1);
	\coordinate (G) at (0.36,-0.5);
	\patate[ball color=lightgray];
	\draw[force] (M) -- ++($0.3*(N)-0.3*(M)$) node[right]{$\overrightarrow{f_{ji}}$};
	\draw[force] (M) -- ++(0.3,0.5) node[left]{$\overrightarrow{f_{i}^{\rm ext}}$};
	\draw[force] (N) -- ++($0.3*(M)-0.3*(N)$) node[above right]{$\overrightarrow{f_{ij}}$};
	\draw[force] (N) -- ++(-0.3,-0.2) node[left]{$\overrightarrow{f_{j}^{\rm ext}}$};
	\draw (M) node{$\bullet$}node[right=2pt]{$\textrm{M}_{i}(m_{i})$};
	\draw (N) node{$\bullet$}node[right=2pt]{$\textrm{M}_{j}(m_{j})$};
	\draw (G) node{$\bullet$} node[left=2pt]{$\textrm{G}$};
	\end{tikzpicture}
	
	
\end{document}