\documentclass{officialexam} 
\usepackage[
europeanresistors
]{circuitikz}
\begin{document}
	{\maketitle}\\
	\borderline{ប្រធានទី០១ }
	\begin{enumerate}[I]
		\item គណនាលីមីតនៃអនុគមន៍ខាងក្រោម៖
		\begin{enumerate}[k,3]
			\item $\lim_{x\to0}\frac{\sin x+\sin3x}{\sin4x+\sin5x}$
			\item $\lim_{x\to0}\frac{e^x-\sin x-1}{1-\sqrt{x+1}}$
			\item $\lim_{x\to0}\frac{(2e^x-2)(1-\cos2x)}{x^3}$
		\end{enumerate}
		\item\begin{enumerate}[k]
			\item ដោះស្រាយសមីការ $Z^2-2\sqrt{2}Z+4=0$ ក្នុងសំណុំចំនួនកុំផ្លិច។ រកម៉ូឌុល និងអាគុយម៉ង់នៃឬសនីមួយៗរបស់សមីការនេះ។
			\item សរសេរ $\mathrm{W}=\left(\frac{\sqrt{2}+i\sqrt{2}}{\sqrt{2}-i\sqrt{2}}\right)^2$ ជាទម្រង់ត្រីកោណមាត្រ ។
		\end{enumerate}
		\item \begin{enumerate}[k]
			\item គណនាអាំងតេក្រាល $\mathrm{I}=\int_0^2\left(6x^2-3x-1\right)dx$ និង $\mathrm{J}=\int_{0}^{\frac{\pi}{2}}\left(1-2\sin^2x\right)dx$ ។
			\item គេមាន $f$ កំណត់លើ $\mathbb{R^*}$ ដោយ $f(x)=-2\left(\frac{x+1}{x^2}\right)$ ។ បង្ហាញថា $f(x)=-\frac{2}{x}-\frac{2}{x^2}$ ។\\
			គណនា $\mathrm{K}=\int_{1}^{e}f(x)dx$ ។ $\ln e=1$
		\end{enumerate}
		\item ក្នុងផង់មួយមានប៊ូល $15$ ដែលចែកជាប៊ូលពណ៌បៃតងចំនួន $7$ និងគេសរសេរលើប៊ូលទាំង $7$ នេះតាមលេខរៀងពី $1$ ដល់ $7$ រួចប៊ូលខៀវចំនួន $5$ និងគេសរសេរលើប៊ូលទាំង $5$ នេះតាមរៀងពី $1$ ដល់ $5$ ចុងក្រោយប៊ូលពណ៌ក្រហមចំនួន $3$ និងគេសរសេរលេខលើប៊ូលទាំង $3$ តាមលេខរៀងពីលេខ $1$ ដល់ $3$ ។ គេចាប់យកប៊ូលមួយចេញពីក្នុងផង់ដោយចៃដន្យ។ រកប្រូបាបនៃប្រឹត្តិការណ៍ខាងក្រោម៖ 
		\begin{enumerate}[k]
			\item $\mathrm{A}:$ ប៊ូលដែលចាប់បានមានពណ៌បៃតង
			\item $\mathrm{B}:$ ប៊ូលដែលចាប់បានមានលេខសេស
			\item $\mathrm{C}:$ ប៊ូលដែលចាប់បានមានពណ៌បៃតង និងលេខសេស
		\end{enumerate}
		\item \begin{enumerate}[1]
			\item គេមានសមីការ $18x^2+10y^2=90$ ។
			\begin{enumerate}[k]
				\item បង្ហាញថាសមីការនេះជាសមីការអេលីប ។ រកប្រវែងអ័ក្សធំ ប្រវែងអ័ក្សតូច និងកូអរដោនេនៃកំពូលទាំងពីរ។
				\item សង់អេលីបនេះ ។
			\end{enumerate}
			\item នៅក្នុងតម្រុយអរតូណម៉ាល់ $\left(O, \vec{i}, \vec{j}, \vec{k}\right)$ គេមានចំណុច $M\left(2,3,4\right), N\left(3,5,6\right), P\left(4,6,7\right), Q\left(3,4,5\right)$ ។
			\begin{enumerate}[k]
				\item រកវ៉ិចទ័រ $\overrightarrow{MN}, \overrightarrow{QP}$
				\item ទាញបង្ហាញថាចតុកោណ $MNPQ$ ជាប្រលេឡូក្រាម រួចគណនាផ្ទៃក្រឡានៃចតុកោណកែងនេះ ។
			\end{enumerate}
		\end{enumerate}
		\item \begin{enumerate}[k]
			\item ដោះស្រាយសមីការឌីផេរ៉ង់ស្យែល $(E): y''+2y'-3y=0$
			\item រកចម្លើយពិសេសមួយនៃសមីការឌីផេរ៉ង់ស្យែល $(E)$ ដែល $y(0)=1, y'(1)=e$ ។ ($e$ ជាចំនួនពិតដែល $\ln e=1$)
		\end{enumerate}
		\item គេមានអនុគមន៍ $f$ កំណត់លើ $\mathbb{R}$ ដោយ $f(x)=x+2-\frac{4e^x}{e^x+3}$ ។\\
		គេតាងក្រាបរបស់វាក្នុងប្លង់ប្រដាប់ដោយតម្រុយអរតូណរម៉ាល់ $\left(O,\vec{i}, \vec{j}\right)$
		\begin{enumerate}[1]
			\item \begin{enumerate}[k]
				\item គណនាលីមីតនៃ $f$ ត្រង់ $-\infty$ និង $+\infty$
				\item សិក្សាទីតាំងនៃក្រាប $C$ ធៀបនឹងបន្ទាត់ $d_1$ ដែលមានសមីការ $y=x+2$ ។
			\end{enumerate}
			\item \begin{enumerate}[k]
				\item ស្រាយបញ្ចាក់ថាចំពោះគ្រប់ចំនួនពិត $x, f(x)=\left(\frac{e^x-3}{e^x+3}\right)^2$ ។
				\item សិក្សាអថេរភាពនៃ $f$ លើ $\mathbb{R}$ និងសង់តារាងអថេរភាពនៃ $f$ ។
			\end{enumerate}
			\item \begin{enumerate}[k]
				\item តើគេអាចថាយ៉ាងណាចំពោះបន្ទាត់ប៉ះ $d_2$ ទៅនឹងក្រាប $C$ ត្រង់ចំណុច $I$ ដែលមានអាប់ស៊ីស $\ln3$ ។
				\item សិក្សាទីតាំងនៃក្រាប $C$ ធៀបនឹងបន្ទាត់ $d_2$ ។
			\end{enumerate}
			\item \begin{enumerate}[k]
				\item បង្ហាញថាបន្ទាត់ប៉ះ $d_3$ ទៅនឹងក្រាប $C$ ត្រង់ចំណុចដែលមានអាប់ស៊ីស $0$ មានសមីការ $y=\frac{1}{4}x+1$
				\item ដោយសន្មត់ថាចំណុច $I$ ជាផ្ចិតឆ្លុះនៃក្រាប $C$ និងក្នុងតម្លៃប្រហែលនៃ $\ln3=1.09$ ចូរសង់ក្រាប $C, d_1, d_2, d_3$ នៅក្នុងតម្រុយតែមួយ $\left(O,\vec{i}, \vec{j}\right)$ ដោយកំណត់យក $1$ ឯកតាស្មើ $2cm$។
			\end{enumerate}
		\end{enumerate}
	\end{enumerate}
	\borderline{ចម្លើយ}\\
	{\color{white}.}\dotfill\\
	{\color{white}.}\dotfill\\
	{\color{white}.}\dotfill
	\\
	{\color{white}.}\dotfill\\
	{\color{white}.}\dotfill\\
	{\color{white}.}\dotfill
	\\
	{\color{white}.}\dotfill\\
	{\color{white}.}\dotfill\\
	{\color{white}.}\dotfill
	\\
	{\color{white}.}\dotfill\\
	{\color{white}.}\dotfill\\
	{\color{white}.}\dotfill
	\\
	{\color{white}.}\dotfill\\
	{\color{white}.}\dotfill\\
	{\color{white}.}\dotfill
	\\
	{\color{white}.}\dotfill\\
	{\color{white}.}\dotfill\\
	{\color{white}.}\dotfill
	\newpage
	{\maketitle}\\
	\borderline{ប្រធានទី០២}
	\begin{enumerate}[I]
		\item គេមានចំនួនកុំផ្លិច $Z_1=-1+i\sqrt{3}$ និង $z_2=1-i\sqrt{3}$ ។
		\begin{enumerate}[k]
			\item គណនា $z_1+z_2, z_1-z_2, z_1\times z_2$ និង $\frac{z_1}{z_2}$ ។
			\item សរសេរជាទម្រង់ត្រីកោណមាត្រនៃចំនួនកុំផ្លិច $z_1-z_2, z_1\times z_2$ និង $\frac{z_1}{z_2}$ ។
			\item គណនា $z_1^{2018}+z_2^{2018}$ រួចទាញការសន្និដ្ឋាន។
		\end{enumerate}
		\item គណនាលីមីត 
		\begin{enumerate}[k, 5]
			\item $\lim_{x\to2}\frac{x^3-8}{\sqrt{x+2}-2}$
			\item $\lim_{x\to3}\frac{x^3-27}{\sqrt{x+6}-3}$
			\item $\lim_{x\to0}\frac{\cos x-1}{\sin^2x}$
			\item $\lim_{x\to0}\frac{2\sin3x}{x}$
			\item $\lim_{x\to0}\frac{-5\sin5x}{8x}$
		\end{enumerate}
		\item ក្នុងស្បោងមួយមានប៊ូលពណ៌ស $3$ ពណ៌ខៀវ $3$ និងក្រហម $2$។គេចាប់យកប៊ូលម្ដង $3$ ក្នុងពេលតែមួយចេញពីស្បោងដោយចែដន្យ។ គេសន្និដ្ឋានថាប្រូបាបដែលចាប់បានប៊ូលមួយៗជាសមប្រូបាប។ គណនាប្រូបាបនៃព្រឹត្តិការណ៍ខាងក្រោម៖
		\begin{enumerate}[k]
			\item $\mathrm{A}:$ «យ៉ាងតិចមានប៊ូល $2$ ពណ៌ខៀវ»។
			\item $\mathrm{B}:$ «ប៊ូលទាំង $3$ មានពណ៌ខុសៗគ្នា»។
			\item $\mathrm{C}:$ «ប៊ូល $1$ គត់មានពណ៌ក្រហម»។
		\end{enumerate}
		\item \begin{enumerate}[k]
			\item គណនាអាំងតេក្រាល $\mathrm{I}=\int_{1}^{2}\left(\frac{x^2}{3}-\frac{x}{2}+3\right)dx$ និង $\mathrm{J}=\int_{0}^{\frac{\pi}{2}}\left(\sin4x+\cos2x\right)dx$ ។
			\item គេមានអនុគមន៍ $f(x)=-\frac{2-x}{\left(x-1\right)^2}$ បង្ហាញថា $f(x)=-\frac{1}{\left(x-1\right)^2}+\frac{1}{x-1}$។ គណនា $K=\int_{-1}^{0}f(x)dx$ ។
		\end{enumerate}
		\item \begin{enumerate}[1]
			\item គេមានវ៉ិចទ័រ $\overrightarrow{u}=\vec{i}-\vec{j}+2\vec{k}, \overrightarrow{v}=-\vec{i}+2\vec{j}+2\vec{k}, \overrightarrow{w}=\vec{i}+\vec{j}-2\vec{k}$ ។ រកវ៉ិចទ័រ
			\begin{enumerate}[k,6]
				\item $\overrightarrow{u}+\overrightarrow{v}$
				\item $\overrightarrow{u}-\overrightarrow{v}$
				\item $\overrightarrow{u}\times\overrightarrow{u}$
				\item $\overrightarrow{v}\times\overrightarrow{v}$
				\item $\overrightarrow{u}\times\overrightarrow{v}$
				\item $\overrightarrow{v}\times\overrightarrow{u}$
			\end{enumerate}
			\item រកសមីការស្តង់ដានៃអេលីប ដែលមានកំណុំមួយមានកូអរដោនេ $\left(-1,0\right)$ និងចំណុចកំពូលពីរមានកូអរដោនេ $\left(-3,0\right)$ និង $\left(3,0\right)$។ សង់អេលីបនេះ ។
		\end{enumerate}
		\item គេមានសមីការឌីផេរ៉ង់ស្យែល $(E): y'+2y=2\frac{e^{-x}}{1+2e^x}$ ។
		\begin{enumerate}[k]
			\item ផ្ទៀងផ្ទាត់ថាអនុគមន៍ $f$ ដែល $f(x)=e^{-2x}\ln\left(1+2e^x\right)$ ជាចម្លើយនៃ $(E)$ ។
			\item បង្ហាញថាអនុគមន៍ $\psi$ ជាចម្លើយនៃ $(E)$ លុះត្រាតែ $\left(\psi-f\right)$ ជាចម្លើយនៃសមីការ $(E'): y'+2y=0$ ។
		\end{enumerate}
		\item \begin{enumerate}[A]
			\item គេមានអនុគមន៍ $g$ កំណត់លើ $\left(0, +\infty\right)$ ដោយ $g(x)=x^2+\ln x$ ។
			\begin{enumerate}[1]
				\item \begin{enumerate}[k]
					\item បង្ហាញថា $g$ ជាអនុគមន៍កើនដាច់ខាតលើ $\left(0, +\infty\right)$។
					\item គណនា $g(1)$។
				\end{enumerate}
				\item \begin{enumerate}[k]
					\item ទាញលទ្ធផលពីសំនួរទី១ បញ្ជាក់ថា បើ $x\geq1$ នោះ $x^2+\ln x\geq1$ និងបើ $0<x\leq1$ នោះ $x^2+\ln x\leq1$ ។
					\item កំណត់សញ្ញានៃកន្សោម $x^2+\ln x-1$ កាលណា $x$ នៅចន្លោះ $\left(0,+\infty\right)$ ។
				\end{enumerate}
			\end{enumerate}
			\item គេមានអនុគមន៍ $f$ កំណត់លើ $\left(0, +\infty\right)$ ដោយ $f(x)=x+1-\frac{\ln x}{x}$​និងតាងដោយក្រាប $C$ ក្នុងតម្រុយអរតូណរមេ $\left(O, \vec{i}, \vec{j}\right)$ ។
			\begin{enumerate}[1]
				\item សិក្សាលីមីតនៃអនុគមន៍ $f$ ត្រង់ $0$ និង $+\infty$ (យើងដឹងថា $\lim_{x\to+\infty}\frac{\ln x}{x}=0$) ។
				\item បង្ហាញថាដេរីវេនៃអនុគមន៍ $f$ គឺ $f'(x)=\frac{x^2+\ln x-1}{x^2}$ ។
				\item ប្រើលទ្ធផលនៃសំនួរ $A$ សិក្សាសញ្ញានៃ $f'(x)$​និងសង់តារាងអថេរភាពនៃអនុគមន៍ $f$ លើ $\left(0, +\infty\right)$ ។
				\item \begin{enumerate}[k]
					\item បង្ហាញថាបន្ទាត់ $\bigtriangleup$ មានសមីការ $y=x+1$ ជាអាស៊ីមតូតនៃក្រាប $C$ ត្រង់ $+\infty$ ។
					\item សិក្សាទីតាំង $C$ ធៀបនឹង $\bigtriangleup$ និងបញ្ជាក់ថាកូអរដោនេនៃចំណុចប្រសព្វ $I$ រវាងក្រាប $C$ និង $\bigtriangleup$។ សង់ $\bigtriangleup$ និង ក្រាប $C$។
				\end{enumerate} 
			\end{enumerate}
		\end{enumerate}
	\end{enumerate}
\borderline{ចម្លើយ}\\
{\color{white}.}\dotfill\\
{\color{white}.}\dotfill\\
{\color{white}.}\dotfill
\\
{\color{white}.}\dotfill\\
{\color{white}.}\dotfill\\
{\color{white}.}\dotfill
\\
{\color{white}.}\dotfill\\
{\color{white}.}\dotfill\\
{\color{white}.}\dotfill
\\
{\color{white}.}\dotfill\\
{\color{white}.}\dotfill\\
{\color{white}.}\dotfill
\\
{\color{white}.}\dotfill\\
{\color{white}.}\dotfill\\
{\color{white}.}\dotfill
\\
{\color{white}.}\dotfill\\
{\color{white}.}\dotfill\\
{\color{white}.}\dotfill
\\
{\color{white}.}\dotfill\\
{\color{white}.}\dotfill
\newpage
\maketitle\\
\borderline{ប្រធានទី០៣}

\begin{enumerate}[I]
	\item គណនាលីមីត \begin{enumerate}[k,4]
		\item $\lim_{x\to1}\frac{1-x^3}{x^3-x^2+x-1}$
		\item $\lim_{x\to0}\frac{\sin3x}{-x}$ 
		\item $\lim_{x\to0}\frac{3-3\cos4x}{\sin^2x}$
		\item $\lim_{x\to0}\frac{\sqrt{2+x}-\sqrt{2-x}}{\sin 2x}$
	\end{enumerate}
	\item ក្នុងថ្នាក់រៀនមួយមានសិស្សពូកែ $10$ នាក់ ដែលក្នុងនោះ $4$ នាក់ជាសិស្សស្រី និង $6$ ជាសិស្សប្រុស។ គេរៀបចំសិស្សជាក្រុមក្នុងមួយក្រុមមានសិស្ស $4$ នាក់ដោយចៃដន្យ យកទៅប្រកួតជាមួយក្រុមសិស្សដ៏ទៃ។ រកប្រូបាបនៃព្រឹត្តិការណ៍ខាងក្រោម៖
		\begin{enumerate}[k]
			\item $A:$ «ក្រុមសិស្សដែលជ្រើសរើសបានសុទ្ធតែស្រី»។
			\item $B:$ «ក្រុមសិស្សដែលជ្រើសរើសបានសុទ្ធតែប្រុស»​។
			\item $C:$ «ក្រុមសិស្សដែលជ្រើសរើសបាន $50\%$ ជាសិស្សប្រុស»។
		\end{enumerate}
	\item គេមានចំនួនកុំផ្លិច $z_1=1+\sqrt{3}i$ និង $z_2=6\left(\cos\frac{\pi}{4}+i\sin\frac{\pi}{4}\right)$ ។
		\begin{enumerate}[k]
			\item សរសេរ $z_1$ ជាទម្រង់ត្រីកោណមាត្រ។
			\item រកម៉ូឌុល និងអាគុយម៉ង់នៃ $z_1^3$ ។
			\item សរសេរផលគុណ $z_1\times z_2$ ជាទម្រង់ពីជគណិត។
		\end{enumerate}
	\item 
	\begin{enumerate}[1]
		\item ក្នុងលំហប្រដាប់ដោយតម្រុយ $\left(O,\vec{i},\vec{j},\vec{k}\right)$ គេមានចំណុច $A(-2,1,0),B(0,1,1),C(1,2,2)$ និង $D(0,3,-4)$។
		\begin{enumerate}[k]
			\item រកវ៉ិចទ័រ $\overrightarrow{AB},\overrightarrow{AC},\overrightarrow{AD},\overrightarrow{BC},\overrightarrow{CD}$ ។
			\item គណនាប្រវែង $AB, AC, AD, BC, CD$។ ទាញបញ្ជាក់ថាត្រីកោណ $ABC$ និង $ACD$ កែងត្រង់ $A$។\\ រួចទាញរកផ្ទៃក្រឡានៃត្រីកោណទាំងពីរនេះ ។
		\end{enumerate}
		\item គេមានសមីការ $9y^2-16x^2=144$ ។ បង្ហាញថាសមីការនេះជាសមីការអីុពែបូល។ រកកូអរដោនេកំពូលទាំងពីរ និងកុំណុំទាំងពីរនៃអីុពែបូល ។
		រកសមីការអាស៊ីមតូតរបស់អីុពែបូល និងសង់អីុពែបូលនេះ ។
	\end{enumerate}
	\item \begin{enumerate}[1]
		\item គណនាអាំងតេក្រាល $\mathrm{I}=\int_{1}^{3}\left(x-2+3x^3\right)dx$ និង $\mathrm{J}=\int_{0}^{\frac{\pi}{4}}\left(\sin2x-\cos x\right)dx$។ 
		\item  គេមានអនុគមន៍ $\mathrm{K}=\int_{0}^{1}\frac{x^3+\left(x+1\right)^2}{x^2+1}dx$ កំណត់លើ $\mathbb{R}$។\\ 
		ដើម្បីគណនា $\mathrm{K}$ យើងត្រូវបង្ហាញថា $\frac{x^3+\left(x+1\right)^2}{x^2+1}=x+1+\frac{x}{x^2+1}$ ។
	\end{enumerate}
	\item \begin{enumerate}[k]
		\item ដោះស្រាយសមីការឌីផេរ៉ង់ស្យែល $(E): y''-3y'+2y=0$។
		\item រកចម្លើយពិសេសមួយនៃសមីការឌីផេរ៉ង់ស្យែល $(E)$ ដែល $y(0)=1$ និង $y'(1)=e^2$ ។
	\end{enumerate}
	\item គេមានអនុគមន៍ $f$ កំណត់លើ $\mathbb{R}$ ដោយ $f(x)=x+\frac{1-3e^x}{1+e^x}$ គេតាងដោយ $C$ ក្រាបរបស់វានៅក្នុងប្លង់ប្រដាប់ដោយតម្រុយអរតូណរម៉ាល់ $\left(O, \vec{i}, \vec{j}\right)$ ។
	\begin{enumerate}[1]
		\item បង្ហាញថា $f(x)=x+1-\frac{4e^x}{1+e^x}$ និងគណនាលីមីតនៃ $f$ ត្រង់ $-\infty$។ ស្រាយបំភ្លឺថាបន្ទាត់ $d_1$ ដែលមានសមីការ $y=x+1$\\ អាស៊ីមតូតទៅនឹងក្រាប $C$ ត្រង់ $-\infty$។ សិក្សាទីតាំងនៃក្រាប $C$ ធៀបនឹងបន្ទាត់​ $d_1$ ។
		\item គណនាលីមីតនៃ $f$ ត្រង់ $+\infty$។ ស្រាយបំភ្លឺថាបន្ទាត់ $d_2$ ដែលមានសមីការ $y=x-3$\\ អាស៊ីមតូតទៅនឹងក្រាប $C$ ត្រង់ $+\infty$។ សិក្សាទីតាំងនៃក្រាប $C$ ធៀបនឹងបន្ទាត់​ $d_2$ ។
		\item 
		\begin{enumerate}[k]
			\item គណនាដេរីវេ $f'(x)$ និងបង្ហាញថាគ្រប់ចំនួនពិត $x, f(x)=\left(\frac{e^x-1}{e^x+1}\right)^2$ ។
			\item សិក្សាអថេរភាពនៃ $f$ រួចសង់តារាងអថេរភាពនៃ $f$។ សង់ក្រាប $C$ និងបន្ទាត់ $d_1, d_2$ របស់វាក្នុងតម្រុយតែមួយ ។
		\end{enumerate}
	\end{enumerate}
\end{enumerate}
\borderline{ចម្លើយ}\\
{\color{white}.}\dotfill\\
{\color{white}.}\dotfill\\
{\color{white}.}\dotfill
\\
{\color{white}.}\dotfill\\
{\color{white}.}\dotfill\\
{\color{white}.}\dotfill
\\
{\color{white}.}\dotfill\\
{\color{white}.}\dotfill\\
{\color{white}.}\dotfill
\\
{\color{white}.}\dotfill\\
{\color{white}.}\dotfill\\
{\color{white}.}\dotfill
\\
{\color{white}.}\dotfill\\
{\color{white}.}\dotfill\\
{\color{white}.}\dotfill
\\
{\color{white}.}\dotfill\\
{\color{white}.}\dotfill\\
{\color{white}.}\dotfill
\\
{\color{white}.}\dotfill\\
{\color{white}.}\dotfill\\
{\color{white}.}\dotfill
\\
{\color{white}.}\dotfill\\
{\color{white}.}\dotfill\\
{\color{white}.}\dotfill
\\
{\color{white}.}\dotfill\\
{\color{white}.}\dotfill

\newpage
	{\maketitle}\\
	\borderline{ប្រធានទី០៤}
	\begin{enumerate}[I]
		\item {\color{khtug}(១៥ ពិន្ទុ)} គណនាលីមីត៖
		\begin{enumerate}[k,4]
			\item $\lim_{x\to1}\frac{x^2-4x+3}{1-x^2}$
			\item $\lim_{x\to\frac{\pi}{2}}\frac{\cos^2x}{\sqrt{2}-\sqrt{1+\sin x}}$
			\item $\lim_{x\to0}\frac{1-\cos^32x}{x\sin3x}$
			\item $\lim_{x\to1}\frac{6x-6}{x^2+3x-4}$
		\end{enumerate}
		\item {\color{khtug}(១៥ ពិន្ទុ)} គេមានចំនួនកុំផ្លិច $z=1+i\sqrt{3}$ និង $w=\sqrt{2}\left(\cos\frac{\pi}{12}+i\sin\frac{\pi}{12}\right)^3$
		\begin{enumerate}[k]
			\item ចូរផ្ទៀងផ្ទាត់ថា $z$ ជាប្ញសនៃសមីការ $z^2-2z+4=0$ រួចទាញរកប្ញសមួយទៀតនៃសមីការនេះ ។
			\item ចូរសរសេរប្ញសទាំងពីរនៃសមីការ $z^2-2z+4=0$ និង $w$ ជាចំនួនកុំផ្លិចទម្រង់ត្រីកោណមាត្រ ។ 
			\item ចូរសរសេរ $w$ ជាចំនួនកុំផ្លិចទម្រង់ពីជគណិត រួចស្រាយបញ្ជាក់ថា $\frac{z}{w}=\frac{\sqrt{3}+1}{2}+i\frac{\sqrt{3}-1}{2}$ ។
		\end{enumerate}
		\item {\color{khtug}(១៥ ពិន្ទុ)} ក្នុងប្រអប់មួយមានប៊ូល ៥ ដោយក្នុងនោះមានប៊ូលពណ៌ខ្មៅ ៣ ត្រូវបានគេចុះលេខពី ១ ដល់ ៣ និងប៊ូលពណ៌ស ២ ត្រូវបានគេចុះលេខពី ១ ដល់ ២ ។ គេចាប់យកប៊ូល ២ ព្រមគ្នាក្នុងពេលតែមួយដោយចៃដន្យចេញពីក្នុងប្រអប់នោះ ។ គណនាប្រូបាបនៃព្រឹត្តិការណ៍ដូចខាងក្រោម៖ 
		\begin{enumerate}[k]
			\item $A$ $ :"$គេចាប់បានប៊ូលមានពណ៌ដូចគ្នា$"$
			\item $B$ $ :"$គេចាប់បានប៊ូលដែលមានផលបូកលេខស្មើ ៣$"$ 
			\item $C$ $ :"$គេចាប់បានប៊ូលដែលមានផលបូកលេខស្មើ ៣ ដោយដឹងថាវាមានពណ៌ដូចគ្នា$"$ 
		\end{enumerate}
		\item \begin{enumerate}[1]
			\item {\color{khtug}(១០ ពិន្ទុ)} គណនាអាំងតេក្រាល៖ $\mathrm{I}=\int_{1}^{2}\left(\frac{x^2}{2}+x-3\right)dx$ ~~និង $\mathrm{J}=\int_{0}^{\frac{\pi}{4}}\left(\frac{1-\sin4x}{4x+\cos4x}\right)dx$ ។
			\item {\color{khtug}(៥ ពិន្ទុ)} គេមានអនុគមន៍ $f(x)=-\frac{4-x}{\left(x-3\right)^2}$ កំណត់ចំពោះគ្រប់ $x\neq3$ បង្ហាញថា $f(x)=\frac{1}{x-3}-\frac{1}{\left(x-3\right)^2}$ ។\\ គណនា $\mathrm{K}=\int_{0}^{2}f(x)dx$ ។ 
		\end{enumerate}
		\item \begin{enumerate}[k]
			\item{\color{khtug}(៥ ពិន្ទុ)} គេមានសមីការឌីផេរ៉ង់ស្យែល $(E): y''-3y'+2y=0$ ។
			\item{\color{khtug}(៥ ពិន្ទុ)} រកចម្លើយពិសេសមួយនៃ $(E)$ ដោយដឹងថាក្រាបនៃចម្លើយរបស់វាប៉ះទៅនឹងបន្ទាត់ដេក $y=1$ ត្រង់ $x=0$ ។
		\end{enumerate}
		\item \begin{enumerate}[1]
			\item{\color{khtug}(១០ ពិន្ទុ)} ក.គេឲ្យខ្សែកោង $(E): \frac{\left(x-4\right)^2}{25}+\frac{y^2}{9}=1$ ។ បញ្ជាក់ប្រភេទនៃខ្សែកោង $(E)$ ។\\
			ខ. កំណត់កូអរដោនេ ផ្ចិត កំពូល កំណុំ ប្រវែងអ័ក្សធំ និងប្រវែងអ័ក្សតូចនៃ $(E)$ រូចសង់ខ្សែកោង $(E)$ ។
			\item {\color{khtug}(១០ ពិន្ទុ)} នៅក្នុងតម្រុយអរតូណរម៉ាល់មានទិសទៅវិជ្ជមាន $\left(O,\vec{i}, \vec{j}, \vec{k}\right)$ គេមានចំណុចបី $A(1,2,1),B(4,2,4),C(5,3,0)$ ។
			\begin{enumerate}[k]
				\item រកប្រវែង $AB, AC, BC$ រួចធ្វើការសន្និដ្ឋាននៃប្រភេទត្រីកោណ $ABC$ ។
				\item គណនាផលគុណ $\overrightarrow{AB}\times\overrightarrow{AC}$ រួចគណនាផ្ទៃក្រឡានៃត្រីកោណ $ABC$ ។
			\end{enumerate}
		\end{enumerate}
		\item {\color{khtug}(៣៥ ពិន្ទុ) \sffamily ផ្នែក $A$} គេមានអនុគមន៍ $g$ កំណត់លើ $(0, +\infty)$ ដោយ $g(x)=-x^2+1-2\ln x$ ។
		\begin{enumerate}[k]
			\item ចូរគណនាដេរីវេ $g'(x)$ រួចទាញថាអនុគមន៍ $g$ ជាអនូកមន៍ចុះជានិច្ចលើចន្លោះ $(0,+\infty)$ ។ 
			\item ចូរគណនាតម្លៃ $g(1)$ ។ ចូរបញ្ជាក់សញ្ញានៃ $g(x)$ លើ $(0, +\infty)$ ។
		\end{enumerate}
		{\color{khtug} \sffamily ផ្នែក $B$} គេឲ្យអនុគមន៍ $f$ កំណត់លើចន្លោះ $(0, +\infty)$ ដោយ $f(x)=\frac{9x^2+6\ln x-1}{2x^3}$ មានក្រាបតាង $C$ ។
		\begin{enumerate}[k]
			\item ចូររកលីមីតនៃ $f$ ត្រង់ $0$ និង $+\infty$ ។ ទាញរកសមីការអាស៊ីមតូតឈរ និងដេកនៃក្រាប $C$ ។
			\item ចូរស្រាយបញ្ជាក់ថាចំពោះគ្រប់ $x>0$ គេបាន $f'(x)=kg(x)$ ដែល $k$ ជាចំនួនពិតត្រូវកំណត់ ។​\\ គូសតារាងអថេរភាពនៃ $f$ ដោយប្រើលទ្ធផលផ្នែក $A$ ។
			\item ចូរគណនា $f\left(\frac{1}{2}\right)$ រួចទាញថា $f(x)=0$ មានប្ញសតែមួយគត់ស្ថិតនៅចន្លោះ $\left[\frac{1}{2},1\right]$
			\item កំណត់សមីការបន្ទាត់ $T$ ប៉ះទៅនឹងក្រាប $C$ ត្រង់អាប់ស៊ីស $1$ ។
			\item គណនា $f(2)$ រួចសង់ក្រាប $C$ និងបន្ទាត់ $T$ ក្នុងតម្រុយអរតូណរម៉ាល់ $(O,\vec{i}, \vec{j})$ ។
		\end{enumerate}
	\end{enumerate}
\borderline{ចម្លើយ}\\
{\color{white}.}\dotfill\\
{\color{white}.}\dotfill\\
{\color{white}.}\dotfill
\\
{\color{white}.}\dotfill\\
{\color{white}.}\dotfill\\
{\color{white}.}\dotfill
\\
{\color{white}.}\dotfill\\
{\color{white}.}\dotfill\\
{\color{white}.}\dotfill
\\
{\color{white}.}\dotfill\\
{\color{white}.}\dotfill\\
{\color{white}.}\dotfill
\\
{\color{white}.}\dotfill\\
{\color{white}.}\dotfill\\
{\color{white}.}\dotfill
\\
{\color{white}.}\dotfill\\
{\color{white}.}\dotfill\\
{\color{white}.}\dotfill
\\
{\color{white}.}\dotfill\\
{\color{white}.}\dotfill\\
{\color{white}.}\dotfill
\\
{\color{white}.}\dotfill\\
{\color{white}.}\dotfill
\newpage
{\maketitle}\\
\borderline{ប្រធានទី០៥}
\begin{enumerate}[I]
	\item ក្នុងថង់មួយមានឃ្លីពណ៌សចំនួន $2$ ឃ្លីពណ៌ក្រហមចំនួន $4$ និងឃ្លីពណ៌ខៀវចំនួន $4$។ គេចាប់យកឃ្លី $3$ព្រមគ្នាដោយចៃដន្យ។\\
	រកប្រូបាបនៃព្រឹត្តិការណ៍:\\ $A$: ឃ្លីទាំង $3$មានពណ៌ក្រហម;\quad $B$: យ៉ាងតិចមានឃ្លី $2$ មានពណ៌ខៀវ; \quad $C$: ឃ្លីទាំង $3$ មានពណ៌ខុសៗគ្នា។
	\item គណនាលីមីត
		\begin{enumerate}[k,3]
			\item $\lim_{x\to1}\frac{x^2\left(x-2\right)+x^2+x-1}{1-x}$
			\item $\lim_{x\to0}\frac{-2x}{\sin3x}$
			\item $\lim_{x\to\frac{\pi}{3}}\frac{\sin x-\sqrt{3}\cos x}{2\left(\pi-3x\right)}$ 
		\end{enumerate}
	\item គេមានចំនួនកុំផ្លិច $z_1=3+3i\sqrt{3}$ និង $z_2=\sqrt{3}+i$ ។
	\begin{enumerate}[k,2]
		\item គណនា $z_1\times z_2$ និង$\frac{z_1}{z_2}$
		\item សរសេរ $z_1\times z_2$ និង $\left(\frac{z_1}{z_2}\right)^{2}$ ជាទម្រង់ត្រីកោណមាត្រ
		\item សរសេរ $\left(\frac{z_1}{z_2}\right)^{3}$ ជាទម្រង់ពីជគណិត។
	\end{enumerate}
	\item គណនាអាំងតេក្រាល:\quad $I=\int_{1}^{2}\left(2-x+x^2\right)dx$;\quad $J=\int_{0}^{\frac{\pi}{4}}\left[\cos2x-\frac{1}{2}\cos4x\right]dx$;\quad $K=\int_{2}^{3}\left(3x-2+\frac{1}{x-1}\right)dx$។
	\item \begin{enumerate}[1]
		\item ក្នុងលំហប្រដាប់ដោយតម្រុយអរតូណរម៉ាល់ $\left(o,\vec{i},\vec{j},\vec{k}\right)$ គេមានចំណុច $A\left(1;2;3\right),~B\left(3;0;1\right),~C\left(-1;0;1\right)$ និង $D\left(2;1;2\right)$។
		\begin{enumerate}[a]
			\item រកវិុចទ័រ $\overrightarrow{AB},\overrightarrow{AC},\overrightarrow{AD},\overrightarrow{BC}$។
			\item បង្ហាញថាចំណុច $A,B$ និង $C$ មិននៅលើបន្ទាត់តែមួយ។
			\item បង្ហាញថាវុិចទ័រ $n\left(0;1;-1\right)$ ជាវុិចទ័រណរម៉ាល់ទៅនឹងប្លង់ $\left(ABC\right)$។
		\end{enumerate}
		\item គេមានសមីការ $\left(2x+3y\right)^2=12\left(xy+3\right)$។\\
		បង្ហាញថាសមីការនេះជាសមីការអេលីប។ រកប្រវែងអ័ក្សតូច អ័ក្សធំ កូអរដោនេនៃកំពូលទាំងពីរ និងសង់អេលីបនេះ។
	\end{enumerate}
	\item \begin{enumerate}[a]
		\item ដោះស្រាយសមីការឌីផេរ៉ង់ស្យែល: $\left(E\right):y''+4y'=5y$។
		\item រកចម្លើយពិសេសមួយនៃសមីការឌីផេរ៉ង់ស្យែល​ $\left(E\right)$ បើគេដឹងថាក្រាប $\left(C\right)$ នៃអនុគមន៍ចម្លើយនេះកាត់តាមចំណុច $\left(0;3\right)$ ហើយបន្ទាត់ប៉ះទៅនឹងក្រាប $\left(C\right)$ ត្រង់ចំណុចនេះមានមេគុណប្រាប់ទិសស្មើ $-3$។
	\end{enumerate}
	\item គេមានអនុគន៍ $f$ កំណត់លើ $\left(1;+\infty\right)$ ដោយ $f(x)=-x+4+\ln\left(\frac{x+1}{x-1}\right)$។ គេតាងដោយ $\left(C\right)$ ក្រាបរបស់វានៅក្នុងប្លង់ប្រដាប់ដោយតម្រុយអរតូណរម៉ាល់ $\left(o;\vec{i};\vec{j}\right)$។
	\begin{enumerate}[1]
		\item គណនាលីមីតនៃ $f$ ត្រង់ $1$ និងត្រង់ $+\infty$។
		\item ស្រាយបំភ្លឺថានៅលើ $\left(1;+\infty\right)$ គេបានដេរីវេនៃអនុគមន៍ $f$ គឺ $f'(x)=\frac{-\left(x^2+1\right)}{\left(x+1\right)\left(x-1\right)}$។ សិក្សាអថេរភាពនៃអនុគមន៍ $f$ និងសង់តារាងអថេរភាពនៃ $f$ លើ $\left(1;+\infty\right)$។
		\item \begin{enumerate}[a]
			\item បង្ហាញថាបន្ទាត់ $d_1$ ដែលមានសមីការ $y=-x+4$ ជាអាសុីមតូតទៅនឹងក្រាប $\left(C\right)$ ត្រង់ $+\infty$។
			\item បង្ហាញថាចំពោះគ្រប់ $x$ លើ $\left(1;+\infty\right)~\frac{x+1}{x-1}>1$ និងទាញយកការប្រៀបធៀបទីតាំងនៃ $\left(C\right)$ ធៀបនឹង $d_1$។ 
		\end{enumerate}\newpage
		\item កំណត់កូអរដោនេនៃចំណុចនៅលើ $\left(C\right)$ ដែលបន្ទាត់ប៉ះ $d_2$ ទៅនឹងក្រាប $\left(C\right)$ ត្រង់ចំណុចនេះមានមេគុណប្រាប់ទិស $-\frac{5}{3}$ និងសរសេរសមីការបន្ទាត់ប៉ះ $d_2$ នេះ។
		\item សង់ក្រាប $\left(C\right)$ អាសុីមតូត $d_1$ និងបន្ទាត់ប៉ះ $d_2$។ ប្រើតម្លៃប្រហែល $\ln3=1.1$ និងក្រាប $\left(C\right)$ កាត់អ័ក្សអាប់សុីសត្រង់ចំណុច $\left(4.5;0\right)$។
	\end{enumerate}
\end{enumerate}
\borderline{ចម្លើយ}\\
{\color{white}.}\dotfill\\
{\color{white}.}\dotfill\\
{\color{white}.}\dotfill
\\
{\color{white}.}\dotfill\\
{\color{white}.}\dotfill\\
{\color{white}.}\dotfill
\\
{\color{white}.}\dotfill\\
{\color{white}.}\dotfill\\
{\color{white}.}\dotfill
\\
{\color{white}.}\dotfill\\
{\color{white}.}\dotfill\\
{\color{white}.}\dotfill
\\
{\color{white}.}\dotfill\\
{\color{white}.}\dotfill\\
{\color{white}.}\dotfill
\\
{\color{white}.}\dotfill\\
{\color{white}.}\dotfill\\
{\color{white}.}\dotfill
\\
{\color{white}.}\dotfill\\
{\color{white}.}\dotfill\\
{\color{white}.}\dotfill
\\
{\color{white}.}\dotfill\\
{\color{white}.}\dotfill
\newpage
{\maketitle}\\
\borderline{ប្រធានទី០៦}
\begin{enumerate}[I]
	\item {\color{khtug}(\kml{១៥ ពិន្ទុ})} គេឲ្យចំនួនកុំផ្លិច $z=\left(1+i\right)^{10}\left(\cos\frac{\pi}{3}+i\sin\frac{\pi}{3}\right)^{20}$។
	\begin{enumerate}[k,2]
		\item សរសេរ $z$ ជាទម្រង់ពីជគណិត និងទម្រង់ត្រីកោណមាត្រ។
		\item កំណត់ចំនួនពិត $x$ និង $y$ ដើម្បីឲ្យ $xz+y\overline{z}=\sqrt{3}$។
	\end{enumerate}
	\item {\color{khtug}(\kml{១៥ ពិន្ទុ})} គណនាលីមីតនៃអនុគមន៍ខាងក្រោមៈ
	\begin{enumerate}[k,3]
		\item $\lim_{x\to1}\frac{x+2\sqrt{x}-3}{x-5\sqrt{x}+4}$
		\item $\lim_{x\to\frac{\pi}{3}}\frac{\sqrt{3}\cos x -\sin x}{6x-2\pi}$
		\item $\lim_{x\to1}\frac{x-1}{x^2+x-1+x^2\left(x-2\right)}$
	\end{enumerate}
	\item {\color{khtug}(\kml{១៥ ពិន្ទុ})} ក្នុងថង់មួយមានប៊ិចបាល់ពណ៍ស $5$ ដើម កូនប៊ិចពណ៍ខៀវ $3$ ដើម និងកូនប៊ិចពណ៍ក្រហម $4$ ដើម។ គេចាប់យកប៊ិច $3$ ចេញព្រមគ្នាដោយចៃដន្យ។ រកប្រូបាបនៃព្រឹត្តិការណ៍ខាងក្រោមៈ
	\begin{enumerate}[k,2]
		\item $A:$ "ចាប់បានប៊ិចទាំង $3$ មានពណ៍ដូចគ្នា"
		\item $B:$ "ចាប់បានប៊ិចពណ៍ខៀវ $2$ គត់"
		\item $C:$ "ចាប់បានប៊ិចទាំង $3$ មានពណ៍ខុសគ្នា"។
	\end{enumerate}
	\item {\color{khtug}(\kml{១៥ ពិន្ទុ})} គណនាអាំងតេក្រាលនៃអនុគមន៍ខាងក្រោមៈ
	\begin{enumerate}[m]
		\item \begin{enumerate}[k,3]
			\item $\int_{0}^{\frac{\pi}{4}}\frac{1+\sin2x}{\sin x + \cos x}dx$
			\item $\int_{\frac{\pi}{6}}^{\frac{\pi}{3}}\frac{1+\cos2x}{1-\cos^22x}dx$
			\item $\int_{0}^{\frac{\pi}{2}}\left(x^2+1\right)\sin2xdx$
		\end{enumerate}
		\item គេឲ្យអនុគមន៍ $f$ កំណត់ដោយ $f(x)=\frac{x+1}{\left(x-1\right)^2}$ ចំពោះ $x\ne 1$។\\ កំណត់ចំនួនពិត $a$ និង $b$ ដើម្បីឲ្យ $f(x)=\frac{a}{x-1}+\frac{b}{\left(x-1\right)^2}$ ចំពោះ $x\ne 1$ រួចគណនា $K=\int_{0}^{2}f(x)dx$។
	\end{enumerate}
	\item {\color{khtug}(\kml{១៥ ពិន្ទុ})} គេឲ្យសមីការឌីផេរ៉ង់ស្យែល $(E): y'-2y=4\cos x$។
	\begin{enumerate}[k]
		\item កំណត់ចម្លើយទូទៅនៃសមីការឌីផេរ៉ង់ស្យែល $(F): y'-2y=0$ ដែលផ្ទៀងផ្ទាត់ $y(0)=1$។ 
		\item កំណត់ចំនួនពិត $a$ និង $b$ ដែលអនុគមន៍ $g$ កំណត់លើ $\mathbb{R}$ ដោយ $g(x)=a\cos x +b\sin x$ ផ្ទៀងផ្ទាត់ $(E)$។
		\item ទាញរកចម្លើយទួទៅនៃសមីការឌីផេរ៉ង់ស្យែល $(E)$។
	\end{enumerate}
	\item \begin{enumerate}[m]
		\item {\color{khtug}(\kml{១០ ពិន្ទុ})} គេឲ្យសមីការនៃកោនិច $(P): \left(1-2x\right)^2-8\left(y+1\right)+7+4x=0$។ បង្ហាញថា $(P)$ ជាប៉ារ៉ាបូល។\\ កំណត់កូអរដោនេកំពូល កំណុំ បន្ទាត់ប្រាប់ទិស និងសង់ក្រាបវា។
		\item {\color{khtug}(\kml{១០ ពិន្ទុ})} ក្នុងតម្រុយអរតូណរម៉ាល់ប្រដាប់ដោយទិសដៅវិជ្ជមាន $\left(o, \vec{i}, \vec{j}, \vec{k}\right)$ គេមានចំណុច $A\left(-1;2;1\right),~B\left(-1;2;3\right)$ និង $\overrightarrow{BC}=3\vec{i}+\vec{j}+\vec{k}$។
		\begin{enumerate}[k]
			\item កំណត់កូអរដោនេនៃចំណុច $C$។ គណនាផលគុណស្កាលែ $\overrightarrow{BC}\cdot\overrightarrow{BA}$ និងទាញរកកូសុីនុសនៃមុំ $B$។
			\item គណនាផលគុណវុិចទ័រ $\overrightarrow{BC}\times\overrightarrow{BA}$ និងកំណត់សមីការប្លង់កាត់តាម $A;B;C$។
		\end{enumerate}
	\end{enumerate}
	\item {\color{khtug}(\kml{៣៥ ពិន្ទុ})} {\color{khtug}\kml{ផ្នែក $A:$}} អនុគមន៍ $g$ កំណត់លើ $\mathbb{R}$ ដោយ $g(x)=\left(2-x\right)e^{-x}-2$។
	\begin{enumerate}[k]
		\item គណនា $\lim_{x\to-\infty}g(x)$ និង $\lim_{x\to+\infty}g(x)$។
		\item គណនាដេរីវ៉េ $g'(x),~g(0)$ រួចគូសតារាងអថេរភាពនៃ $g(x)$។ ទាញរកសញ្ញានៃ $g$ ទៅតាមតម្លៃនៃ $x$។
	\end{enumerate}
	{\color{khtug}\kml{ផ្នែក $B:$}} អនុគមន៍ $f(x)=\frac{2e^x+1}{\left(x-3\right)e^x}$ មានក្រាប $C$ ក្នុងតម្រុយអរតូណរមេ $\left(o,\vec{i}, \vec{j}\right)$។
	\begin{enumerate}[k]
		\item រកដែនកំណត់ $\mathbb{D}៌_f$ នៃអនុគមន៍ $f$ រួចគណនាលីមីតនៃ $f$ ត្រង់ចុងដែនកំណត់។\\ ទាញរកសមីការអាសុីមតូតទាំងអស់នៃក្រាប $C$។
		\item ចំពោះគ្រប់ $x\in \mathbb{D}_f$ គេបាន $f'(x)=\frac{\left(2-x\right)e^{-x}-2}{\left(x-3\right)^2e^{2x}}$។ រួចបង្ហាញថា $f'(x)$ មានសញ្ញាដូច $g(x)$។
		\item គូសតារាងអថេរភាពនៃ $f$ និងសង់ក្រាប $C$។ 
	\end{enumerate}
\end{enumerate}
\borderline{ចម្លើយ}\\
{\color{white}.}\dotfill\\
{\color{white}.}\dotfill\\
{\color{white}.}\dotfill
\\
{\color{white}.}\dotfill\\
{\color{white}.}\dotfill\\
{\color{white}.}\dotfill
\\
{\color{white}.}\dotfill\\
{\color{white}.}\dotfill\\
{\color{white}.}\dotfill
\\
{\color{white}.}\dotfill\\
{\color{white}.}\dotfill\\
{\color{white}.}\dotfill
\\
{\color{white}.}\dotfill\\
{\color{white}.}\dotfill\\
{\color{white}.}\dotfill
\\
{\color{white}.}\dotfill\\
{\color{white}.}\dotfill\\
{\color{white}.}\dotfill
\\
{\color{white}.}\dotfill\\
{\color{white}.}\dotfill\\
{\color{white}.}\dotfill
\\
{\color{white}.}\dotfill\\
{\color{white}.}\dotfill
\begin{center}
	\sffamily\color{blue}
	សូមសំណាងល្អ!
\end{center}\newpage
{\maketitle}\\
\borderline{ប្រធានទី០៧}
\begin{enumerate}[I]
	\item {\color{khtug}(\kml{១៥ ពិន្ទុ})} គណនាលីមីតៈ $A=\lim_{x\to0}\left(\frac{3x^2-2x+1}{x}-\frac{1}{x}\right),\quad B=\lim_{x\to0}\frac{2x^2-\sin3x}{\sin^2x+4x},\quad C=\lim_{x\to\frac{\pi}{6}}\frac{\sqrt{3}\sin x-\cos x}{\frac{\pi}{6}-x}$
	\item {\color{khtug}(\kml{១៥ ពិន្ទុ})} គណនាអាំងតេក្រាលៈ​ $I=\int \left(\frac{4x-5x^2+6x^3}{x^3}\right)dx,\quad J=\int_{0}^{1}\left(\frac{1}{x^2-4x+4}\right)dx,\quad K=\int_{0}^{\frac{\pi}{2}}\left(\sin x \sqrt{1-\cos x}\right)dx$
	\item {\color{khtug}(\kml{១៥ ពិន្ទុ})} គេមានចំនួនកុំផ្លិច $z_1=2\sqrt{3}+2i$ និង $z_2=3+3\sqrt{3}i$។ \quad {\color{khtug}ក.}គណនា $z_1\times z_2$ និង $\frac{z_1}{z_2}$។\\
	{\color{khtug}ខ.}សរសេរស $z_1\times z_2$ និង $\frac{z_1}{z_2}$ ជាទម្រង់ត្រីកោណមាត្រ។ $\quad$ {\color{khtug}គ.}បង្ហាញថា $x=\left(\frac{9i}{4}\right)^{1008}\left(\frac{z_1}{z_2}\right)^{2016}$ ជាចំនួនពិត។
	\item {\color{khtug}(\kml{១០ ពិន្ទុ})} {\color{khtug}ក.} ដោះស្រាយសមីការឌីផេរ៉ង់ស្យែល $\left(E\right):~y''+3y'=4y$ ។\\
	{\color{khtug}ខ.}រកចម្លើយពិសេសមួយនៃសមីការឌីផេរ៉ង់ស្យែល $\left(E\right)$ ដោយដឹងថាក្រាបតាងអនុគមន៍ចម្លើយប៉ះទៅនឹងបន្ទាត់ $T$ ដែលមានសមីការ $y+4x=0$ ត្រង់ចំណុច $A\left(0,6\right)$។ 
	\item {\color{khtug}(\kml{១០ ពិន្ទុ})} ក្នុងថង់មួយមានឃ្លីក្រហមចំនួន $3$ ឃ្លីសចំនួន $2$ និងឃ្លីខ្មៅចំនួន $4$។ គេចាប់យកឃ្លីម្តងមួយៗចំនួន $3$ ចេញមិនដាក់ចូលវិញដោយចៃដន្យ។ គណនាប្រូបាបនៃព្រឹត្តិការណ៍ដែលចាប់បានៈ\\
	$A:$ ឃ្លីទាំង $3$ ជាឃ្លីពណ៌ដូចគ្នា,\quad $B:$ ឃ្លីទី $1$ និង $2$ ជាឃ្លីពណ៌ស, \quad $C:$ ឃ្លីទាំង $3$ ជាឃ្លីពណ៌ខុសគ្នា
	\item {\color{khtug}(\kml{៣៥ ពិន្ទុ})} ក្នុងតម្រុយ $\left(o,\vec{i},\vec{j}\right)$ អនុគមន៍ $f$ កំណត់ $x\in\left(-1,+\infty\right)$ ដោយ $f(x)=x+\frac{\ln\left(1+x\right)}{1+x}$ មានក្រាប $\left(C\right)$។ 
	\begin{enumerate}[m]
		\item អនុគមន៍ $g$ កំណត់ $x\in\left(-1,+\infty\right)$ ដោយ $g(x)=\left(1+x\right)^2-1+\ln\left(1+x\right)$ ។
		\begin{enumerate}[k]
			\item សិក្សាអថេរភាពនៃអនុគមន៍ $g(x)$។ គណនា $g(0)$។
			\item សង់តារាងអថេរភាពនៃអនុគមន៍ $g$ ដោយមិនចាំបាច់គណនាលីមីត រួចសិក្សាសញ្ញា $g(x), x\in\left(-1,+\infty\right)$។
		\end{enumerate}
		\item\begin{enumerate}[k]
			\item គណនាលីមីតចុងដែនកំណត់នៃអនុគមន៍ $f$ ដោយប្រើ $\lim_{x\to0^{+}}\frac{\ln x}{x}=-\infty$ និង $\lim_{x\to+\infty}\frac{\ln x}{x}=0$។\\ រួចទាញបញ្ជាក់សមីការអាសុីមតូតនៃក្រាប $\left(C\right)$។
			\item ផ្ទៀងផ្ទាត់ថា $f'(x)=\frac{g(x)}{\left(1+x\right)^2},~x\in\left(-1,+\infty\right)$ រួចសិក្សាសញ្ញា $f'(x)$។ សង់តារាងអថេរភាពនៃអនុគមន៍ $f$។
			\item បង្ហាញថាបន្ទាត់ $L$ មានសមីការ $y=x$ ជាសមីការអាសុីមតូតទ្រេតនៃក្រាប $\left(C\right)$ ខាង $+\infty$ រួចសិក្សាទីតាំងធៀប។ 
			\item សង់ក្រាប $\left(C\right)$ និង $L$ ក្នុងតម្រុយតែមួយ។ \\គណនាផ្ទៃក្រឡាផ្នែកនៃប្លង់ខណ្ឌដោយក្រាប $\left(C\right)$ និង $L$ បន្ទាត់ឈរ $x=1$ និង $x=3$។
		\end{enumerate}
	\end{enumerate}
	\item {\color{khtug}(\kml{២៥ ពិន្ទុ})} {\color{khtug}១.} ក្នុងតម្រុយអរតូណម៉ាល់ $\left(o,\vec{i},\vec{j},\vec{k}\right)$ គេមានចំណុច $A\left(1,3,-1\right),~B\left(3,0,1\right),~C\left(2,1,-3\right)$ និងបន្ទាត់ $L$ ដែលមានសមីការ $x=2-t,y=2t$ និង $z=1-t,t\in \mathbb{R}$។
	\begin{enumerate}[k]
		\item បង្ហាញថាចំណុច $A,B$ និង $C$ កំណត់បានប្លង់ $ABC$ មួយ រួចកំណត់សមីការប្លង់ $ABC$។
		\item គណនាផ្ទៃក្រឡាត្រីកោណ $ABC$។ រកកូអរដោនេចំណុច $M$ ប្រសព្វរវាងប្លង់ $ABC$ និងបន្ទាត់ $L$។
	\end{enumerate}
	{\color{khtug}២.} បញ្ជាក់សមីការ $\left(E\right): 4x^2-100=25y^2$ ជាអុីពែបូល។ រកកូអរដោនេផ្ចិត កំពូល កំណុំ និងសមីការអាសុីមតូត រួចសង់។
\end{enumerate}
\begin{center}
	\sffamily\color{blue}
	សូមអានប្រធានលំហាត់ឲ្យបានច្បាស់មុនធ្វើលំហាត់!
\end{center}\newpage
\borderline{ចម្លើយ}\\
{\color{white}.}\dotfill\\
{\color{white}.}\dotfill\\
{\color{white}.}\dotfill
\\
{\color{white}.}\dotfill\\
{\color{white}.}\dotfill\\
{\color{white}.}\dotfill
\\
{\color{white}.}\dotfill\\
{\color{white}.}\dotfill\\
{\color{white}.}\dotfill
\\
{\color{white}.}\dotfill\\
{\color{white}.}\dotfill\\
{\color{white}.}\dotfill
\\
{\color{white}.}\dotfill\\
{\color{white}.}\dotfill\\
{\color{white}.}\dotfill
\\
{\color{white}.}\dotfill\\
{\color{white}.}\dotfill\\
{\color{white}.}\dotfill
\\
{\color{white}.}\dotfill\\
{\color{white}.}\dotfill\\
{\color{white}.}\dotfill
\\
{\color{white}.}\dotfill\\
{\color{white}.}\dotfill
\begin{center}
	\sffamily\color{blue}
	សូមសំណាងល្អ!
\end{center}\newpage
{\maketitle}\\
\borderline{ប្រធានទី០៨}
\begin{enumerate}[I]
	\item  {\color{khtug}(\kml{១០ ពិន្ទុ})} គេមានចំនួនកុំផ្លិច $z_1=-\sqrt{2}\left(\cos\frac{\pi}{3}+i\sin\frac{\pi}{3}\right)$ និង $z_2=1+i$។
	\begin{enumerate}[k,2]
		\item សរសេរចំនួនកុំផ្លិច $z_1$ និង $z_2$ ជាទម្រង់ត្រីកោណមាត្រ។
		\item រកម៉ូឌុល និងអាគុយម៉ុង់នៃចំនួនកុំផ្លិច $w=\frac{z_1}{z_2}$។
	\end{enumerate}
	\item {\color{khtug}(\kml{១៥ ពិន្ទុ})} គណនាលីមីតៈ $A=\lim_{x\to1}\frac{4-\sqrt{x+15}}{x^2-1}\quad;\quad B=\lim_{x\to0}\frac{1-\cos6x}{\sin^{2}5x}\quad;\quad C=\lim_{x\to+\infty}\left[\ln\left(x^2-5x+6\right)-\ln x\right]$
	\item {\color{khtug}(\kml{១៥ ពិន្ទុ})} {\color{khtug}១.} គណនាអាំងតេក្រាលៈ $I=\int_{1}^{2}\left(1-3x+2x^2\right)dx\quad;\quad J=\int_{0}^{\frac{\pi}{4}}\frac{1-\cos2x}{1+\cos2x}dx$\\
	{\color{khtug}២.} {\color{khtug}ក.} កំណត់ចំនួនពិត $a,b$ និង $c$ ដើម្បីឲ្យ $\frac{3x^2+6x+7}{\left(x+2\right)\left(x+3\right)}=a+\frac{b}{x+2}+\frac{c}{x+3}$ ចំពោះគ្រប់ $x\in\mathbb{R}-\left \{-2;-3\right \}$។\\
	{\color{khtug}\quad~ខ.} គណនាអាំងតេក្រាល $K=\int\frac{3x^2+6x+7}{\left(x+2\right)\left(x+3\right)}$។
	\item {\color{khtug}(\kml{១០ ពិន្ទុ})} {\color{khtug}ក.} ដោះស្រាយសមីការឌីផេរ៉ង់ស្យែល $\left(E\right): y''+4y'+4y=0$។\\
	{\color{khtug}ខ.} រកចម្លើយពិសេសមួយនៃ $(E)$ បើគេដឹងថាខ្សែកោង $(H)$ តាងអនុគមន៍ចម្លើយនេះកាត់តាមចំណុច $M\left(-1;1\right)$ ហើយបន្ទាត់ប៉ះត្រង់ចំណុចនេះស្របនឹងបន្ទាត់ដែលមានសមីការ $y=2x+3$។
	\item {\color{khtug}(\kml{១៥ ពិន្ទុ})} ក្នុងថង់មួយមានប៊ិចពណ៌ខៀវ $5$ ដើម ប៊ិចពណ៌ក្រហម $4$ ដើម និងប៊ិចពណ៌ខ្មៅ $3$ ដើម។\\
	គេចាប់យកប៊ិច $4$ ដើមព្រមគ្នាចេញពីថង់ដោយចៃដន្យ។ គណនាប្រូបាបនៃព្រឹត្តិការណ៍ៈ\\
	{\color{khtug}ក.} $A:$ ប៊ិចពណ៌ខៀវទាំង $4$ ដើម\quad {\color{khtug}ខ.} $B:$ ប៊ិចទាំង $4$ ដើមមានពណ៌ដូចគ្នា\quad
	{\color{khtug}គ.} $C:$ យ៉ាងតិចមានប៊ិច $3$ ដើមពណ៌ដូចគ្នា។
	\item {\color{khtug}(\kml{២៥ ពិន្ទុ})} {\color{khtug}១.} គេមានសមីការ $4\left(5y-2x\right)\left(2x+5y\right)=-400$។ បង្ហាញថាសមីការនេះជាសមីការអុីពែបូល។ រកកូអរដោនេផ្ចិត កំពូល កំណុំ និងសមីការអាសុីមតូតទាំងពីរ រួចសង់អុីពែបូលនេះ។\\
	{\color{khtug}២.} ក្នុងតម្រុយអរតូណម៉ាល់មានទិសដៅវិជ្ជមាន $\left(o;\vec{i};\vec{j};\vec{k}\right)$ គេមាន $A\left(3;2;-1\right),B\left(-6;1;1\right);C\left(4;-3;3\right),D\left(-1;-5;-1\right)$ និង $H\left(1;-1;3\right)$។
	\begin{enumerate}[k]
		\item គណនាប្រវែង $AH$។ សរសេរសមីការប្លង់ $\left(P\right)$ ដែលកាត់តាមចំណុច $H$ ហើយកែងនឹងបន្ទាត់ $\left(AH\right)$។
		\item បង្ហាញថា $B;C;D$ ស្ថិតនៅលើប្លង់ $\left(P\right)$។
		\item គណនាកូអរដោនេនៃវុិចទ័រ $\overrightarrow{BC}\times\overrightarrow{BD}$ រួចគណនាផ្ទៃក្រឡាត្រីកោណ $BCD$។
	\end{enumerate}
	\item {\color{khtug}(\kml{៣៥ ពិន្ទុ})} គេឲ្យអនុគមន៍ $f$ កំណត់លើ $I=\left(4;+\infty\right)$ ដោយ $f(x)=-2x+5+3\ln\left(\frac{x+1}{x-4}\right)$ និង$\left(C\right)$ ជាក្រាបតំណាងអនុគមន៍ $f$ ក្នុងតម្រុយអរតូណរមេ $\left(o;\vec{i};\vec{j}\right)$ ដែលមានឯកតាលើអ័ក្ស $1cm$។
	\begin{enumerate}[m]
		\item គណនាលីមីតនៃអនុគមន៍ $f$ ត្រង់ $4$ និង $+\infty$។
		\item បង្ហាញថាចំពោះគ្រប់ $x\in I$ គេបាន $f'(x)=\frac{-2x^2+6x-7}{\left(x+1\right)\left(x-4\right)}$។​សិក្សាសញ្ញា $f'(x)$ ចំពោះគ្រប់ $x\in I$។\\ សង់តារាងអថេរភាពនៃអនុគមន៍ $f$។
		\item \begin{enumerate}[k]
			\item បង្ហាញថាបន្ទាត់ $(D)$ ដែលមានសមីការ $y=-2x+5$ ជាអាសុីមតូតនៃ $\left(C\right)$។
			\item ចំពោះគ្រប់ $x>4$ ចូរបង្ហាញថា $\frac{x+1}{x-4}>1$ រួចសិក្សាទីតាំងធៀបរវាង $(C)$ និង $(D)$។
			\item កំណត់កូអរដោនេនៃចំណុចស្ថិតនៅលើខ្សែកោង $(C)$ ដែលបន្ទាត់ $\left(\Delta\right)$ ប៉ះខ្សែកោង $(C)$\\ ត្រង់ចំណុចនោះមានមេគុណប្រាប់ទិសស្មើនឹង $-\frac{9}{2}$ រួចសរសេរសមីការបន្ទាត់ប៉ះ $\left(\Delta\right)$។
			\item សង់ក្រាប $(C)$ និងបន្ទាត់ $(D),\left(\Delta\right)$ នៅក្នុងតម្រុយតែមួយ។ (គេឲ្យ $ln6=1.8$) 
		\end{enumerate}
	\end{enumerate}
	\begin{center}
		\sffamily\color{blue}
		សូមអានប្រធានលំហាត់ឲ្យបានច្បាស់មុនធ្វើលំហាត់!
	\end{center}
\end{enumerate}
\borderline{ចម្លើយ}\\
{\color{white}.}\dotfill
\\
{\color{white}.}\dotfill\\
{\color{white}.}\dotfill\\
{\color{white}.}\dotfill
\\
{\color{white}.}\dotfill\\
{\color{white}.}\dotfill\\
{\color{white}.}\dotfill
\\
{\color{white}.}\dotfill\\
{\color{white}.}\dotfill\\
{\color{white}.}\dotfill
\\
{\color{white}.}\dotfill\\
{\color{white}.}\dotfill\\
{\color{white}.}\dotfill
\\
{\color{white}.}\dotfill\\
{\color{white}.}\dotfill\\
{\color{white}.}\dotfill
\\
{\color{white}.}\dotfill\\
{\color{white}.}\dotfill\\
{\color{white}.}\dotfill
\\
{\color{white}.}\dotfill\\
{\color{white}.}\dotfill
\begin{center}
	\sffamily\color{blue}
	សូមសំណាងល្អ!
\end{center}\newpage
{\maketitle}\\
\borderline{ប្រធានទី០៩}
\begin{enumerate}[I]
	\item {\color{khtug}(\kml{១៥ ពិន្ទុ})} ក្នុងកាបូបមួយមានប៊ិច $4$ ដើម ខ្មៅដែ $3$ ដើម និងបន្ទាត់ $2$ ដើម។ សិស្យម្នាក់លូកចាប់យកវត្ថុ $3$ ព្រមគ្នាចេញពីកាបូបដោយចៃដន្យ។ គណនាប្រូបាបនៃព្រឹត្តិការណ៍ៈ\\
	{\color{khtug}ក.} $A:$ ចាប់បានប៊ិចទាំង $3$ ដើម។ \quad {\color{khtug}ខ.} $B:$ ចាប់បានបន្ទាត់មួយដើមយ៉ាងតិច។ \quad {\color{khtug}គ.} $C:$ ចាប់បានវត្ថុមួយមុខមួយដើម។
	\item {\color{khtug}(\kml{១០ ពិន្ទុ})} {\color{khtug}ក.} សរសេរចំនួនកុំផ្លិច $z_1=\sqrt{3}+i$ និង $z_2=-1+i\sqrt{3}$ ជាទម្រង់ត្រីកោណមាត្រ។\\
	{\color{khtug}ខ.} គណនា $z^{6}_1+z^{6}_2$។ 
	\item {\color{khtug}(\kml{១៥ ពិន្ទុ})} គណនាលីមីតៈ\quad {\color{khtug}ក.} $\lim_{x\to2}\frac{4x^2-16}{\sqrt{2x}-2}$\quad {\color{khtug}ខ.} $\lim_{x\to\frac{\pi}{2}}\frac{1-\cos\left(2x-\pi\right)}{x-\frac{\pi}{2}}$\quad {\color{khtug}គ.} $\lim_{x\to-\infty}\left(e^{-x+1}-e^{1-2x}\right)$ 
	\item {\color{khtug}(\kml{១៥ ពិន្ទុ})} {\color{khtug}១.} ចូរគណនាអាំងតេក្រាលៈ\quad $I=\int_{1}^{2}\frac{\left(x-1\right)^2}{x^2}dx$\quad និង $J=\int_{0}^{\frac{\pi}{2}}\frac{2-2\sin2x}{\left(\sin x-\cos x\right)^2}dx$ ។\\
	{\color{khtug}២.} {\color{khtug}ក.} កំណត់ចំនួនពិត $A;B$ និង $C$ ដើម្បីឲ្យ $\frac{3x^2}{x^3-1}=\frac{A}{x-1}+\frac{Bx+C}{x^2+x+1}, x\ne1$។ {\color{khtug}ខ.} គណនាអាំងតេក្រាល $K=\int\frac{3x^2}{x^3-1}dx$
	\item {\color{khtug}(\kml{១០ ពិន្ទុ})} {\color{khtug}១.} ដោះស្រាយសមីការឌីផេរ៉ង់ស្យែល $(E): y''-3y'-4y=0$។\\
	{\color{khtug}២.} រកចម្លើយពិសេសមួយនៃ $(E)$ បើគេដឹងថាខ្សែកោងតាងអនុគមន៍ចម្លើយកាត់តាមចំណុច $\left(0;1\right)$ ហើយប៉ះទៅនឹងបន្ទាត់ត្រង់ចំណុចនេះមានមេគុណប្រាប់ទិសស្មើនឹង $9$។
	\item {\color{khtug}(\kml{២៥ ពិន្ទុ})} {\color{khtug}១.} {\color{khtug}ក.} រកកូអរដោនេនៃកំពូល កំណុំ និងសមីការបន្ទាត់ប្រាប់ទិសនៃប៉ារ៉ាបូល $P: 2y^2+8y+3x-4=0$។\\
	{\color{khtug}ខ.} រកកូអរដោនេនៃចំណុចប្រសព្វរវាងប៉ារ៉ាបូល $P$ និងបន្ទាត់មានសមីការ $x=-2$ រួចសង់ប៉ារ៉ាបូល $P$។\\
	{\color{khtug}២.} គេមានកូអរដោនេនៃចំណុច $M\left(2,1,0\right)~;~N\left(1,-2,2\right)$ និង $P\left(0,-2,1\right)$។
	\begin{enumerate}[k]
		\item បង្ហាញថាចំណុច $M,N$ និង $P$ ជាកំពូលទាំងបីនៃត្រីកោណសមបាត។
		\item សរសេរសមីការប្លង់ $\left(\alpha\right)$ ដែលកាត់តាមចំណុច $M,N$ និង $P$។
		\item រកសមីការទូទៅស៊្វែរ $\left(S\right)$ មួយដែលមានផ្ចិត $A\left(1,2,-3\right)$ ហើយប៉ះទៅនឹងប្លង់ $\left(\alpha\right)$។
	\end{enumerate}
	\item {\color{khtug}(\kml{៣៥ ពិន្ទុ})} {\color{khtug}\kml ផ្នែក​ \en A} គេឲ្យអនុគមន៍ $g$ កំណត់លើ $\left(0;+\infty\right)$ ដោយ $g(x)=4x^2+1-\ln x$។
	\begin{enumerate}[k]
		\item គណនាដេរីវេ $g'(x)$ និងសិក្សាសញ្ញានៃដេរីវេ $g'(x)$ លើចន្លោះ $\left(0;+\infty\right)$។
		\item សិក្សាអថេរភាពនៃអនុគមន៍ $g(x)$(ដោយមិនចាំបាច់គណនាលីមីតត្រង់ $0$ និង $+\infty$) រួចទាញរកសញ្ញានៃ $g(x)$។
	\end{enumerate}
	{\color{khtug}\kml ផ្នែក​ \en B} គេមានអនុគមន៍ $f(x)=4x-4+\frac{\ln x}{x}$ ហើយមានខ្សែកោង $\left(C\right)$ ក្នុងតម្រុយអរតូណមេ $\left(o;\vec{i};\vec{j}\right)$ \\ដែលមានឯកតាលើអ័ក្សស្មើនឹង $2cm$។
	\begin{enumerate}[k]
		\item រកលីមីត $\lim_{x\to0^{+}}f(x)$ និង $\lim_{x\to+\infty}f(x)$ រួចទាញរកអាសុីមតូតនៃក្រាប $(C)$។
		\item គណនាដេរីវេ $f'(x)$ និងបញ្ជាក់ថា $f'(x)$ មានសញ្ញាដូច $g(x)$ លើចន្លោះ $\left(0;+\infty\right)$។ គូសតារាងអថេរភាពនៃ $f(x)$។
		\item បង្ហាញថាបន្ទាត់ $\Delta: y=4x-4$ ជាអាសុីមតូតទ្រេតនៃ $(C)$ រួចសិក្សាទីតាំងធៀបនៃក្រាប $(C)$ និងបន្ទាត់ $\Delta$។
		\item សង់ក្រាប $(C)$ និងបន្ទាត់ $\Delta$ ក្នុងតម្រុយអរតូណរមេ $\left(O;\vec{i};\vec{j}\right)$។
		\item គណនាផ្ទៃក្រឡាផ្នែកប្លង់ដែលខណ្ឌដោយខ្សែកោង អ័ក្សអាប់សុីស និងបន្ទាត់ឈរ $x=1$ និង $x=e$។ (គេឲ្យៈ $\ln2=0.7$)
	\end{enumerate}
\end{enumerate}
\borderline{ចម្លើយ}\\
{\color{white}.}\dotfill
\\
{\color{white}.}\dotfill\\
{\color{white}.}\dotfill\\
{\color{white}.}\dotfill
\\
{\color{white}.}\dotfill\\
{\color{white}.}\dotfill\\
{\color{white}.}\dotfill
\\
{\color{white}.}\dotfill\\
{\color{white}.}\dotfill\\
{\color{white}.}\dotfill
\\
{\color{white}.}\dotfill\\
{\color{white}.}\dotfill\\
{\color{white}.}\dotfill
\\
{\color{white}.}\dotfill\\
{\color{white}.}\dotfill\\
{\color{white}.}\dotfill
\\
{\color{white}.}\dotfill\\
{\color{white}.}\dotfill\\
{\color{white}.}\dotfill
\\
{\color{white}.}\dotfill\\
{\color{white}.}\dotfill
\begin{center}
	\sffamily\color{blue}
	សូមសំណាងល្អ!
\end{center}\newpage
{\maketitle}\\
\borderline{ប្រធានទី១០}
\begin{enumerate}[I]
	\item {\color{khtug}(\kml{១០ ពិន្ទុ})} ថង់មួយមានសៀវភៅពណ៌ក្រហម $5$ ក្បាល ពណ៌ខ្មៅ $3$ ក្បាល និងពណ៌ស $2$ ក្បាល។
\end{enumerate}
\borderline{ចម្លើយ}\\
{\color{white}.}\dotfill
\\
{\color{white}.}\dotfill\\
{\color{white}.}\dotfill\\
{\color{white}.}\dotfill
\\
{\color{white}.}\dotfill\\
{\color{white}.}\dotfill\\
{\color{white}.}\dotfill
\\
{\color{white}.}\dotfill\\
{\color{white}.}\dotfill\\
{\color{white}.}\dotfill
\\
{\color{white}.}\dotfill\\
{\color{white}.}\dotfill\\
{\color{white}.}\dotfill
\\
{\color{white}.}\dotfill\\
{\color{white}.}\dotfill\\
{\color{white}.}\dotfill
\\
{\color{white}.}\dotfill\\
{\color{white}.}\dotfill\\
{\color{white}.}\dotfill
\\
{\color{white}.}\dotfill\\
{\color{white}.}\dotfill
\begin{center}
	\sffamily\color{blue}
	សូមសំណាងល្អ!
\end{center}\newpage
\end{document}