\documentclass[12pt, a4paper]{article}
%%import package named hightest
\usepackage{hightest}
%\usepackage{mathpazo}% change math font
%\usepackage[no-math]{fontspec}% font specfication
\header{រៀនគណិតវិទ្យាទាំងអស់គ្នា}{គណិតវិទ្យា}{០៨/០៣/២០១៨}
\footer{រៀបរៀង និងបង្រៀនដោយ ស៊ុំ សំអុន}{ទំព័រ \thepage}{០៩៦ ៩៤០ ៥៨៤០}
\everymath{\protect\displaystyle\protect\color{black}}
\begin{document}
\maketitle
\begin{enumerate}[m]
	\item បង្ហាញថា $F(x)$ ជាព្រីមីទីវមួយនៃ $f(x)$ តាមករណីដូចខាងក្រោម៖
	\begin{enumerate}[k]
		\item $F(x)=x^3+2x^2-3$ និង $f(x)=3x^2+4x$
		\item $F(x)=e^x-\sin x +x^2+4$ និង $f(x)=e^x-\cos x +2x$
		\item $F(x)=\ln|x|+\sin2x-2x$ និង $f(x)=\frac{1}{x}+2\cos2x -2$
		\item $F(x)=-\cos4x$ និង $f(x)=4\sin4x$
	\end{enumerate}
	\item គណនាអាំងតេក្រាលមិនកំណត់នៃអនុគមន៍ខាងក្រោម៖
	\begin{enumerate}[k, 4]
		\item $\int 6 dx$
		\item $\int -2018 dx$
		\item $\int x^3 dx$
		\item $\int (x^{2018}+1) dx$
		\item $\int 6x dx$
		\item $\int (5x+2) dx$
		\item $\int (7x^6+3x^2) dx$
		\item $\int (x^{-2}+x^2) dx$
		\item $\int \sqrt{x} dx$
		\item $\int \sqrt[3]{x^2} dx$
		\item $\int \bigg(\frac{1}{x}+x\bigg) dx$
		\item $\int \frac{1}{x^{20}} dx$
		\item $\int \frac{400}{x^5} dx$
		\item $\int \frac{1}{\sqrt{x}} dx$
		\item $\int \frac{100}{3\sqrt{x}}$
		\item $\int \frac{2}{\sqrt[4]{x^3}} dx$
	\end{enumerate}
	\item គណនាអាំតេក្រាលមិនកំណត់នៃអនុគមន៍ត្រីកោណមាត្រខាងក្រោម៖
	\begin{enumerate}[k, 4]
		\item $\int 2\cos x dx$
		\item $\int (2x-\sin x) dx$
		\item $\int (1+\tan^2x) dx$
		\item $\int (2+\tan^2x) dx$
		\item $\int (1+\cot^2x) dx$
		\item $\int (3+3\cot^2x) dx$
		\item $\int 20\tan^2x dx$
		\item $\int 4\cot^2x dx$
		\item $\int\frac{\sin2x}{2\cos x} dx$
		\item $\int \frac{dx}{\cos^2x\sin^2x} $
		\item $\int \frac{\cos2x}{\sin x+\cos x} dx$
		\item $\int \frac{\cos2x}{\cos^2x\sin^2x} dx$
		\item $\int ( 6x^2-\tan^2x) dx$
		\item $\int (1-\cot^2x) dx$
		\item $\int (4x+2\cot^2x) dx$
		\item $\int \frac{20}{\cos^2x} dx$
	\end{enumerate}
	\item គណនាអាំងតេក្រាលមិនកំណត់នៃអនុគមន៍អ៊ិចស្ប៉ូណង់ស្យែលខាងក្រោម៖
	\begin{enumerate}[k, 4]
		\item $\int 4e^x dx$
		\item $\int (e^x-x) dx$
		\item $\int \frac{20e^{2x}}{e^x} dx$
		\item $\int (\sqrt{2}e^x+2x) dx$
		\item $\int (2-3e^x) dx$
		\item $\int (4e^x+9x^2) dx$
		\item $\int (\sqrt[3]{x} +2e^x) dx$
		\item $\int (\sqrt{3}e^x+x^{-2}) dx$
	\end{enumerate}
	\item គណនាអាំងតេក្រាលមិនកំណត់ដោយប្រើអថេរជំនួយនៃអនុគមន៍ខាងក្រោម៖
	\begin{enumerate}[k, 3]
		\item $\int 3(x+2)^2 dx$
		\item $\int 8(2x-1)^3 dx$
		\item $\int (2-x)^{-2} dx$
		\item $\int (2x-3)^{-3} dx$
		\item $\int \sqrt{5x-3} dx$
		\item $\int 2x(x^2-1)^2 dx$
		\item $\int (2x-3)(x^2-3x+2) dx$
		\item $\int \frac{2x}{x^2-1} dx$
		\item $\int \frac{20}{(x+1)^2} dx$
		\item $\int x^2(2x-3)^{10} dx$
		\item $\int x^5(4+x)^{16} dx$
		\item $\int \frac{xdx}{x^2-9}$
		\item $\int (2x-3)(x-1) dx$
		\item $\int (x^2-1)(x^2+2) dx$
		\item $\int (1-2x)(1+x-x^2)^3 dx$
		\item $\int (2-x)(-\frac{1}{2}x^2+2x)^4 dx$
		\item $\int (x^2+1)^4x^3 dx$
		\item $\int x^3(x^2+1)^5 dx$
		\item $\int \frac{x^2}{(x+1)^10} dx$
		\item $\int \frac{x^2}{(1-x)^{100}} dx$
		\item $\int \frac{-3}{(x-1)^2} dx$
	\end{enumerate}
	\item គណនាអាំងតេក្រាលមិនកំណត់ដោយប្រើអថេរជំនួយនៃអនុគមន៍ត្រីកោណមាត្រខាងក្រោម៖
	\begin{enumerate}[k, 3]
		\item $\int \sin x \cos x dx$
		\item $\int 3\cos x \sin^2x dx$
		\item $\int \frac{\sin x}{(1-\cos x)^3} dx$
		\item $\int \frac{\cos x}{(1+\sin x)^4} dx$
		\item $\int \frac{\cos x}{\sqrt{1+\sin x}} dx$
		\item $\int \frac{\sin x}{\sqrt{2-\cos x}} dx$
		\item $\int \sin x(3-\cos x)^{20} dx$
		\item $\int (2-\sin x)(2x+\cos x)^4 dx$
		\item $\int \frac{1+\cos x}{(x+\sin x)^{33}} dx$
		\item $\int 2x\cos(x^2-3) dx$
		\item $\int 3(x^2-1)\cos(x^3-3x) dx$
		\item $\int 2(x-2)\sin(x^2-4x) dx$
		\item $\int \sin4x dx$
		\item $\int (2x+\cos7x) dx$
		\item $\int (x^3+\sin100x) dx$
		\item $\int \tan x(1+\tan^2x) dx$
		\item $\int \frac{\tan x}{\cos^2x} dx$
		\item $\int \frac{2\cot x}{\sin^2x} dx$
	\end{enumerate}	
	\item គណនាអាំងតេក្រាលមិនកំណត់ដោយប្រើអថេរជំនួយនៃអនុគមន៍អ៊ិចស្ប៉ូណង់ស្យែលខាងក្រោម៖
	\begin{enumerate}[k, 3]
		\item $\int 2e^{2x-1} dx$
		\item $\int 3x^2e^{x^3} dx$
		\item $\int 2xe^{x^2} dx$
		\item $\int 4e^{4-3x} dx$
		\item $\int (2x-3)e^{x^2-3x+2}dx$
		\item $\int 4x^2e^{x^3} dx$
		\item $\int (x+1)e^{x^2+2x} dx$
		\item $\int \sin xe^{\cos x} dx$
		\item $\int e^{\frac{1}{\sin^2x}} \tan xdx$
	\end{enumerate}
	\item គណនាអាំងតេក្រាលមិនកំណត់ដោយប្រើរូបមន្តអាំងតេក្រាលដោយផ្នែកនៃអនុគមន៍ខាងក្រោម៖
	\begin{enumerate}[k, 4]
		\item $\int xe^x dx$
		\item $\int xe^{2x} dx$
		\item $\int x^2\sin x dx$
		\item $\int 2x\cos x dx$
		\item $\int xe^{3x} dx$
		\item $\int (x-3)e^x dx$
		\item $\int x^2e^x dx$
		\item $\int e^x\cos x dx$
		\item $\int e^x \sin x dx$
		\item $\int x\cos x dx$
		\item $\int (2x-3)\cos5x dx$
		\item $\int e^x\cos3x dx$
		\item $\int e^{2x}\sin(2x+1) dx$
		\item $\int e^{3x}\sin x dx$
		\item $\int e^{2x}\cos x dx$
		\item $\int (x+3)e^x dx$
		\item $\int (x^2+2x+1)e^x dx$
		\item $\int (e^x+1)\cos x dx$
		\item $\int \ln x dx$
		\item $\int x\ln x dx$
		\item $\int x^2 \ln2x  dx$
		\item $\int (x+1)\ln x dx$
		\item $\int (x^2+2)\ln x dx$
		\item $\int (x+2)\ln^2x dx$ 
	\end{enumerate}
	\begin{center}
		\sffamily\color{black}
		សូមសំណាងល្អ!
	\end{center}\newpage
	%==========================================================
	\begin{center}
		\sffamily\color{black}
		\circled{០២}\\
		ព្រីមីទីវ និងអាំងតេក្រាលមិនកំណត់ \\
		រៀបរៀង និងបង្រៀនដោយៈ ស៊ុំ សំអុន\\
		ទូរស័ព្ទៈ ០៩៦ ៩៤០	៥៨៤០
	\end{center}
	%=========================================================
	\item គណនាអាំងតេក្រាលមិនកំណត់នៃអនុគមន៍ខាងក្រោម៖
		\begin{enumerate}[k, 3]
			\item $\int \sin2x\cos3x dx$
			\item $\int \sin4x\cos6x dx$
			\item $\int \sin7x\cos5x dx$
			\item $\int \sin9x\cos4x dx$
			\item $\int \cos2x\cos x dx$
			\item $\int \cos3x\cos5x dx$
			\item $\int \cos7x\cos3x dx$
			\item $\int \cos8x\cos10x dx$
			\item $\int \sin6x\sin2x dx$
			\item $\int \sin5x\sin8x dx$
			\item $\int \sin^2x\cos^3x dx$
			\item $\int \sin^4x\cos^3x dx$
			\item $\int \sin^6x\cos^5x dx$
			\item $\int \sin^8x\cos^5x dx$
			\item $\int \cos^2x\sin^3x dx$
			\item $\int \cos^4x\sin^3x dx$
			\item $\int \cos^6x\sin^5x dx$
			\item $\int \cos^8x\sin^5x dx$
			\item $\int \sin^4x\cos^5x dx$
			\item $\int \sin^5x\cos^4x dx$
			\item $\int \sin^3x\cos^5x dx$
			\item $\int \sin^5x\cos^3x dx$
			\item $\int \sin^3x\cos^6x dx$
			\item $\int \cos^3x\sin^6x dx$
			\item $\int \sin^2x\cos^2x dx$
			\item $\int \cos^2x\sin^4x dx$
		\end{enumerate}
		\item គណនាអាំងតេក្រាលមិនកំណត់នៃអនុគមន៍ខាងក្រោម៖
		\begin{enumerate}[k, 3]
			\item $\int \tan^2x dx$
			\item $\int \tan^3x dx$
			\item $\int \tan^4x dx$
			\item $\int \tan^5x dx$
			\item $\int \tan^6x dx$
			\item $\int \tan^7x dx$
			\item $\int \tan^8x dx$
			\item $\int \cot^2x dx$
			\item $\int \cot^3x dx$
			\item $\int \cot^4x dx$
			\item $\int \cot^5x dx$
			\item $\int \cot^6x dx$
			\item $\int \cot^7x dx$
			\item $\int \sin^2x dx$
			\item $\int \sin^3x dx$
			\item $\int \sin^4x dx$
			\item $\int \sin^5x dx$
			\item $\int \sin^6x dx$
			\item $\int \cos^2x dx$
			\item $\int \cos^3x dx$
			\item $\int \cos^4x dx$
			\item $\int \cos^5x dx$
			\item $\int \cos^6x dx$
			\item $\int \tan^9x dx$
			\item $\int \cot^8x dx$
		\end{enumerate}
	\item គេមានអនុគមន៍ $f(x)=\frac{\cos x}{\cos x+\sin x}$ និង $g(x)=\frac{\sin x}{\cos x+\sin x}$ ។
	\begin{enumerate}[k]
		\item គណនាអាំងតេក្រាល $\int [f(x)+g(x)] dx$ និង $\int [f(x)-g(x)] dx$
		\item ទាញរកអាំងតេក្រាល $\int f(x) dx$ និង $\int g(x) dx$
	\end{enumerate}
	\item គេមានអនុគមន៍ $I=\int \frac{\cos x}{2\cos x+3\sin x}$ និង $J=\int \frac{\sin x}{2\cos x+3\sin x}$ ។
	\begin{enumerate}[k]
		\item គណនាអាំងតេក្រាល $2I+3J$ និង $3I-2J$
		\item គណនាអាំងតេក្រាល $I$ និង $J$
		\item គណនាអាំងតេក្រាល $\int \frac{4\cos x+5\sin x}{2\cos x +3\sin x} dx$
	\end{enumerate}
	\item គេមានអនុគមន៍ $f(x)=\frac{-\cos x+7\sin x}{3\cos x+4\sin x}$ ។
	\begin{enumerate}[k]
		\item ចូរកំណត់រកចំនួនពិត $a$ និង $b$ ដែល $f(x)=a+b\bigg(\frac{-3\cos x+4\sin x}{3\cos x+4\sin x}\bigg)$ ។
		\item គណនាអាំងតេក្រាល $\int f(x) dx$ ។
	\end{enumerate}
	\item គេមានអនុគមន៍ $f(x)=\frac{1}{e^x+1}$ ។
	\begin{enumerate}[k]
		\item កំណត់រកចំនួនពិត $a$ និង $b$ ដើម្បីឲ្យ $f(x)=a+\frac{be^x}{e^x+1}$ ។
		\item គណនាអាំងតេក្រាល $\int f(x) dx$ ។
	\end{enumerate}
	\item គេមានអនុគមន៍ $f(x)=\frac{2}{e^{2x}+3e^x+2}$ ។
	\begin{enumerate}[k]
		\item កំណត់រកចំនួនពិត $a, ~c~$ និង $c$ ដើម្បីឲ្យ $f(x)=a+\frac{be^x}{e^x+1}+\frac{ce^x}{e^x+2}$ ។
		\item គណនាអាំងតេក្រាល $\int f(x) dx$ ។
	\end{enumerate}
	\item គេមានអនុគមន៍ $f(x)=\frac{-3x+2}{x^4-2x^3+x^2}$ កំណត់ចំពោះគ្រប់ $x\neq0$ និង$x\neq1$ ។
	\begin{enumerate}[k]
		\item កំណត់រកចំនួនពិត $a~,~b~,~c$ និង $d$ ដើម្បីឲ្យ $f(x)=\frac{a}{x}+\frac{b}{x^2}+\frac{c}{x-1}+\frac{d}{(x-1)^2}$ ។
		\item គណនាអាំងតេក្រាល $\int f(x) dx$ 
	\end{enumerate}
	\item គេមានអនុគមន៍ $f(x)=\frac{(3x^3+1)^2}{x^2}$ ។
	\begin{enumerate}[k]
		\item សរសេរ $f(x)$ ជារាង $f(x)=Ax^2+B+\frac{C}{x^2}$ រួចកំណត់ចំនួនពិត $A, B$ និង $C$ ។
		\item គណនា $\int f(x) dx$ ។
	\end{enumerate}
%	\item គេមានអនុគមន៍ $f(x)=\sin x -\cos x+1$ ។
%	\begin{enumerate}[k]
%		\item កំណត់ចំនួនពិត $A, B$ និង $C$ ដើម្បីឲ្យ $f(x)$ អាចសរសេរជារាង \\$f(x)=A(\sin x +2\cos x +3)+B(\cos x-2\sin x)+C$
%		\item គណនា $\int \frac{\sin x-\cos x+1}{\sin x+2\cos x+3} dx$
%	\end{enumerate}
		\begin{center}
			\sffamily\color{black}
			សូមសំណាងល្អ!
		\end{center}\newpage
%============================================================
	\begin{center}
		\sffamily\color{black}
		\circled{០៣}\\
		 អាំងតេក្រាលកំណត់ \\
		រៀបរៀង និងបង្រៀនដោយៈ ស៊ុំ សំអុន\\
		 ទូរស័ព្ទៈ ០៩៦ ៩៤០	៥៨៤០
	\end{center}
%============================================================
	\item 
\end{enumerate}
	\begin{center}
		\sffamily\color{black}
		សូមសំណាងល្អ!
	\end{center}\newpage
\end{document}