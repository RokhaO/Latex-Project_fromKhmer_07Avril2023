\documentclass[a4paper, 12pt]{exam}
\makeatletter
\usepackage[top=0.5cm, left=1cm, bottom=1.8cm, right=1.5cm]{geometry}
\usepackage{amsmath,amssymb}
\usepackage{tcolorbox}
\usepackage[export]{adjustbox}
\usepackage{graphicx}
\usepackage{wrapfig}
\usepackage{pgf}
\usepackage{tikz}% graphic drawing
\usetikzlibrary{arrows}
\pagestyle{empty}
\usepackage{wasysym}
\usepackage{mathpazo}% change math font
\usepackage{enumitem}% change list environment like enumerate, itemize and description
\usepackage{multicol}% multi columns
\usepackage{xcolor}
\newcommand{\teacher}{ស៊ំុ សំអុន}
\newcommand{\tell}{០៩៦ ៩៤០ ៥៨៤០}
\newcommand{\class}{ត្រៀមប្រឡង~ឆមាសលើកទី~០១}
\newcommand{\dateofexam}{សម័យប្រឡង៖~មករា ២០ ២០១៨}
\newcommand{\subject}{វិញ្ញាសា៖~គណិតវិទ្យា~(វិទ្យាសាស្រ្តពិត)}
\newcommand{\timelimit}{១២០~នាទី}
\newcommand{\score}{ពិន្ទុសរុប៖~១២៥~ពិន្ទុ}
\usepackage[no-math]{fontspec}% font specfication
\setmainfont{Khmer OS Content}% set default font to Khmer OS
\setsansfont[Ligatures=TeX,AutoFakeBold=0,AutoFakeSlant=0.25]{Khmer OS Muol Light}% sans serif font
%
\newcommand{\heart}{\ensuremath\heartsuit}
\newcommand{\butt}{\rotatebox[origin=c]{180}{\heart}}
\newcommand*\circled[1]{\tikz[baseline=(char.base)]{
		\node[shape=circle,draw,inner sep=2pt] (char) {#1};}}
%
\SetEnumitemKey{I}{%
	leftmargin=*,
	label={\protect\tikz[baseline=-0.9ex]\protect\node[draw=gray,thick,circle,minimum height=.65cm,inner sep=1pt,text=black,fill=white]{\Roman*};},%
	font=\small\sffamily\bfseries,%
	labelsep=1ex,%
	topsep=0pt}
%
\SetEnumitemKey{a}{%
	leftmargin=*,%
	label={\protect\tikz[baseline=-0.9ex]\protect\node[draw=gray,thick,circle,minimum height=.5cm,inner sep=1pt,text=blue,fill=magenta!5!white]{\alph*};},%
	font=\small\sffamily\bfseries,%
	labelsep=1ex,%
	topsep=0pt}
%
\SetEnumitemKey{1}{leftmargin=*,%
	label={\protect\tikz[baseline=-0.9ex]\protect\node[draw=gray,thick,circle,minimum height=.5cm,inner sep=1pt,text=black,fill=cyan!20!white]{\arabic*};},%
	font=\small\sffamily\bfseries,%
	labelsep=1ex,%
	topsep=0pt}
%
\def\hard{\leavevmode\makebox[0pt][r]{\large\ensuremath{\star}\hspace{2em}}}
%
\def\hhard{\leavevmode\makebox[0pt][r]{\large\ensuremath{\star\star}\hspace{2em}}}
%
\everymath{\protect\displaystyle\protect\color{black}}
%
\pagecolor{cyan!1!white}
%
\usepackage{amsmath}
\usepackage{amssymb}
\usepackage{wasysym}
\makeatother

\pagestyle{foot}
\firstpagefooter{}{ ទំព័រ~\thepage\ នៃ \numpages}{}
\runningfooter{រៀបរៀងដោយ~\teacher}{ ទំព័រ \thepage\ នៃ \numpages}{ទូរស័ព្ទ~\tell}
\runningfootrule

\begin{document}
\noindent
%\sffamily\color{black}
\begin{tabular*}{\textwidth \sffamily\color{black}}{l @{\extracolsep{\fill}} r @{\extracolsep{6pt}} l}
\textbf{\class} & \textbf{មណ្ឌលប្រឡង} & \makebox[2in]{\hrulefill}\\
\textbf{\dateofexam} & \textbf{លេខបន្ទប់} & \makebox[2in]{\hrulefill}\\
\textbf{\subject} & \textbf{លេខតុ} & \makebox[2in]{\hrulefill}\\
\textbf{\score} & \textbf{ឈ្មោះបេក្ខជន} & \makebox[2in]{\hrulefill}\\
\textbf{រយៈពេលសរុប៖ \timelimit} & \textbf{ហេត្ថលេខា} & \makebox[2in]{\hrulefill}
\end{tabular*}\\
%\rule[2ex]{\textwidth\color{magenta}}{2pt}
%\begin{center}
%	\heart វិញ្ញាសាទី\textbf{០១} នេះមាន \numpages\ ទំព័រ (រួមបញ្ចូលទាំងទំព័រនេះផងដែរ) និង 6 សំណួរ ។\heart
%%	\sffamily\color{black}
%%	\circled{០១}
%\end{center}
%\begin{center}
%	\sffamily\color{black}
%	បទបញ្ជានៃការប្រឡង
%\end{center}
%\begin{multicols}{2}
%	\begin{enumerate}[1]
%		\item ហាមមើលគ្នាក្នុងពេលកំពុងប្រឡង.
%		\item មិនត្រូវប្តូរកិច្ចការរបស់ខ្លួនជាមួយមិត្តភក្ភិឡើយ 
%		\item ហាមខ្ចីរបស់គ្នាប្រើក្នុងពេលកំពុងប្រឡង 
%		\item ហាមនិយាយឡូឡាក្នុងពេលកំពុងប្រឡង
%		\item មិនត្រូវនាំអាវុធ ឬគ្រឿងផ្ទុះចូរបន្ទប់ប្រឡង 
%		\item មិនត្រូវធ្វើសញ្ញាសម្គាល់អ្វីមួយនៅលើក្រដាស់កិច្ចការឡើយ~។ 
%	\end{enumerate}
%\end{multicols}
\noindent
\rule[2ex]{\textwidth\color{magenta}}{2pt}
\begin{center}
	\sffamily\color{black}
	ប្រធានលំហាត់\\
\end{center}
\begin{enumerate}[I]
	\item ($ 15 $ ពិន្ទុ) គេឲ្យអនុគមន៍ $f$ កំណត់ដោយ $f(x)=\frac{8-8\cos x}{x(e^{2x}-1)} (x\neq0)$ ហើយ $f(0)=\frac{1}{1008}(m-2)$ ។
	\begin{enumerate}[1]
		\item គណនាលីមីត $\lim\limits_{x\to0}f(x)$
		\item កំណត់តម្លៃ $m$ ដើម្បីឲ្យ $f$ ជាប់ត្រង់ $x=0$ ។
	\end{enumerate}
	\item ($ 15 $ ពិន្ទុ) គណនាលីមីតខាងក្រោម៖
	\begin{multicols}{3}
		\begin{enumerate}[a]
			\item $ \lim\limits_{x\to 3}\frac{x^2-4x+3}{9-x^2} $
			\item $ \lim\limits_{x\to 2}\frac{x\sqrt{x}-2\sqrt{2}}{\sqrt{x}-\sqrt{2}} $
			\item $ \lim\limits_{x\to 0}\frac{e^{x^2}+\sin (x^2)-1}{2x\sin x} $
		\end{enumerate}
	\end{multicols}
	\item ($ 15 $ ពិន្ទុ) គេមានអនុគមន៍ $h$ កំណត់ដោយ $ y=h(x)=\sin(\cos^2x)$~។
	\begin{enumerate}[1]
		\item បង្ហាញថា $h'(x)+\sin2x\cdot\cos(\cos^2x)=0$~។
		\item រកសមីការបន្ទាត់ប៉ះនឹងក្រាបតាងអនុគមន៍ $f$ ត្រង់ $x=\frac{\pi}{4}$~។
	\end{enumerate}
	\item ($ 15 $ ពិន្ទុ) គេឲ្យចំនួនកុំផ្លិច $a=\frac{\sqrt{3}}{2}-\frac{1}{2}i$
	និង $b=2+2\sqrt{3}i$ ។
	\begin{enumerate}[1]
		\item សរសេរចំនួនកុំផ្លិច $\frac{a}{b}$ ជាទម្រង់ពីជគណិត ។
		\item សរសេរចំនួនកផ្លិច $a, b$ និង $\frac{a}{b}$ ជាទម្រង់ត្រីកោណមាត្រ ។
	\end{enumerate}
	\item\hard ($ 30 $ ពិន្ទុ) អនុវត្តន៍ $ f $ កំណត់ដោយ $f(x)=x+2-\frac{4}{x-1}$ និងមានខ្សែកោង $C$ ។
	\begin{enumerate}[a]
		\item រកដែនកំណត់នៃអនុគមន៍ $f$ ។ គណនា និងសិក្សាសញ្ញាដេរីវេ $f'(x)$ ។
		\item រកតម្លៃអតិបរមា និងអប្យបរមានៃ $f$ ។
		\item កំណត់សមីការនៃអាស៊ីមតូតឈរ និងទ្រេតនៃខ្សែកោង $C$ ។
		\item សិក្សាទីតាំងធៀបរវាងអាស៊ីមតូតទ្រេត និងខ្សែកោង $C$ ។
		\item សង់តារាងអថេរភាពនៃអនុគមន៍ $f$ និងសង់ខ្សែកោង $C$ ។
	\end{enumerate}
	
	\item\hhard ($ 35 $ ពិន្ទុ) ក្នុងលំហរប្រដាប់ដោយតម្រុយអតូណរម៉ាល់មានទិសដៅវិជ្ជមាន $ (O,\vec{i},\vec{j},\vec{k}) $ គេមានចំណុច​\\
	$ A(1,0,0),B(0,1,0) $ និង $ C(0,0,1) $~។
	\begin{enumerate}[a]
		\item បង្ហាញថាត្រីកោណ $ABC$ ជាត្រីកោណសម័ង្ស ។
		
		\item គណនាផលគុណ $ \overrightarrow{n}= \overrightarrow{AB}\times\overrightarrow{AC}$ រួចរកសមីការប្លង់ $(ABC)$~។
		\item រកចម្ងាយពីចំណុច $D(0, 1, 1)$~ទៅប្លង់ $(ABC)$~។
		\item រកសមីការស្វ៊ែ $(S)$~ ដែលមានអង្គត់ផ្ចិត $AC$~។
		\item រកសមីការប្លង់ $(P)$~ ប៉ះស្វ៊ែ $(S)$~ត្រង់ $C$~។ 
	\end{enumerate}
\end{enumerate}
\begin{center}
	\sffamily\color{black}
	សូមសំណាងល្អ!
\end{center}\newpage

\end{document}
