\documentclass[12pt, 4appaer]{book}
\usepackage{amsmath}
\usepackage{amsfonts}
\usepackage{amssymb}
\usepackage[no-math]{fontspec}
\setmainfont[Ligatures=TeX,AutoFakeBold=1, AutoFakeSlant=0.25]{Khmer OS Battambang} % roman font
\setsansfont[Ligatures=TeX,AutoFakeBold=0.25,AutoFakeSlant=0.25]{Khmer OS Muol Light}% sans serif font
\setmonofont[Ligatures=TeX,AutoFakeBold=0.25,AutoFakeSlant=0.25]{Khmer OS Bokor}% typewriter font
\XeTeXlinebreaklocale "kh"
\XeTeXlinebreakskip = 0pt plus 1pt minus 1pt% line break skip
\setmainfont[Scale = 0.8333333334, Script=Khmer]{Khmer OS Battambang}
\setmathrm{Times New Roman}
\newfontfamily{\kbk}{Khmer OS Bokor}
\newfontfamily{\kml}{Khmer OS Muol Light}
\def\absractname{\kml សង្ខេប}
\def\appendixname{\kml សេចក្តីផ្តើម}
\def\bibname{\kml គន្ថនិទ្ទេស}
\def\chaptername{\kml ជំពូក}
\def\contentsname{\kml មាតិកា}
\def\figurename{\kml រូបភាព}
\def\indexname{\kml លិបិក្រម}
\def\listfigurename{\kml បញ្ជីរូបភាព}
\def\listtablename{\kml បញ្ជីតារាង}
\def\partname{\kml ផ្នែក}
\def\refname{\kml ឯកសារយោង}
\def\tablename{\kml តារាង}
\begin{document}
	\frontmatter
	\tableofcontents
	\mainmatter
	\part{ចំណងជើង}
	\chapter{ចំណងជើង}
	\section{ចំណងជើង}
	\chapter{ចំណងជើង}
	\section{ចំណងជើង}
	\appendix
	\backmatter
	\begin{thebibliography}{2}
		\bibitem{tobias15} Tobias Oetiker , \emph{The Not So Short Introduction to
			\LaTeXe}, Version 5.05, July 18, 2015
		\bibitem{leslie94} Leslie Lamport , \emph{\LaTeX: A Document Preparation
			System}, 2nd Edition , Addison -Wesley Professional , 1994.
	\end{thebibliography}
\end{document}