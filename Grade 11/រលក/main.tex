\documentclass{officialexam} 
\usepackage{chemfig}
\usepackage{pgf,tikz}
\usetikzlibrary{arrows}
\usepackage{amsmath,amssymb}
\usepackage{bm} % math bold
\usepackage[outline]{contour} % glow around text
\usepackage{tikz}
\usetikzlibrary{patterns}
\tikzset{>=latex}
\usetikzlibrary{calc}
\usetikzlibrary{angles,quotes} % for pic
\usepackage{wrapfig}
\usepackage{graphicx}
\usepackage{float}
\usepackage[table]{xcolor}
\usepackage{pgfplots}
\def\tablename{តារាង}
\renewcommand{\thefigure}{\khmer{figure}}
\def\figurename{រូបទី}
\graphicspath{ {./images/} }
\usepackage{esvect}
\usepackage[version=4]{mhchem}
\begin{document}
	\definecolor{xfqqff}{rgb}{0.5,0,1}
	\definecolor{qqqqff}{rgb}{0,0,1}
	\definecolor{ffqqqq}{rgb}{1,0,0}
	\definecolor{qqqqff}{rgb}{0,0,1}
	\borderline{មេរៀនទី ៥ លំហាត់ ចលនាខួប}
	\begin{enumerate}[m]
		\item ចល័តមួយចងភ្ជាប់ទៅនឹងរ៉ឺសរបញ្ឃរសន្ធឹងមួយប្រវែង $10cm$ ត្រូវបានទាញចុះក្រោមប្រវែង $A=5.0cm$ រួចលែងដោយគ្មានល្បឿនដើម។ គេឲ្យ $k=29.4N/m$ និងពុលសាស្យុង $\omega = 9.90rad/s$ ។ រកប្រកង់ និងខួបនៃចលនា។
		\item លំយោលសុីនុយសូអុីតមួយមានអំព្លីទុត $5cm$ និងខួប $2s$។ នៅខណៈដើមពេលភាគល្អិតស្ថិតត្រង់ទីតាំង $25cm$ ។ ចូរកំណត់សមីការនៃបម្លាស់ទីភាគល្អិតជាអនុគមន៍នៃពេល។
		\item អង្គធាតុមួយត្រូវបានគេចងព្យួរទៅនឹងរ៉ឺសរមួយ។ គេទាញវាចុះក្រោមរួចប្រលែង នៅខណៈនោះវាផ្លាស់ទីបានអំព្លីទុត $A=50cm$ ។ គេឲ្យពុលសាស្យុង $\omega=10rad/s$ ។
		\begin{enumerate}[k]
			\item គណនាប្រេកង់នៃចលនា។
			\item គណនាខួបនៃចលនា។
			\item សរសេរសមីការនៃលំយោល។
		\end{enumerate}
		\item ប៉ោលរ៉ឺសរដងដេកមួយធ្វើឡើងពីរ៉ឺសរមានថេរកម្រាញ $k=29.4N/m$ និងភ្ជាប់ដោយម៉ាសមួយ $m=0.30kg$ ។ ចូររកខួប និងប្រកង់នៃលំយោល។
		\item ឃ្លីមួយត្រូវបានចងភ្ជាប់ជាមួយនឹងខ្សែ $l=1.6m$ ព្យួរទៅនឹងបង្គោលឈរដោយដែកគោលរួចហើយធ្វើឃ្លីឲ្យវិលជាចលនាវង់ស្មើរាល់មួយវិនាទី $24$ ជុំ។ គណនាសំទុះរបស់គ្រាប់ឃ្លី ដោយគម្លាតមុំ របស់ខ្សែ $\alpha = 30^\circ$ ។
		\item អង្គធាតុមួយធ្វើចលនាអាកម៉ូនិចលើគន្លងត្រង់មួយជុំវិញទីតាំងលំនឹង $O$ ជាមួយនឹងខួប $T=0.3s$ ដោយដឹងថា $t=0$ អង្គធាតុមានអេឡុងកាស្យុង $x=-9cm$ ជាមួយនឹងល្បឿនដើមស្មើសូន្យ។
		\begin{enumerate}[k]
			\item សរសេរសមីការលំយោល។
			\item គណនាល្បឿនអតិបរមា។
		\end{enumerate}
		\item ប៉ោលរ៉ឺសរមួយយោលដោយអំព្លីទុត $4cm$ និងខួប $T=0.1s$។ សរសេរសមីការលំយោលរបស់ប៉ោលនោះ បើនៅខណៈពេល $t=0s$ ប៉ោលរ៉ឺសរនោះមានអេឡុង​​កា​​ស្យុង $x=2cm$។ គណនារយៈពេលខ្លីបំផុតដើម្បីឲ្យប៉ោលយោលពី $x_1=2cm$ ទៅ $x_2=4cm$ ។
		\item សមីការរបស់រួបធាតុមួធ្វើលំយោលអាកម៉ូនិចមានទម្រង់ $x=10\sin\left(5\pi + \frac{\pi}{6}\right)$ ។
		\begin{enumerate}[k]
			\item កំណត់ខួប ប្រេកង់មុំ អំព្លីទុត និងផាសដើមរបស់លំយោល។
			\item កំណត់អេឡុងកា​​​ស្យុង $x$ នៅពេលខណៈ $t=0.4s$។
			\item គណនាអេឡុងកា​​​ស្យុងពេលដែលផាសយោលបាន $-\frac{\pi}{4}$ ។
		\end{enumerate}
		\item គេចងព្យួរប៉ោលទី១ មានប្រវែង $l_1$ និងខួប $T_1=0.3s$ ហើយប៉ោលទី២ មានប្រវែង $l_2$ និងខួប $T_2=0.4s$។ ចូរគណនាខួបនៃប៉ោលទោលដែលមានប្រវែង $\left(l_1+l_2\right)$ នៅត្រង់កន្លែងនោះ។
		\item សរសេរសមីការផ្គួបនៃចលនាលំយោលអាកម៉ូនិចពីរដែលមានសមីការ $x_1=10\sin\left(\omega t -\frac{\pi}{6}\right)$ និង $x_2=10\sin\left(\omega t+ \frac{\pi}{3}\right)$ ដែល $x$ គិតជា $cm$ និង $t$ គិតជា $s$ ។ គេឲ្យ៖ ពុល​​សា​​ស្យុង$\omega = 50 rd/s$
	\end{enumerate}
	\borderline{ចប់}\\
	{\color{white}.}\dotfill\\
	{\color{white}.}\dotfill\\
	\begin{center}
		\sffamily\color{blue}
		សូមសំណាងល្អ!
	\end{center}\newpage
	\borderline{មេរៀនទី ៦ លំហាត់ រលក}
	\begin{enumerate}[m]
		\item គេឲ្យសមីការនៃបម្លាស់ទីរបស់ភាគល្អិតមួយកំណត់ដោយ៖
		 \begin{enumerate}[k, 2]
		 	\item $y=5\sin\left(\pi t + \frac{\pi}{3}\right)$ ដែល $y$ គិតជា $cm$ និង $t$ គិតជា $s$
		 	\item $y=3\sin\left(\pi t-\frac{\pi}{3}\right)$ ដែល $y$ គិតជា $cm$ និង $t$ គិតជា $s$
		 	\item $y=\left(5cm\right)\sin\left(5-\pi t\right)$ ដែល $t$ គិតជា $s$
		 	\item $y=5\sin2\left(3t-\frac{\pi}{3}\right)$ ដែល $y$ គិតជា $cm$ និង $t$ គិតជា $s$
		 	\item $y=3\cos\left(\pi t + 3\right)$ ដែល $y$ គិតជា $cm$ និង $t$ គិតជា $s$
		 	\item $y=-5\sin\left(2t+6\right)$ ដែល $y$ គិតជា $cm$ និង $t$ គិតជា $s$
		 \end{enumerate}
		 ចូរកំណត់ អំព្លីទុត ប្រកង់មំ មុំផាសដើមពេល ខួប និងប្រេកង់នៃបម្លាស់ទីនេះ។
		 \item ត្រចៀកមនុស្យអាចស្តាប់បានចាប់ពីប្រកង់ $20Hz$ ដល់ $20000Hz$ ។ \\
		 កំណត់ជំហានរលកនៃសម្លេងកម្រិត បើល្បឿនដំណាលសូរ $340m/s$ ។
		 \item ស្ថានីយ៍វិទ្យុផ្សាយចេញនូវប្រកង់ $760kHz$ ដោយល្បឿនរលកវិទ្យុ $3\times10^{8}m/s$ ។ គណនាជំហានរលកនៃរលកវិទ្យុនេះ។
		 \item ខ្សែមួយមានលំញ័រយោលដោយល្បឿន $\nu = 34.3m/s$ និងប្រេកង $f=262Hz$។ គណនាជំហានរលករបស់ខ្សែ។
		 \item ខ្សែយឺតមួយមានលំញ័រទទឹង ហើយមានទិសដៅវិជ្ជមានតាម $x$ ដោយមានជំហានរលក $\lambda=40cm$ និងអំព្លីទុត $a=15cm$ ហើយ និងប្រេកង់ $f=8Hz$ ដាលចេញពីគល់ទៅដល់ចំណុច $M$ ខណៈ $t$ និង $x=20cm$ ។
		 \begin{enumerate}[k]
		 	\item រកប្រេកង់មុំ ខួបលំយោល និងល្បឿនរបស់រលក។
		 	\item សរសេរកន្សោមសមីការចលនារបស់រលកនៅត្រង់ចំណុច $M$។
		 \end{enumerate}
	 	\item លំញ័រមួយចាប់ផ្តើមដាលពីចំណុច $A$ ដោយមានខួប $2s$ និងមានអំព្លីទុតថេរ $5cm$ ។
	 	\begin{enumerate}[k]
	 		\item នៅខណៈ $t=0$ លំយោលដាលចេញពីទីតាំងលំនឹងត្រង់ចំណុច $A$។ សរសេរសមីការលំយោលត្រង់ $A_y$ ។
	 		\item គេដឹងថាលំយោលដាលដោយល្បឿន $5m/s$ ។\\
	 		ចូរសរសេរសមីការលំយោលត្រង់ចំណុច $M$ មួយដែលស្ថិតចម្ងាយ $2.5m$ ពី $A$ ។
	 	\end{enumerate}
	 	\item ខ្សែមួយមានប្រវែង $5m$ និងមានម៉ាស $0.52kg$ ។ គេទាញវាឲ្យសន្ធឹងដោយកម្លាំង $46N$។ គណនា៖
	 	\begin{enumerate}[k]
	 		\item ល្បឿនដំណាលនៃលំញ័រមួយនៅលើខ្សែ។
	 		\item ប្រវែងរលកក្នុងករណីដែលគេដឹងថាលំញ័រមានប្រកង់ $400Hz$ ។
	 	\end{enumerate}
 		\item ប្រភពលំញ័រមួយមានសមីការចលនា $y=3\sin\left(125t + \frac{\pi}{3}\right)$ ។ ប្រភពនេះបញ្ចួនរលកដាលផុតខ្សែប្រវែង $25m$ ក្នុង $2.5s$ ។\\ តើល្បឿនដំណាល ខួប ប្រេកង់ និងប្រវែងរបស់រកមានតម្លៃប៉ុន្មាន?
 		\item ប្រភពលំញ័រមួយមានចលនាតាមសមីការ $y=4\sin\left(160t+\frac{\pi}{3}\right)$ ដែល $y$ គិតជា $cm$ និង $t$ គិតជា $s$ ។
 		ប្រភពនេះបញ្ជួនរលកដាលផុតខ្សែប្រវែង $30cm$ តែក្នុងរយៈពេល $3$ វិនាទី។
 		គណនាល្បឿនដំណាល $\nu$ ខួប $T$ និងជំហានរលក $\lambda$ ។
 		\item គេមានសមីការលំយោលសុីនុយសូអុីតពីរ $y_1=4\sin\left(5\pi t + \frac{\pi}{3}\right)$ និងមានជំហានរលក $\lambda_1=25cm$ និង $y_2=\sin\left(2\pi t + \frac{\pi}{3}\right)$ មានជំហានរលក $\lambda_2=30cm$។ ដែល $y_1$ និង $y_2$ គិតជា $cm;$ $t$ គិតជា $s$ ។
 		\begin{enumerate}[k]
 			\item កំណត់អំព្លីទុត ផាសដើម ប្រេកង់ និងខួបនៃលំយោលនីមួយៗ។
 			\item កំណត់ល្បឿននៃរលកនីមួយៗ។
 		\end{enumerate}
	\end{enumerate}
\borderline{ចប់}\\
\begin{center}
	\sffamily\color{blue}
	សូមសំណាងល្អ!
\end{center}\newpage
\borderline{មេរៀនទី ៧ លំហាត់ ចលនាក្នុងប្លង់}
	\begin{enumerate}[m]
		\item ចំណុចរូបធាតុមួយផ្លាស់ទីពីទីតាំងទី១ ដែល $\vec{r_1} = \left(-3.0m\right)\vec{i}+\left(2.0m\right)\vec{j}$ ទៅទីតាំងទី២ ដែល $\vec{r_2}=\left(9.0m\right)\vec{i}+\left(3.0m\right)\vec{j}$ ។\\
		រកបម្លាស់ទីរបស់ចំណុចរូបធាតុដែលផ្លាស់ទីពីទីតាំងទី១ ទៅទីតាំងទី២ ព្រមទាំងគូសក្រាបបញ្ជាក់ពីបម្លាស់ទី។
		\item តាមលំហាត់ទី១ បើចំណុចរូបធាតុនោះផ្លាស់ទីក្នុងរយៈពេល $\Delta t=2.0s$។ គណនាតម្លៃនៃវុិចទ័រល្បឿនមធ្យមនៃបម្លាស់ទីនោះ។
		\item ចំណុចរូបធាតុមួយផ្លាស់ទីពីចំណុច $A$ ដែល $\vec{r_A}\left[\left(0.0m\right)\vec{i}+\left(2.0m\right)\vec{j}\right]$ ទៅចំណុច $B$ ដែល $\vec{r_B}\left[\left(3.0m\right)\vec{i}+\left(6.0m\right)\vec{j}\right]$ ក្នុងរយៈពេល $2.0s$។
		\begin{enumerate}[k]
			\item គូសទីតាំង $A$ និងទីតាំង $B$ នៃចំណុចរូបធាតុ។
			\item គណនាបម្លាស់ទីពី $A$ ទៅ $B$។
			\item គណនាវិុទ័រល្បឿនមធ្យមរបស់ចំណុចរូបធាតុ។
		\end{enumerate}
		\item ចល័តមួយផ្លាស់ទីពីទីតាំងទី១ $x_1=\left(2+5t\right)m$ និង $y_1=\left(-4+2t\right)m$ ទៅទីតាំងទី២ $x_2=\left(4+5t\right)m$ និង $y_2=\left(-4-2t\right)m$។ គណនាបម្លាស់ទីនៃចល័តនោះនៅខណៈ $t=2.0s$ ។
		\item នៅខណៈ $t$ វិុទ័រល្បឿន $\vec{v}=\left(5.0m/s\right)\vec{i}+\left(2.0m/s\right)\vec{j}$ ។ ចូររកតម្លៃនៃវិុទ័រល្បឿននៅខណៈនោះ។
		\item គេចោលគ្រាប់ក្រូសមួយដោយល្បឿនដើម $v_0=2m/s$ ដែលមានទិសបង្កើតជាមួយទិសដេកបានមុំ $30^\circ$ ។
		\begin{enumerate}[k]
			\item សរសេរសមីការគន្លង
			\item គណនា $y$ បើ $x=2m$ ។
		\end{enumerate}
		\item នៅខណៈ $t=0$ គេទាត់បាល់មួយចេញពីចំណុច $0$ ដោយវុិចទ័រល្បឿនដែលមានទិសបង្កើតបានមុំ $45^\circ$ ធៀបនឹងអ័ក្សដេក $\vv{Ox}$ និងមានតម្លៃ $v=8.0m/s$ ។ គណនា៖
		\begin{enumerate}[k]
			\item ចម្ងាយធ្លាក់ ។
			\item កម្ពស់ឡើង​ ។
			\item ខណៈដែលបាល់ទៅដល់កំពូល $S$ នៃប៉ារ៉ាបូល និងកន្លែងបាល់ធ្លាក់ ។
		\end{enumerate}
		\begin{multicols}{2}
			\item យន្តហោះជួយសង្រ្គោះមួយ ហោះតាមទិសដេកដោយល្បឿនថេរ $180km/h$ នៅរយៈកម្ពស់ $490m$ ពីផ្ទៃទឹក។ អ្នកជួយសង្រ្គោះចង់ចាកចេញពីយន្តហោះទៅជួយស្រង់អ្នករងគ្រោះម្នាក់ដោយគាត់លិចទូក ដែលកំពុងព្យាយាមហែលទឹក។ គេចាត់ទុកកម្លាំងទប់នៃខ្យល់លើអ្នកជួយសង្រ្គោះអាចចោលបាន។
			\begin{enumerate}[k]
				\item តើមុំ $\alpha$ មានតម្លៃស្មើនឹងប៉ុន្មាន?
				\item នៅខណៈដែលអ្នកជួយសង្រ្គោះមកដល់ផ្ទៃទឹក\\ តើវិុចទ័រល្បឿនមានតម្លៃស្មើនឹងប៉ុន្មាន?
				តើវុិចទ័រល្បឿនមានទិសបង្កើតជាមួយខ្សែដេកបានមុំ $\theta$ មានតម្លៃស្មើនឹងប៉ុន្មាន? គេឲ្យ៖ $g=9.8m/s^2$
			\end{enumerate}
			\begin{figure}[H]
				\centering
				\begin{tikzpicture}[x=1.0cm,y=1.0cm, scale=.80]
				\draw [shift={(0,4)},color=xfqqff,fill=xfqqff,fill opacity=0.1] (0,0) -- (-90:0.53) arc (-90:-45:0.53) -- cycle;
				\draw [->,color=qqqqff] (0,4) -- (5,4);
				\draw [->,color=qqqqff] (0,4) -- (0,-1);
				\draw [->,color=qqqqff] (0,4) -- (5,-1);
				\draw [->,color=qqqqff] (-0.48,2.54) -- (-0.48,4.02);
				\draw [->,color=qqqqff] (-0.5,1.7) -- (-0.5,-0.04);
				\draw [->,color=qqqqff] (1.52,-0.34) -- (0.1,-0.36);
				\draw [->,color=qqqqff] (2.26,-0.34) -- (4.02,-0.34);
				\draw [->,color=qqqqff] (-0.02,4.24) -- (0.98,4.22);
				\draw (0.3,5.08) node[anchor=north west] {$\vv{v_0}$};
				\draw (-0.75,2.49) node[anchor=north west] {$h$};
				\draw (1.60,-0.01) node[anchor=north west] {$x$};
				\draw (4.0,-0.47) node[anchor=north west] {$ \vv{v} $};
				\draw (-0.60,4.40) node[anchor=north west] {$O$};
				\draw (-0.55,-0.55) node[anchor=north west] {$y$};
				\draw (4.60,3.92) node[anchor=north west] {$x$};
				\draw (0.17,3.44) node[anchor=north west] {$ \alpha $};
				\draw [color=qqqqff] (5,0)-- (0,0);
				\draw (4.50,0.009) node[anchor=north west] {$ \beta $};
				\draw [shift={(4.3,-0.13)}] plot[domain=-1.91:1.24,variable=\t]({1*0.14*cos(\t r)+0*0.14*sin(\t r)},{0*0.14*cos(\t r)+1*0.14*sin(\t r)});
				\end{tikzpicture}
				\caption{គន្លងចលនាអ្នកជួយសង្រ្គោះ}
			\end{figure}
		\end{multicols}
		\item អង្គធាតុមួយមានចលនាវង់ស្មើដោយល្បឿនថេរ $10m/s$។ គន្លងវង់នោះមានកាំ $15m$។\\ រកសំទុះចូរផ្ចិតនៃចលនារបស់អង្គធាតុនោះ។
		\newpage
		\item ចល័តមួយផ្លាស់ទីលើរង្វង់មួយដែលមានកាំ $5m$ ដោយចលនាស្មើ។ វាវិលបាន $2$ ជុំក្នុងរយៈពេល $4s$។
		\begin{enumerate}[k]
			\item រករយៈពេលដែលចល័តនោះវិលបានមួយជុំ។
			\item គណនាល្បឿនរង្វិលរបស់ចល័ត។
			\item គណនាសំទុះចូរផ្ចិត។
		\end{enumerate}
		\item ចល័តមួយផ្លាស់ទីតាមទិសដែលបង្កើតបានមុំ $30^\circ$ ជាមួយទិសដេក។ ដោយវិុចទ័រល្បឿន $v=35m/s$។ ចូររកវិុចទ័រល្បឿន $v_x$ តាមទិសដេក និងតាមទិសឈរ $v_y$។
		\item រថភ្លើងមួយផ្លាស់ទីក្នុងពេលមានភ្លៀងនិងខ្យល់សំដៅទិសខាងត្បូងដោយល្បឿនថេរ $27.0m/s$ ធៀបនឹងដី។ អ្នកសង្កេតម្នាក់ដែលឈរនៅលើដីឃើញតំណក់ទឹកភ្លៀងធ្លាក់មានទិសបង្កើតជាមួយទិសឈរបានមុំ $60^\circ C$ ។ អ្នកសង្កេតម្នាក់ទៀតនៅអង្គុយក្នុងរថភ្លើងឃើញតំណក់ទឹកភ្លៀងធ្លាក់តាមទិសឈរ។ ចូរកំណត់ល្បឿនតំណក់ទឹកភ្លៀងធ្លាក់ធៀបនឹងដី។
	\end{enumerate}
\borderline{ចប់}\\
\begin{center}
	\sffamily\color{blue}
	សូមសំណាងល្អ!
\end{center}\newpage
\borderline{មេរៀនទី ៨ លំហាត់ អនុវត្តន៍ច្បាប់ញូតុន}
\begin{enumerate}[m]
	\item ឃ្លីមួយមានម៉ាស $m=0.20kg$ ចងភ្ជាប់ទៅនឹងចុងម្ខាងនៃខ្សែដែលមានប្រវែង $1.0m$។ គេគ្រវីឃ្លីនោះដោយល្បឿនថេរ។ ឃ្លីមានចលនាវង់ស្មើក្នុងប្លង់ដេក។ បើតំណឹងខ្សែធំបំផុត $50.0N$ តើល្បឿនអតិបរមារបស់ឃ្លីមានតម្លៃស្មើនឹងបូន្មានមុននឹងធ្វើឲ្យខ្សែដាច់។
	\item 
\end{enumerate}
\borderline{ចប់}\\
\begin{center}
	\sffamily\color{blue}
	សូមសំណាងល្អ!
\end{center}\newpage
\end{document}