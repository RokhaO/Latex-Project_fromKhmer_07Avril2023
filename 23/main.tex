\documentclass[12pt, a4paper]{article}
%%import package named hightest
\usepackage{hightest}
\usepackage[export]{adjustbox}
\usepackage{wrapfig}
\usepackage{tkz-tab}
\usepackage{chemfig}
\usepackage[version=3]{mhchem}
%\usepackage{mathpazo}% change math font
%\usepackage[no-math]{fontspec}% font specfication
\header{រៀនគណិតវិទ្យាទាំងអស់គ្នា}{គីមីវិទ្យា}{\khmerdate}
\footer{រៀបរៀង និងបង្រៀនដោយ ស៊ុំ សំអុន}{ទំព័រ \thepage}{០៩៦ ៩៤០ ៥៨៤០}
\everymath{\protect\displaystyle\protect\color{blue}}
\begin{document}
\maketitle
\begin{enumerate}[m]
	\item ដូចម្តេចដែលហៅថាសូលុយស្យុងស្តង់ដា និង សូលុយស្យុងអត្រា ?
	\item តើអង្គធាតុអង្អុលពណ៍មាននាទីជាអ្វី ក្នុងអត្រាកម្ម ? ចូរប្រាប់អង្គធាតុចង្អុលពណ៍សំខាន់ៗដែលគេប្រើក្នុងអត្រាកម្ម ។
	\item គេធ្វើអត្រាកម្ម $25mL$ នៃសូលុយស្យុង $HCl$ ដោយសូលុយស្យុង $NaOH$ កំហាប់ $0.025M$ ។ ចំណុចសមមូលកើតមានកាលណាគេប្រើសូលុយស្យុង $NaOH$ អស់ $16.5mL$ ។ គណនាកំហាប់ជាម៉ូលនៃសូលុយស្យុង $HCl$ ?
	\item គេធ្វើអត្រាកម្ម $25mL$ នៃសូលុយស្យុង $HNO_3$ ដោយសូលុយស្យុង $NaOH$ ដែលមាន $pH=12.8$ ។ នៅពេលល្បាយសូលុយស្យុងទទួលាបានមាន $pH=7$ គេត្រូវប្រើសូលុយស្យុង $NaOH$ អស់ $18.2mL$ ។ គណនា $pH$ នៃសូលុយស្យុង $HCl$។
	\item គេធ្វើអត្រាកម្ម $25mL$ នៃសូលុយស្យុង $NaOH$ ដោយសូលុយស្យុង $HCl$ កំហាប់ $0.015M$ ។ អង្គធាតុចង្អុលពណ៍ប្រែពណ៍នៅពេលគេប្រើសូលុយស្យុង $HCl$ អស់ $12mL$ ។ គណនាកំហាប់ជាម៉ូលនៃសូលុយស្យុង $NaOH$ ។ 
	\item គេរំលាយក្រាម $NaOH$ ទៅក្នុងទឹកគេបានសូលុយស្យុង $S_1$ ដែលមានមាឌ $1L$ ។ បើគេយក $25mL$ នៃសូលុយស្យុង $S_1$ ទៅធ្វើអត្រាកម្មដោយ $HCl$ កំហាប់$0.02M$។ ចំណុចសមមូលកើតមានកាលណាគេប្រើសូលុយស្យុង $HCl$ អស់ $20mL$។
	\begin{enumerate}[k]
		\item គណនាកំហាប់ជាម៉ូលនៃសូលុយស្យុង $S_1$ 
		\item គណនាម៉ាសក្រាម $NaOH$ ដែលត្រូវប្រើដើម្បីទង្វើសូលុយស្យុង $S_1$
	\end{enumerate}
	\begin{center}
		\sffamily\color{black}
		សូមសំណាងល្អ!
	\end{center}\newpage
	\begin{center}
		\sffamily\color{black}
		\circled{០២}\\
		ជំពូក ៣ អត្រាកម្មអាស៊ីត-បាស(លំហាត់សុទ្ធ)
	\end{center}
	\item ដូចម្តេចដែលហៅថាចំណុចសមមូលអាស៊ីតបាស?
	\item នៅចំណុចសមមូលក្នុងអត្រាកម្ម តើមានវត្តមានអ្វីក្នុងសូលុយស្យុង?
	\item គេដាក់អត្រាករ $NaOH$ ចំនួន $72mL$ នៅកំហាប់ $0.55M$ ដើម្បីបន្សាបសូលុយស្យុងអាស៊ីត $HCl$ $220mL$ ។ចូរគណនាកំហាប់ $[H_3O^+]$ ក្នុងសូលុយស្យុងអាស៊ីតនេះ។
	\item គេដាក់ $29.5mL$ សូលុយស្យុង $HCl$ $0.15M$ ធ្វើប្រតិកម្មបន្សាបជាមួយបាស $25mL$ ចូរគណនា $[OH^-]$ ដែលមានក្នុងសូលុយស្យុងបាស $(NaOH)$ ។
	\item រកមាឌអាស៊ីតនីឌ្រិចកំហាប់ $0.25M$ ដែលត្រូវការដើម្បីបន្សាបសូលុយស្យុងប៉ូតាស្បូមអ៊ីដ្រុកស៊ីតចំនួន $17.35mL$ នៅ $0.195M$ ។
	\item រកតម្លៃ $pH$ នៃល្បាយដែលបានមកពីរប្រតិកម្មនៃសូលុយស្យុង $NH_4OH$ ចំនួន $25mL$ នៅកំហាប់ $0.05M$ ជាមួយសូលុយស្យុង $HNO_3$ ចំនួន $25mL$ នៅកំហាប់ $0.05M$ ។
	\item ដើម្បីបន្សាបសូ. $HCl$ ចំនួន $10mL$ នៅកំហាប់ $2\times10^{-3}M$ គេចាំបាច់បន្តក់សូ. $NaOH$ អស់មាឌ $20mL$ ទើបសម្រេចបានសមមូលអាស៊ីតបាស ។
	\begin{enumerate}[k]
		\item តើវាជាអត្រាកម្មអ្វី? ហើយគេប្រើអង្គធាតុចង្អុលពណ៍អ្វីសាកសមនឹងយកមកប្រើក្នុងអត្រាកម្មនេះ?
		\item រកកំហាប់ជាម៉ូលនៃសូលុយស្យុងចំាបាច់ដែលត្រូវប្រើ ។
	\end{enumerate}
	\begin{center}
		\sffamily\color{black}
		សូមសំណាងល្អ!
	\end{center}\newpage
	\begin{center}
		\sffamily\color{black}
		\circled{០៣}\\
		ជំពូក ៣ អត្រាកម្មអាស៊ីត-បាស(លំហាត់សុទ្ធ)
	\end{center}
	\item ដើម្បីបន្សាបសូ. $KOH$ (ប៉ូតាស) ចំនួន $50mL$ នៅកំហាប់ $10^{-3}M$ គេចាំបាច់ត្រូវបន្តក់សូ. $HNO_3$ មានកំហាប់ $2\times 10^{3}M$ រហូតសម្រេចបានសមមូលអាស៊ីតបាស។
	\begin{enumerate}[k,2]
		\item តើវាជាអត្រាកម្មអ្វី? 
		\item រកមាឌអាស៊ីតចាំបាច់ដែលត្រូវប្រើ
	\end{enumerate}
	\item គេបន្តក់សូលុយស្យុង $H_2SO_4$ នៅកំហាប់ $2\times10^{-3}M$ អស់មាឌ $10mL$ ទៅលើសូលុយស្យុង $NaOH$ $20mL$នៅកំហាប់មិនស្គាល់ រហូតមានសមមូលអាស៊ីតបាស។
	\begin{enumerate}[k]
		\item តើមួយណាជាូលុយស្យុងស្តង់ដា?
		\item សរសេរសមីការនៃលំនាំអត្រាកម្មនេះ ។
		\item រកកំហាប់សូលុយស្យុង $NaOH$ ចំបាច់ដែលត្រូវប្រើយកមកធ្វើអត្រា ?
	\end{enumerate}
	\item គេបន្តក់សូ. $HCl$ ចំនួន $50mL$ នៅកំហាប់ $2\times 10^{-2}M$ ទៅលើ $50mL$ នៃទឹកកំបោរថ្លាដែលមានកំហាប់មិនស្គាល់ ។
	\begin{enumerate}[k]
		\item រកកំហាប់ជាមូលនៃទឹកកំបោរថ្លាចំបាច់ដែលត្រូវយកមកធ្វើអត្រា ។
		\item គណនាម៉ាស $Ca(OH)_2$ ដែលចូរប្រតិកម្ម ។
	\end{enumerate}
	\item គេលាយ $100mL$ នៃសូ. $HCl$ នៅកំហាប់ $2\times 10^{-2}M$ ជាមួយ $150mL$ នៃសូ. $NH_4OH$ ដែលមានកំហាប់ $2\times10^{-2}M$ ។
	\begin{enumerate}[k]
		\item ចូរសរសេរសមីការប្រតិកម្មនេះ?
		\item តើអាស៊ីត ឬបាសដែលនៅសល់? ប៉ុន្មានម៉ូល?
	\end{enumerate}
	\begin{center}
	\sffamily\color{black}
	សូមសំណាងល្អ!
	\end{center}\newpage
	\begin{center}
		\sffamily\color{black}
		\circled{០៤}\\
		ជំពូក ៣ អត្រាកម្មអាស៊ីត-បាស(លំហាត់សុទ្ធ)
	\end{center}
	\item គេលាយ $20mL$ នៃសូ. $H_2SO_4$ ជាមួយសូ. $NH_4OH$ ចំនួន $20mL$ នៅកំហាប់ $2\times 10^{-3}M$ ។
	\begin{enumerate}[k]
		\item កំណត់ $pH$ នៃសូ. $NH_4OH$ មុនការលាយ ។
		\item រកកំហាប់អាស៊ីតចាំបាច់ដែលត្រូវប្រើ ។
	\end{enumerate}
	\item គេមានសូ. $S_1$ នៃម៉ូណូអាស៊ីត $HA$ មានកំហាប់ $5\times 10^{-2}M$ និងមាន$pH=1.3$។
	\begin{enumerate}[k]
		\item តើ $HA$ ជាអាស៊ីតខ្លាំង ឬខ្សោយ?
		\item សរសេរសមីការតាងប្រតិកម្មរវាង $HA$ ជាមួយ $H_2O$ និងសរសេរគូអាស៊ីតបាសរបស់វា ។
		\item រកមាឌទឹកដែលចាំបាច់ត្រូវថែមលើសូ. $S_1$ $25mL$ ដើម្បីឲ្យគេទទួលបានសូ. $S_2$ ដែលមាន $pH់=2$ ។
		\item គេបន្តក់សូ. $S_2$ ទៅលើ $20mL$ នៃសូ. $KOH$ រហូតដល់បានចំណុចសមមូលអាស៊ីតបាស គេចាំបាច់ប្រើសូ. $S_2$ អស់ $20mL$ ។ គណនាកំហាប់ $C_b$ នៃសូ. $KOH$ ។
	\end{enumerate}
	\item គេលាយ $20mL$ នៃសូ. $H_2SO_2$ នៅកំហាប់ $5\times 10^{-2}M$ ជាមួយ $30mL$ នៃកំហាប់ $5\times10^{-2}M$ នៃសូ.$NaOH$។
	\begin{enumerate}[k]
		\item ចូរសរសេរសមីការប្រតិកម្មនេះ។
		\item តើអាស៊ីត ឬបាសដែលនៅសល់?
		\item រកបរិមាណជាម៉ូលនៃ $H_3O^+$ ឬ $OH^-$ ដែលបាននៅសល់។
		\item រកមាឌអាស៊ីត ឬបាសដែលចាំបាច់ត្រូវបន្ថែមដើម្បីឲ្យសម្រេចបានសមមូលអាស៊ីតបាស។
	\end{enumerate}
\begin{center}
	\sffamily\color{black}
	សូមសំណាងល្អ!
\end{center}
	\newpage
	\begin{center}
		\sffamily\color{black}
		\circled{០៥}\\
		ជំពូក ៣ អត្រាកម្មអាស៊ីត-បាស(លំហាត់សុទ្ធ)
	\end{center}
	\item\begin{enumerate}[k]
		\item ដូចម្តេចដែលហៅថាអត្រាកម្ម?
		\item ចូរសរសេរទំនាក់ទំនងនៅសមមូល ក្នុងអត្រាកម្ម
		\begin{itemize}
			\item ម៉ូណូអាស៊ីតខ្លាំង ដោយម៉ូណូបាសខ្លាំង (ម៉ូណូ)
			\item ឌីអាស៊ីតខ្លាំង ដោយម៉ូណូបាសខ្លាំង
		\end{itemize}
	\end{enumerate}
	\item ក្នុងកែវបេស៊ែមួយមានដាក់សូ.អាស៊ីតនីឌ្រិច $20mL$ បន្ទាប់មកគេថែមសូ. ប៉ូតាស្យូមអ៊ីដ្រុកស៊ីត នៅកំហាប់ $C_b=2\times10^{-2}mol\cdot L^{-1}$ រហូតអស់មាឌ $V_b=16mL$ ទើបដល់ចំណុចសមមូលអាស៊ីត-បាស ។
	\begin{enumerate}[k]
		\item សរសេរសមីការតុល្យការប្រតិកម្ម
		\item រកកំហាប់ $C_a$ នៃសូ.អាស៊ីតនីឌ្រិច
		\item តើសូ.អាស៊ីតនីឌ្រិចមាន $pH$ ស្មើប៉ុន្មាន?
	\end{enumerate}
	\item គេរំលាយក្រាមសូដ្យូមអ៊ីដ្រុកស៊ីត $0.8g$ ក្នុងទឹកគេទទួលបានសូ. $S_1$ $500mL$ ។ គេបន្ថែមសូ.សូដ្យូមអ៊ីដ្រុកស៊ីត $S_2$ ដែលមាន $pH=12$ ចំនួន $1L$ ទៅក្នុងសូ. $S_1$ គេទទួលបានសូ. $S_3$ ។
	\begin{enumerate}[k]
		\item គណនាបរិមាណអ៊ីយ៉ុង $OH^-$ គិតជាម៉ូលដែលមានក្នុងសូ. $S_3$ ។
		\item គណនា $pH$ នៃសូលុយស្យុង $S_3$ ។\\
		គេឲ្យ៖ $\log2=0.3$ $K_e=1\times10^{-14}$
		\begin{center}
%			\sffamily\color{black}
			ប្រឡងថ្នាក់ជាតិៈ ០៥ សីហា ២០០៣
		\end{center}
	\end{enumerate}
	\begin{center}
		\sffamily\color{black}
		សូមសំណាងល្អ!
	\end{center}\newpage
	\begin{center}
		\sffamily\color{black}
		\circled{០៦}\\
		ជំពូក ៣ អត្រាកម្មអាស៊ីត-បាស(លំហាត់សុទ្ធ)
	\end{center}
	\item \begin{enumerate}[k]
		\item ដូចម្តេចដែលហៅថាចំណុចសមមូល?
		\item តើគេប្រើអង្គធាតុចង្អុលពណ៍អ្វី ក្នុងអត្រាកម្ម អាស៊ីតខ្លាំង ដោយបាសខ្លាំង? ព្រោះអ្វី?
	\end{enumerate}
	\item ក្នុងកែវមួយមានដាក់សូ.ទឹកកំបោរថ្លា $500mL$ បន្ទាប់មក គេថែមសូ. អាស៊ីតក្លរីឌ្រិច នៅកំហាប់ $0.2M$ អស់ចំនួន $100mL$ ទើបទទួលបាន សមមូលអាស៊ីត បាស។
	\begin{enumerate}[k]
		\item សរសេរមីការតុល្យការតាងប្រតិកម្ម ។
		\item ចូរកំណត់រកកំហាប់ $C_b$ នៃសូ.ទឹកកំបោរថ្លា ។
		\item រកម៉ាសអំបិលដែលកើតក្រោយប្រតិកម្ម ។\\
		គេឲ្យ៖ $Ca=40 $ $Cl=35.5$
	\end{enumerate}
	\item គេយកសូ.អាស៊ីតក្លរីឌ្រិច $(S_1)$ ដែលមានកំហាប់ $0.1M$ ចំនួន $20mL$ មកបន្ថែមលើទឹកបិតរហូតបានមាឌ $50mL$ ។
	\begin{enumerate}[k]
		\item រកកំហាប់របស់សូ.អាស៊ីតថ្មី $(S_1)$ និងគណនាតម្លៃ $pH$ របស់វា ។
		\item គេបន្ថែមសូ.ស៊ូតដែលមានកំហាប់ $0.02M$ ទៅលើសូ. $S_2$ $200mL$ ។ តើគេត្រូវប្រើមាឌសូ.ស៊ូតអស់ប៉ុន្មាន $mL$ ដើម្បីទៅដល់ចំណុចសមមូល? $\log4=0.6$
%		\begin{center}
%			ប្រឡងឆមាសលើកទី២ៈ ២៥ មិថុនា ២០១២  (សង្គម)
%		\end{center}
	\end{enumerate}
	\item \begin{enumerate}[k]
		\item ចូរសរសេរសមីការតាងប្រតិកម្មរវាងអាស៊ីតក្លរីឌ្រិច($HCl$) ជាមួយទឹក។
		\item គណនាកំហាប់ប្រភេទគីមីដែលមានវត្តមាននៅក្នុងសូ.អាស៊ីតក្លរីឌ្រិច ដែលមាន $pH=2$។
		\item គេបន្ថែម $4mL$នៃសូ. $NaOH$ ដែលមានកំហាប់ $C_b=1.5\times10^{-2}mol\cdot L^{-1}$ ទៅលើ $20mL$នៃសូ.$HCl$ខាងលើ។ គណនា $pH$នៃសូ.ដែលទទួលបាន។
	\end{enumerate}
\begin{center}
	\sffamily\color{black}
	សូមសំណាងល្អ!
\end{center}
	\newpage
	\begin{center}
		\sffamily\color{black}
		\circled{០៧}\\
		ជំពូក ៣ អត្រាកម្មអាស៊ីត-បាស(លំហាត់សុទ្ធ)
	\end{center}
	\item \begin{enumerate}[k]
		\item ក្នុងអត្រាកម្ម អាស៊ីតខ្សោយ ដោយបាសខ្លាំង តើគេត្រូវប្រើអង្គធាតុចង្អុលពណ៍មួយណា? ព្រោះអ្វី?
		\item តើ $[H_3O^+]$ របស់ទឹកសុទ្ធនៅសីតុណ្ហភាព $25^\circ C$ ស្មើប៉ុន្មាន? តើវាមានតម្លៃ ដូចនេះគ្រប់សីតុណ្ហភាព ឬទេ?
	\end{enumerate}
	\item គេលាយល្បាយ $20mL$ នៃសូលុយស្យុងអាស៊ីតក្លរីឌ្រិច($HCl$) ដែលមានកំហាប់ $5\times10^{-2}M$ និង $19mL$ នៃសូលុយស្យុងស៊ូតដែលមានកំហាប់ $5\times10^{-2}M$ ។
	\begin{enumerate}[k]
		\item គណនា $pH$ នៃសូ.នីមួយៗមុនពេលប្រតិកម្ម?
		\item សរសេរសមីការតាងប្រតិកម្ម?
		\item ចូរកំណត់មជ្ឈដ្ឋាននៃសូ.ទទួលបាន និងគណនា $pH$ នៃល្បាយទទួលបាន?
		\item តើគេត្រូវបន្ថែមសូ.អាស៊ីត ឬបាសប៉ុន្មាន $mL$ ទៀតទើបគេទទួលបាន សូ.មួយមាន $pH=7$ ។
	\end{enumerate}
	\item សូ.អាស៊ីតនីឌ្រិច $HNO_3$ ចំនួន$25mL$ មាន $pH=2.5$។
	\begin{enumerate}[k]
		\item គណនាបរិមាណអ៊ីយ៉ុងអ៊ីដ្រូញ៉ូមដែលមានក្នុងសូ.អាស៊ីតននេះ?
		\item គេបន្ថែម $25mL$ នៃសូ.កាល់ស្យូមអ៊ីដ្រុកស៊ីត $Ca(OH)_2$ ដែលមានកំហាប់ $10^{-2}M$ ទៅក្នុងសូ.អាស៊ីតនីឌ្រិចខាងលើ។
		\begin{enumerate}[1]
			\item សរសេរសមីការតាងប្រតិកម្មដែលកើតមានឡើង?
			\item គណនា $pH$ នៃសូ.ដែលទទួលបាន?
		\end{enumerate}
	\end{enumerate}
	\begin{center}
		\sffamily\color{black}
		សូមសំណាងល្អ!
	\end{center}\newpage
	\begin{center}
		\sffamily\color{black}
		\circled{០៨}\\
		ជំពូក ៣ អត្រាកម្មអាស៊ីត-បាស(លំហាត់សុទ្ធ)
	\end{center}
	\item \begin{enumerate}[k]
		\item គេចង់ធ្វើសូ.ស៊ូតចំនួន $500mL$ កំហាប់ $2\times10^{-2}M$ ។ តើគេត្រូវប្រើ សូដ្យូមអ៊ីដ្រុកស៊ីត $NaOH$ ប៉ុន្មានក្រាម និងគណនា $pH$ នៃសូ.?
		\item គេយកសូ.ខាងលើ $10mL$ ទៅដាក់ក្នុងកែវបេស៊ែមួយរួចគេបន្តក់ សូ.អាស៊ីតក្លរីឌ្រិចកំហាប់ $5\times10^{-3}M$។
		\begin{enumerate}[a]
			\item សរសេរសមីការតុល្យការតាងប្រតិកម្ម និងបង្ហាញអ៊ីយ៉ុងទស្យនិក?
			\item កំណត់មាឌ $V_a$ នៃសូ.អាស៊ីតដើម្បីបានល្បាយមួយមាន $pH=7$។
		\end{enumerate}
	\end{enumerate}
	\item គេយក $10mL$នៃសូ. $(H_3O^+ ; Cl^-)$ កំហាប់ $5.5\times10^{-2}M$ ទៅលាយជាមួយ $5mL$ នៃសូ.ស៊ូត $NaOH$ ដែលមានកំហាប់ $5\times10^{-2}M$។
	\begin{enumerate}[k]
		\item សរសេរមីការតាងប្រតិកម្មដែលកើតមាន?
		\item ចូរកំណត់មជ្ឈដ្ឋានសូ.ដែលទទួលបាន?
		\item គណនា $pH$ នៃសូ.ដែលទទួលបានក្រោយប្រតិកម្ម?
		\item គណនាកំហាប់ប្រភេទគីមីដែលមានវត្តមាននៅក្នុងសូ.ក្រោយប្រតិកម្ម?
	\end{enumerate}
	\item គេរំលាយក្រាមសូដ្យូមអ៊ីដ្រុកស៊ីត $NaOH$ ចំនួន $0.8g$ ទៅក្នុងទឹក $500mL$ គេទទួលបានសូ. $S_1$ ដែលមាន $pH=12.6$ ។
	\begin{enumerate}[k]
		\item តើសូ.ស៊ូតជាបាសខ្លាំង ឬខ្សោយ?
		\item គេចាក់សូ.អាស៊ីត ក្លរីឌ្រិចដែលមានកំហាប់ $2\times10^{-3}M$ ទៅក្នុងសូ.$S_1$ រហូតដល់សូ.ក្រោយប្រតិកម្មមាន $pH=7$ ។ គណនា $V_a$ ដែលត្រូវប្រើ?
	\end{enumerate}
	\begin{center}
		\sffamily\color{black}
		សូមសំណាងល្អ!
	\end{center}\newpage
	\begin{center}
		\sffamily\color{black}
		\circled{០៩}\\
		ជំពូក ៣ អត្រាកម្មអាស៊ីត-បាស(លំហាត់សុទ្ធ)
	\end{center}
	\item ក្នុងកែវបេស៊ែមួយមានដាក់ $50mL$ នៃសូ. $HNO_3$ នៅកំហាប់ $3\times10^{-2}M$។ បន្ទាប់មកគេថែមសូ.ស៊ូតនៅកំហាប់ $2\times10^{-2}M$។
	\begin{enumerate}[k]
		\item សរសេរមីការតុល្យការតាងប្រតិកម្ម។
		\item តើសមមូលអាស៊ីត បាសកើតឡើង ឬទេ បើគេបន្ថែមសូ.ស៊ូត $50mL$?
	\end{enumerate}
	\item គេយកសូ.អាស៊ីតក្លរីឌ្រិច $5mL$ ទៅធ្វើអត្រាកម្មដោយសូ.ស៊ូត $15mL$ ចូរ គេឃើញអង្គធាតុចង្អុលពណ៍ប្រែពណ៍។
	\begin{enumerate}[k]
		\item គូសគំនូសបំព្រួញនៃការធ្វើអត្រាកម្មនេះ។
		\item តើអង្គធាតុចង្អុលពណ៍ណាមួយ ដែលសមស្របជាងគេសម្រាប់ធ្វើអត្រាកម្មនេះ?
		\item សរសេមីការតុល្យការប្រតិកម្មនៃអត្រាកម្ម និងគណនាកំហាប់សូ.អាស៊ីតក្លរីឌ្រិច។
		\item គេយកសូ.អាស៊ីក្លរីឌ្រិច $5mL$ ដដែលទៅលាយជាមួយទឹក $10$ដង ដើម្បីធ្វើអត្រាកម្ម។ តើមាឌសូ.ស៊ូតដែលប្រើប្រែប្រួល ឬទេ?
	\end{enumerate}
	\item សូ.អាស៊ីត $HA$ មួយមានកំហាប់ $C_a=1.0\times10^{-2}M$ និងមាន $pH=3.4$។
	\begin{enumerate}[k]
		\item តើ $HA$ ជាអាស៊ីតខ្លាំង ឬខ្សោយ? ចូរពន្យល់
		\item សរសេរសមីការប្រតិកម្ម រវាង $HA$ និងទឹក។ ចូររាប់ប្រភេទគីមីក្នុងសូ.ក្រោយប្រតិកម្ម។ $10^{0.6}=4$
	\end{enumerate}
	\begin{center}
		\sffamily\color{black}
		សូមសំណាងល្អ!
	\end{center}\newpage
	\begin{center}
		\sffamily\color{black}
		\circled{១០}\\
		ជំពូក ៣ អត្រាកម្មអាស៊ីត-បាស(លំហាត់សុទ្ធ)
	\end{center}
	\item សូ.អាស៊ីត $HA$ មួយមានកំហាប់ $C_a=5\times10^{-2}M$ និងមាន $pH=1.3$។
	\begin{enumerate}[k]
		\item តើ $HA$ ជាសអាស៊ីតខ្លាំង ឬខ្សោយ? ចូរពន្យល់
		\item សរសេរមីការប្រតិកម្ម រវាង $HA$ និងទឹក។ ចូររាប់ប្រភេទគីមីក្នុងសូ.ក្រោយប្រតិកម្ម។ $10^{0.7}=5$
		\item គេយកសូ.$HA$ $10mL$ទៅថែមទឹកសុទ្ធ គេបានសូ.អាស៊ីតថ្មី ដែលមាន\\$pH=2$។  រកមាឌទឹកដែលត្រូូវថែម ។
	\end{enumerate}
	\item ទឹកកំបោរថ្លាគឺជាសូ.ឆ្អែតនៃកាល់ស្យូមអ៊ីដ្រុកស៊ីតដែលចាត់ទុកជាឌីបាសខ្លាំង។
	\begin{enumerate}[k]
		\item សរសេរសមីការបំបែកជាអ៊ីយ៉ុងនៃកាល់ស្យូមអ៊ីដ្រុកស៊ីត
		\item គណនាម៉ាសកាល់ស្យូមអ៊ីដ្រុកស៊ីត ដែលរលាយចូរក្នុងទឹក $1L$ នៅសីតុ. $25^\circ C$ បើសូ.មាន$pH=12.6$ ។
		\item គេយក $40mL$ នៃសូ.អាស៊ីតក្លរីឌ្រិចដែលមានកំហាប់ $0.1M$ បន្តក់ចូរទៅក្នុងទឹកកំបោរ។
		ចូរសរសេរសមីការតុល្យការ និងគណនាមាឌសូ.ទឹកកំបោរដែលធ្វើប្រតិកម្មដើម្បីទទួលបានសមមូល។
	\end{enumerate}
	\item ក្នុងកែវមួយមានដាក់សូ. $HCl$ $10mL$ នៅកំហាប់ $C_a=3\times10^{-2}M$ បន្ទាប់មកគេថែមសូ. $NaOH$ $10mL$ នៅកំហាប់ $C_b=2\times10^{-2}M$។
	\begin{enumerate}[k]
		\item សរសេរមីការតុល្យការប្រតិកម្ម។
		\item តើសូ.ទទួលបានក្រោយប្រតិកម្មជាសូ.អាស៊ីត ឬបាស ឬណឺត?
		\item គណនា$pH$ នៃល្បាយក្រោយប្រតិកម្ម។ $\log5=0.7$
	\end{enumerate}
	\begin{center}
		\sffamily\color{black}
		សូមសំណាងល្អ!
	\end{center}\newpage
	\begin{center}
		\sffamily\color{black}
		\circled{១១}\\
		ជំពូក ៣ អត្រាកម្មអាស៊ីត-បាស(លំហាត់សុទ្ធ)
	\end{center}
	\item 
	\begin{center}
		\sffamily\color{black}
		សូមសំណាងល្អ!
	\end{center}\newpage
\end{enumerate}
\end{document}