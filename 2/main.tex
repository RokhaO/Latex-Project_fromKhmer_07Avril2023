\documentclass[12pt, a4paper]{article}
%%import package named hightest
\usepackage{hightest}
%\usepackage{mathpazo}% change math font
%\usepackage[no-math]{fontspec}% font specfication
\header{រៀនគណិតវិទ្យាទាំងអស់គ្នា}{គណិតវិទ្យា}{២៥/០១/២០១៨}
\footer{រៀបរៀង និងបង្រៀនដោយ ស៊ុំ សំអុន}{ទំព័រ \thepage}{០៩៦ ៩៤០ ៥៨៤០}
\everymath{\protect\displaystyle\protect\color{black}}
\begin{document}
\maketitle
\begin{enumerate}
	\item\textbf{(១៥ ពិន្ទុ)}~គេមានអនុគមន៍ $h(x)=\frac{e^{-2x^2}-2\cos2x+1}{x^2}$ ចំពោះ $x\neq0$
		និង $h(0)=2\ln(e^k+1)$។\\ កំណត់ចំនួនពិត $k$ ដើម្បីឲ្យ $h(x)$ ជាប់ត្រង់ $x=0$ ។
	\item\textbf{(១៥ ពិន្ទុ)}~គេឲ្យចំនួនកំុផ្លិច $ z=(1-i)(\cos\frac{\pi}{6}+i\sin\frac{\pi}{6})$~ និង $w=(1+i)^2$~។
	\begin{enumerate}[m]
		\item សរសេរ $z$ និង $w$ ជាទម្រង់ពីជគណិត ។
		\item សរសេរ $z\times w$ និង $\frac{z}{w}$~ ជាទម្រង់ត្រីកោណមាត្រ ។
		\item សរសេរប្ញសទី៤ទាំងអស់នៃ $w$~ជាទម្រង់ពីគណិត ។
	\end{enumerate}
	\item\textbf{(១៥ ពិន្ទុ)}~ គណនាលីមីតនៃអនុគមន៍ខាងក្រោម៖
		\begin{enumerate}[k,3]
			\item $\lim\limits_{x\to+\infty}\frac{x^2+\ln x}{3x^2-1}$~
			\item $\lim\limits_{x\to+\infty}\frac{x^3-e^x}{2e^x+3x+1}$~
			\item $\lim\limits_{x\to0}\frac{(e^{-3x+2}-e^2)\sin(\pi x)}{4x^2}$~
		\end{enumerate}
	\item\textbf{(១៥ ពិន្ទុ)}~
		\begin{enumerate}[m]
			\item ចំពោះអនុគមន៍ $y=\cos x$~ចូរបង្ហាញថា $y+y'+y''+y'''=0$~។
			\item គណនាដេរីវេ $f'(x)$~នៃ $f(x)=\sin(\sin x)+\cos(\sin x)$~។
			\item កំណត់តម្លៃធំបំផុតនៃ $S=6x-3x^2$~។
		\end{enumerate}
	\item\textbf{(៣០ ពិន្ទុ)}~អនុគមន៍ $f$ កំណត់ចំពោះ $x\neq2$~ដោយ $f(x)=\frac{ax^2+bx+c}{(x-2)^2}$~ហើយមានក្រាប $C$~។
	\begin{enumerate}[m]
		\item រកចំនួនពិត $a,b$ និង $c$ ដោយដឹងថាក្រាប $C$ កាត់អ័ក្សអាប់ស៊ីសត្រង់ $x_1=1, x_2=3$ និងមានអាស៊ីមតូតដេកជាបន្ទាត់ $y=1$~។
		\item រកសមីការអាស៊ីមតូតឈរនៃក្រាប $C$ និងសង់តារាងអថេរភាពនៃអនុគមន៍ $f$~ដែលបានកំណត់ក្នុងសំណួរទី១។
		\item សង់ក្រាប $C$~នៃអនុគមន៍ $f$~។
	\end{enumerate}
	\item\textbf{(៣៥ ពិន្ទុ)}~ក្នុងលំហរប្រដាប់ដោយតម្រុយអតូណរម៉ាល់មានទិសដៅវិជ្ជមាន $ (O,\vec{i},\vec{j},\vec{k}) $ គេមានចំណុច​\\
	$ A(2,0,1),B(0,1,3) $ និង $ C(0,3,2) $~។
		\begin{enumerate}[m]
			\item គណនាកូអដោនេនៃវ៉ិចទ័រ$\overrightarrow{AB}$ និង $\overrightarrow{BC}$~។ បង្ហាញថាវ៉ិចទ័រ$\overrightarrow{AB}$ និង $\overrightarrow{BC}$~ជាវ៉ិចទ័រអរតូកូណាល់~។
			\item គណនាផលគុណវ៉ិចទ័រ $\overrightarrow{AB}$ និង $\overrightarrow{BC}$~។
			\item សរសេរសមីការប៉ារ៉ាមែត្រនៃបន្ទាត់ $(D)$~ដែលកាត់តាម $C$~ហើយស្របនឹងវ៉ិចទ័រ$\overrightarrow{AB}$~។
			\item រកសមីការប្លង់ $(P)$~ដែលកាត់តាម $A$~និងមានវ៉ិចទ័រណរម៉ាល់ $\overrightarrow{BC}$~។
			\item រកសមីការស្វ៊ែ $(S)$ ដែលមានអង្គត់ផ្ចិត $[AC]$~។ ផ្ទៀងផ្ទាត់ថាចំណុច $B$ ជាចំណុចរបស់ស្វ៊ែ $(S)$~។
		\end{enumerate}
	\end{enumerate}
	\begin{center}
		\sffamily\color{black}
		សូមសំណាងល្អ!
	\end{center}\newpage
\end{document}