\documentclass{officialexam} 
\begin{document}
	{\maketitle}
	\borderline{ប្រធាន}
	\begin{enumerate}[I]
		\item គណនាលីមីតនៃអនុគមន៍ខាងក្រោម៖
		\begin{enumerate}[k,3]
			\item $\lim_{x\to3}\frac{x^4+6x+1}{x^2+1}$
			\item $\lim_{x\to+\infty}\frac{x-1}{\left(x+1\right)^2}$
			\item $\lim_{x\to+\infty}\left(x^2+2-\ln x\right)$
		\end{enumerate}
		\item ក្នុងថ្នាក់រៀនមួយមានសិស្ស $15$ នាក់ ក្នុងនោះសិស្យប្រុស $9$ នាក់ និងសិស្សស្រី $6$ នាក់ ។ \\ 
		គេជ្រើសរើសសិស្ស $3$ នាក់ ដោយចៃដន្យជាតំណាងទៅសម្ភាសន៍ ។ គណនាប្រូបាបនៃព្រឹត្តិការណ៍ខាងក្រោម៖
		\begin{enumerate}[A]
			\item ក្រុមសិស្សទាំង $3$ នាក់ សុទ្ធតែជាសិស្សស្រី ។
			\item ក្រុមសិស្សទាំង $3$ នាក់ សុទ្ធតែជាសិស្សប្រុស ។
			\item ក្រុមសិស្សទាំង $3$ នាក់ មាន $2$ នាក់ជាសិស្សស្រី ។
		\end{enumerate}
		\item គណនាអាំងតេក្រាលខាងក្រោម៖ 
		\begin{enumerate}[k,3]
			\item $\mathrm{I}=\int_{1}^{2}\left(3x^2-2x+3\right)dx$
			\item $\mathrm{J}=\int_{0}^{1}\left(e^{2x}-e^x+1\right)dx$
			\item $\mathrm{K}=\int_{1}^{2}\left(\frac{1}{x+3}+\frac{1}{x^2}\right)dx$
		\end{enumerate}
		\item គេមានប៉ារ៉ាបូលមួយដែលមានកំពូលជាចំណុច $O\left(0,0\right)$ និងកំណុំ $F$ ស្ថិតនៅលើអ័ក្សអាប់ស៊ីស ។
		\begin{enumerate}[k]
			\item រកសមីការស្តង់ដានៃប៉ារ៉ាបូលនេះ បើគេដឹងថាវាកាត់តាមចំណុច $A\left(\frac{3}{2};-3\right)$ ។
			\item រកកូអរដោនេរបស់កំណុំ សមីការបន្ទាត់ប្រាប់ទិស រួចសង់ប៉ារ៉ាបូលនេះ ។
		\end{enumerate}
		\item គេមានអនុគមន៍ $f$ កំណត់ដោយ $f(x)=\frac{2x^2-7x+5}{x^2-5x+7}$ ។ យើងតាងដោយក្រាប $C$ របស់វាលើតម្រុយអរតូណរម៉ាល់ $\left(O, \vec{i}, \vec{j}\right)$ ។
		\begin{enumerate}[1]
			\item រកដែនកំណត់ $\mathbb{D}$ នៃអនុគមន៍ $f$ ។
			\item សិក្សាលីមីតនៃអនុគមន៍ $f(x)$ ត្រង់ $-\infty$ និងត្រង់ $+\infty$ ។ ទាញរកសមីការអាស៊ីមតូត $d$ ទៅនឹងក្រាប $C$ ត្រង់ $-\infty$ និង $+\infty$ ។
			\item \begin{enumerate}[k]
				\item ស្រាយបំភ្លឺថាគ្រប់ចំនួនពិត $x\in\mathbb{D}~,$ ដេរីវេ $f'(x)=\frac{-3\left(x^2-6x+8\right)}{\left(x^2-5x+7\right)}$ ។
				\item សិក្សាអថេរភាពនៃអនុគមន៍ $f$ និងសង់តារាអថេរភាពនៃអនុគមន៍ $f$ ។
				\item សង់ក្រាប $C$ នៃអនុគមន៍ $f$ ។
			\end{enumerate}
		\end{enumerate}
	\end{enumerate}
\end{document}