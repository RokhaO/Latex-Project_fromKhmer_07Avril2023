\documentclass[a4paper, 12pt]{exam}
\makeatletter
\usepackage[top=0.5cm, left=1cm, bottom=1.8cm, right=1.5cm]{geometry}
\usepackage{amsmath,amssymb}
\usepackage{gensymb}
\usepackage{textcomp}
\usepackage{tcolorbox}
\usepackage[export]{adjustbox}
\usepackage{graphicx}
\usepackage{wrapfig}
\usepackage{pgf}
\usepackage{tikz}% graphic drawing
\usetikzlibrary{arrows}
\pagestyle{empty}
\usepackage{wasysym}
\usepackage{mathpazo}% change math font
\usepackage{enumitem}% change list environment like enumerate, itemize and description
\usepackage{multicol}% multi columns
\usepackage{xcolor}
\newcommand{\teacher}{ស៊ំុ សំអុន}
\newcommand{\tell}{០៩៦ ៩៤០ ៥៨៤០}
\newcommand{\class}{ត្រៀមប្រឡង~ឆមាសលើកទី~០១}
\newcommand{\dateofexam}{សម័យប្រឡង៖~~~\dots\dots~~~\dots\dots ~~~២០១៨}
\newcommand{\subject}{វិញ្ញាសា៖~រូបវិទ្យា~(វិទ្យាសាស្រ្តពិត)}
\newcommand{\timelimit}{៩០~នាទី}
\newcommand{\score}{ពិន្ទុសរុប៖~៧៥~ពិន្ទុ}
\usepackage[no-math]{fontspec}% font specfication
\setmainfont{Khmer OS Content}% set default font to Khmer OS
\setsansfont[Ligatures=TeX,AutoFakeBold=0,AutoFakeSlant=0.25]{Khmer OS Muol Light}% sans serif font
%
\newcommand{\heart}{\ensuremath\heartsuit}
\newcommand{\butt}{\rotatebox[origin=c]{180}{\heart}}
\newcommand*\circled[1]{\tikz[baseline=(char.base)]{
		\node[shape=circle,draw,inner sep=2pt] (char) {#1};}}
%
\SetEnumitemKey{I}{%
	leftmargin=*,
	label={\protect\tikz[baseline=-0.9ex]\protect\node[draw=gray,thick,circle,minimum height=.65cm,inner sep=1pt,text=black,fill=white]{\Roman*};},%
	font=\small\sffamily\bfseries,%
	labelsep=1ex,%
	topsep=0pt}
%
\SetEnumitemKey{a}{%
	leftmargin=*,%
	label={\protect\tikz[baseline=-0.9ex]\protect\node[draw=gray,thick,circle,minimum height=.5cm,inner sep=1pt,text=blue,fill=magenta!5!white]{\alph*};},%
	font=\small\sffamily\bfseries,%
	labelsep=1ex,%
	topsep=0pt}
%
\SetEnumitemKey{1}{leftmargin=*,%
	label={\protect\tikz[baseline=-0.9ex]\protect\node[draw=gray,thick,circle,minimum height=.5cm,inner sep=1pt,text=black,fill=cyan!20!white]{\arabic*};},%
	font=\small\sffamily\bfseries,%
	labelsep=1ex,%
	topsep=0pt}
%
\def\hard{\leavevmode\makebox[0pt][r]{\large\ensuremath{\star}\hspace{2em}}}
%
\def\hhard{\leavevmode\makebox[0pt][r]{\large\ensuremath{\star\star}\hspace{2em}}}
%
\everymath{\protect\displaystyle\protect\color{black}}
%
\pagecolor{cyan!1!white}
%
\usepackage{amsmath}
\usepackage{amssymb}
\usepackage{wasysym}
\makeatother

\pagestyle{foot}
\firstpagefooter{}{ ទំព័រ~\thepage\ នៃ \numpages}{}
\runningfooter{រៀបរៀងដោយ~\teacher}{ ទំព័រ \thepage\ នៃ \numpages}{ទូរស័ព្ទ~\tell}
\runningfootrule

\begin{document}
\noindent
%\sffamily\color{black}
\begin{tabular*}{\textwidth \sffamily\color{black}}{l @{\extracolsep{\fill}} r @{\extracolsep{6pt}} l}
\textbf{\class} & \textbf{មណ្ឌលប្រឡង} & \makebox[2in]{\hrulefill}\\
\textbf{\dateofexam} & \textbf{លេខបន្ទប់} & \makebox[2in]{\hrulefill}\\
\textbf{\subject} & \textbf{លេខតុ} & \makebox[2in]{\hrulefill}\\
\textbf{\score} & \textbf{ឈ្មោះបេក្ខជន} & \makebox[2in]{\hrulefill}\\
\textbf{រយៈពេលសរុប៖ \timelimit} & \textbf{ហេត្ថលេខា} & \makebox[2in]{\hrulefill}
\end{tabular*}
\noindent
\rule[2ex]{\textwidth\color{magenta}}{2pt}
\begin{center}
	\sffamily\color{black}
	ប្រធានលំហាត់ទី០២\\
\end{center}
\begin{enumerate}[I]
	\item ($15$ ពិន្ទុ)
	\begin{enumerate}[1]
		\item គណនាម៉ាសម៉ូលេគុលអ៊ីដ្រូសែននីមូួយៗ បើអ៊ីដ្រូសែនមានម៉ាសម៉ូល $M=2\times10^{-3}Kg/mol$~។
		\item គណនាល្បឿនប្ញសការេនៃការេល្បឿនមធ្យមម៉ូលេគុលឧស្ម័នអ៊ីដ្រូសែននៅសីតុណ្ហភាព $27^\circ C$ ។
		\item គណនាថាមពលស៊ីនេទិចមធ្យមនៃឧស្ម័នអ៊ីដ្រូសែននីមួយៗនៅសីតុណ្ហភាព $127^\circ C$~។
	\end{enumerate}
	\item($10$ ពិន្ទុ) ស៊ីឡាំងមួយចែកចេញជាពីផ្នែកដែលខណ្ឌដោយរបាំងពិស្តុងមានកម្រាស់ស្តើងអាចចោលបាន ។ \\ផ្នែកទីមួយដាក់ឧស្ម័នបរិសុទ្ធមួយប្រភេទដែលមានចំនួនម៉ូល $n_1$ ឯផ្នែកទីពីរដាក់ឧស្ម័នបរិសុទ្ធមួយប្រភេទទៀត\\ដែលមានចំនួនម៉ូល $n_2$ ។ គេដឹងថាក្នុងស៊ីឡាំងឧស្ម័នទាំងពីរប្រភេទមានចំនួនម៉ូលសរុប $n=21mol$ មានសីតុណ្ហភាព និងសម្ពាធដូចគ្នា ។\\	គណនា $n_1$ និង $n_2$ បើប្រវែង $l_1=60cm$ និង $l_2=10cm$ ។
	\item($10$ ពិន្ទុ) ប្រព័ន្ធឧស្ម័នមួយស្រូបកម្ដៅ $5000J$ ប្រព័ន្ធថយចុះថាមពលក្នុងអស់ $1500J$ ។ 
	\begin{enumerate}[a]
		\item គណនាកម្មន្តដែលប្រព័ន្ធបញ្ចេញ ។
		\item គណនាបម្រែបម្រួលមាឌឧស្ម័ន បើប្រព័ន្ធមានសម្ពាធថេរ $6.5atm$ ។ 
	\end{enumerate}
	\item($15$ ពិន្ទុ) ម៉ាស៊ីនកាណូត៍មួយស្រូបកម្ដៅ $2000J$ ក្នុងរយៈពេលមួយវដ្តពីធុងក្តៅ និងបញ្ជួនថាមពលកម្តៅ $1200J$ ទៅកាន់ធុងត្រជាក់ ។
	\begin{enumerate}[a]
		\item គណនាប្រសិទ្ធភាពកម្តៅនៃម៉ាស៊ីន ។
		\item គណនាកម្មន្តដែលម៉ាស៊ីនធ្វើក្នុងមួយវដ្ត ។
		\item គណនាអនុភាពរបស់ម៉ាស៊ីន បើម៉ាស៊ីនដំណើរការបាន $4$ វដ្តក្នុងរយៈពេល $2s$ ។
	\end{enumerate}
	\item 
\end{enumerate}
\begin{center}
	\sffamily\color{black}
	សូមសំណាងល្អ!
\end{center}\newpage

\end{document}