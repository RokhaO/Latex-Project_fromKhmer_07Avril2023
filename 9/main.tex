\documentclass[a4paper, 12pt]{exam}
\makeatletter
\usepackage[top=0.5cm, left=1cm, bottom=1.8cm, right=1.5cm]{geometry}
\usepackage{amsmath,amssymb}
\usepackage{tcolorbox}
\usepackage[export]{adjustbox}
\usepackage{graphicx}
\usepackage{wrapfig}
\usepackage{pgf}
\usepackage{tikz}% graphic drawing
\usetikzlibrary{arrows}
\pagestyle{empty}
\usepackage{wasysym}
\usepackage{mathpazo}% change math font
\usepackage{enumitem}% change list environment like enumerate, itemize and description
\usepackage{multicol}% multi columns
\usepackage{xcolor}
\newcommand{\teacher}{ស៊ំុ សំអុន}
\newcommand{\tell}{០៩៦ ៩៤០ ៥៨៤០}
\newcommand{\class}{ប្រឡង~ឆមាសលើកទី~០១}
\newcommand{\dateofexam}{សម័យប្រឡង៖~កុម្ភៈ ០៧ ២០១៨}
\newcommand{\subject}{វិញ្ញាសា៖~គណិតវិទ្យា~(វិទ្យាសាស្រ្តពិត)}
\newcommand{\timelimit}{១២០~នាទី}
\newcommand{\score}{ពិន្ទុសរុប៖~១២៥~ពិន្ទុ}
\usepackage[no-math]{fontspec}% font specfication
\setmainfont{Khmer OS Content}% set default font to Khmer OS
\setsansfont[Ligatures=TeX,AutoFakeBold=0,AutoFakeSlant=0.25]{Khmer OS Muol Light}% sans serif font
%
\newcommand{\heart}{\ensuremath\heartsuit}
\newcommand{\butt}{\rotatebox[origin=c]{180}{\heart}}
\newcommand*\circled[1]{\tikz[baseline=(char.base)]{
		\node[shape=circle,draw,inner sep=2pt] (char) {#1};}}
%
\SetEnumitemKey{I}{%
	leftmargin=*,
	label={\protect\tikz[baseline=-0.9ex]\protect\node[draw=gray,thick,circle,minimum height=.65cm,inner sep=1pt,text=black,fill=white]{\Roman*};},%
	font=\small\sffamily\bfseries,%
	labelsep=1ex,%
	topsep=0pt}
%
\SetEnumitemKey{a}{%
	leftmargin=*,%
	label={\protect\tikz[baseline=-0.9ex]\protect\node[draw=gray,thick,circle,minimum height=.5cm,inner sep=1pt,text=blue,fill=magenta!5!white]{\alph*};},%
	font=\small\sffamily\bfseries,%
	labelsep=1ex,%
	topsep=0pt}
%
\SetEnumitemKey{1}{leftmargin=*,%
	label={\protect\tikz[baseline=-0.9ex]\protect\node[draw=gray,thick,circle,minimum height=.5cm,inner sep=1pt,text=black,fill=cyan!20!white]{\arabic*};},%
	font=\small\sffamily\bfseries,%
	labelsep=1ex,%
	topsep=0pt}
%
\def\hard{\leavevmode\makebox[0pt][r]{\large\ensuremath{\star}\hspace{2em}}}
%
\def\hhard{\leavevmode\makebox[0pt][r]{\large\ensuremath{\star\star}\hspace{2em}}}
%
\everymath{\protect\displaystyle\protect\color{black}}
%
\pagecolor{cyan!1!white}
%
\usepackage{amsmath}
\usepackage{amssymb}
\usepackage{wasysym}
\makeatother

\pagestyle{foot}
\firstpagefooter{}{ ទំព័រ~\thepage\ នៃ \numpages}{}
\runningfooter{រៀបរៀងដោយ~\teacher}{ ទំព័រ \thepage\ នៃ \numpages}{ទូរស័ព្ទ~\tell}
\runningfootrule

\begin{document}
\noindent
%\sffamily\color{black}
\begin{tabular*}{\textwidth \sffamily\color{black}}{l @{\extracolsep{\fill}} r @{\extracolsep{6pt}} l}
\textbf{\class} & \textbf{មណ្ឌលប្រឡង} & \makebox[2in]{\hrulefill}\\
\textbf{\dateofexam} & \textbf{លេខបន្ទប់} & \makebox[2in]{\hrulefill}\\
\textbf{\subject} & \textbf{លេខតុ} & \makebox[2in]{\hrulefill}\\
\textbf{\score} & \textbf{ឈ្មោះបេក្ខជន} & \makebox[2in]{\hrulefill}\\
\textbf{រយៈពេលសរុប៖ \timelimit} & \textbf{ហេត្ថលេខា} & \makebox[2in]{\hrulefill}
\end{tabular*}\\
\rule[2ex]{\textwidth\color{magenta}}{2pt}
\begin{center}
	\heart វិញ្ញាសាគណិតវិទ្យា នេះមាន \numpages\ ទំព័រ (រួមបញ្ចូលទាំងទំព័រនេះផងដែរ) និង 6 សំណួរ ។\heart
\end{center}
\begin{center}
	\sffamily\color{black}
	បទបញ្ជានៃការប្រឡង
\end{center}
\begin{multicols}{2}
	\begin{enumerate}[1]
		\item ហាមមើលគ្នាក្នុងពេលកំពុងប្រឡង
		\item ហាមប្តូរកិច្ចការរបស់ខ្លួនជាមួយមិត្តភក្តិ 
		\item ហាមខ្ចីរបស់គ្នាប្រើក្នុងពេលកំពុងប្រឡង 
		\item ហាមនិយាយឡូឡាក្នុងពេលកំពុងប្រឡង
		\item ហាមនាំអាវុធ ឬគ្រឿងផ្ទុះចូរបន្ទប់ប្រឡង 
		\item ហាមធ្វើសញ្ញាសម្គាល់អ្វីមួយនៅលើក្រដាស់កិច្ចការ~។ 
	\end{enumerate}
\end{multicols}
\noindent
\rule[2ex]{\textwidth\color{magenta}}{2pt}
\begin{center}
	\sffamily\color{black}
	ប្រធានលំហាត់\\
\end{center}
	\begin{enumerate}[I]
		\item ចូរគណនាលីមីតនៃអនុគមន៍ខាងក្រោម៖
		\begin{multicols}{3}
			\begin{enumerate}[a]
				\item $\lim\limits_{x\to 0}\frac{e^{2019x}-e^x}{x}$
				\item $\lim\limits_{x\to 0}\frac{xe^{2014x}-x+\sin^22x}{x^2}$
				\item $\lim\limits_{x\to\infty}\bigg(\frac{x+1009}{x}\bigg)^\frac{x}{2}$
			\end{enumerate}
		\end{multicols}
		\item គេឲ្យចំនួនកុំផ្លិច $z=\frac{\cos\frac{\pi}{4}+i\sin\frac{\pi}{4}}{\sqrt{3}+i}$ ។
		\begin{enumerate}[1]
			\item សរសេរ $z$ ជាទម្រង់ពីជគណិត និងជាទម្រង់ត្រីកោណមាត្រ។ រួចទាញរកតម្លៃប្រាកដនៃ $\cos\frac{\pi}{12}$ និង $\sin\frac{\pi}{12}$
			\item សរសេរ $\bigg(z+\frac{1}{2}\bigg)^3$~ជាទម្រង់ត្រីកោណមាត្រ ។
		\end{enumerate}
		\item គេមានអនុគមន៍ $f(x)=\frac{-\cos x+7\sin x}{3\cos x+4\sin x}$ ។
		\begin{enumerate}[1]
			\item ចូរកំណត់រកចំនួនពិត $a$ និង $b$ ដែល $f(x)=a+b\bigg(\frac{-3\cos x+4\sin x}{3\cos x+4\sin x}\bigg)$ ។
			\item គណនាអាំងតេក្រាល $I=\int f(x) dx$ ។
		\end{enumerate}
		\item ផលបូកការេមួយ និងរង្វង់មួយមានប្រវែងស្មើនឹង $a$ ។ គណនាផលធៀបរវាងរង្វាស់កាំនៃរង្វង់ និងជ្រុងការេ\\ ដើម្បីឲ្យផ្ទៃក្រឡាសរុបមានតម្លៃតូចបំផុត ។
		\item 
	\end{enumerate}
\begin{center}
	\sffamily\color{black}
	សូមសំណាងល្អ!
\end{center}\newpage

\end{document}
