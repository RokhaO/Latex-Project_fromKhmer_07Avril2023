
\documentclass{beamer}

\mode<presentation>
{
  \usetheme{Madrid}       
  \usecolortheme{default} 
  \usefonttheme{serif}    
  \setbeamertemplate{navigation symbols}{}
  \setbeamertemplate{caption}[numbered]
} 

\usepackage[english]{babel}
\usepackage[utf8x]{inputenc}
\usepackage{chemfig}
\usepackage[version=3]{mhchem}


\usepackage{pgfpages}
\pgfpagesuselayout{resize to}[%
  physical paper width=8in, physical paper height=6in]

\title[Maxwell's Equations and the Wave Equation]{A Derivation of the Wave Equation from Maxwell's Equations}
\author{Curtis Peterson}
\institute{Arizona State University}
\date{\today}

\begin{document}
%%%%%
\begin{frame}
  \titlepage
\end{frame}
%%%%%
\begin{frame}{What the Heck Is the Wave Equation}
The Wave Equation Satisfies the Following PDE
\begin{equation}
\frac{1}{c^2}\frac{\partial^2\psi}{\partial t^2}=\frac{\partial^2\psi}{\partial x^2}
\end{equation}
where psi is given by the general solution obtained via separation of variables. 




Take note that this is the one dimensional form of the wave equation (one, unspecified component of a column matrix). The multidimentional wave equation satisfies the following PDE.
\begin{equation}
\frac{1}{c^2}\frac{\partial^2\mathbf{\psi}}{\partial t^2} = \nabla^2 \circ \mathbf{\psi}
\end{equation}
\end{frame}
%%%%%
\begin{frame}{The Derivation Itself}
By Faraday's Law
\begin{equation}
\nabla \times \textbf{E}=- \frac{\partial \textbf{B}}{\partial t}
\end{equation}
Arbitrarily take the curl of both sides of (3)
\begin{equation}
\nabla \times \nabla \times \textbf{E} =- \nabla \times \frac{\partial \textbf{B}}{\partial t}
\end{equation}
Note the following vector identity
$$\nabla \times \nabla \times \textbf{A} = \nabla \big( \nabla \circ \textbf{A} \big) - \nabla^2 \textbf{A}$$
Pay attention to the left side of (4) and apply this identity
$$\nabla \big( \nabla \circ \textbf{E} \big) - \nabla^2 \textbf{E} =- \nabla \times \frac{\partial \textbf{B}}{\partial t}$$
Now take a look at the right side an note that the derivative operator commutes
\begin{equation}
\nabla \big( \nabla \circ \textbf{E} \big) - \nabla^2 \textbf{E} =- \frac{\partial}{\partial t} \big( \nabla \times \textbf{B} \big)
\end{equation}
\end{frame}
%%%%%
\begin{frame}{The Derivation Itself Part II}
Cool, now take note of the Ampere-Maxwell equation below
\begin{equation}
\nabla \times \textbf{B} = \epsilon \big( \ \textbf{J} + \mu \frac{\partial \textbf{E}}{\partial t} \big)
\end{equation}
Now apply this to (5)
\begin{equation}
\nabla \big( \nabla \circ \textbf{E} \big) - \nabla^2 \textbf{E} = \frac{\partial}{\partial t} \big( \ \textbf{J} + \mu \frac{\partial \textbf{E}}{\partial t} \big) 
\end{equation}
Assume that the current density, J, is equal to zero and apply this to (7)
$$\nabla \big( \nabla \circ \textbf{E} \big) - \nabla^2 \textbf{E} =- \mu \epsilon \frac{\partial^2 \textbf{E}}{\partial t^2}$$
Assume, also, that 
$$\nabla \circ \textbf{E} = 0$$
meaning that the divergence of the electric field has to necessarily be zero. Apply this assumption to (7) as well.
\end{frame}
%%%%%
\begin{frame}{The Derivation Itself Part III}
Well look at that! We have the wave equation!
$$\nabla^2 \textbf{E} = \mu \epsilon \frac{\partial^2 \textbf{E}}{\partial t^2}$$
What is c?
$$\frac{1}{c^2}=\mu \epsilon$$
so
$$c=\sqrt{\frac{1}{\mu \epsilon}}$$
which is the propagation speed of light waves, confirmed by experiment.
\end{frame}
\end{document}

