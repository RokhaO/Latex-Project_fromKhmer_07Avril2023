\documentclass[12pt, a4paper]{article}
%%import package named techno
\usepackage{techno}
\usepackage[export]{adjustbox}
\usepackage{wrapfig}
\usepackage{tkz-tab}
%សរសេរគីមីវិទ្យា
%\usepackage[version=3]{mhchem}
%\usepackage{mathpazo}% change math font
%\usepackage[no-math]{fontspec}% font specfication
\everymath{\protect\displaystyle\protect\color{blue}}
\begin{document}
\maketitle\koc
\begin{enumerate}[m]
	\item ចូរពោលទ្រឹស្តីសុីនេទិចនៃឧស្ម័ន។
	\item ចូរសរសេរសមីការភាពនៃឧស្ម័នបរិសុទ្ធ។
	\item ចូរសរសេររូបមន្តថាមពលសុីនេទិចមធ្យមនៃម៉ូលេគុលឧស្ម័ននីមួយៗ។
	\item ចូរសរសេររូបមន្តថាមពលសុីនេទិចសរុបនៃម៉ូលេគុលឧស្ម័ន។
	\item ចូរសរសេររូបមន្តល្បឿនប្ញសការេនៃការេល្បឿនមធ្យមម៉ូលេគុលឧស្ម័ន។
	\item ប្រសិនបើអ្នកអាចប្រើពោះ និងសាច់ដុំទ្រូងដើម្បីបន្ថយមាឌរបស់ខ្លួនអ្នកបាន $20\%$ តើសម្ពាធដែលអ្នកត្រូវធ្វើនេះស្មើប៉ុន្មាន?
	\item ផង់នីមួយៗមានម៉ាស $m_{0}$ និងផ្លាស់ទីដោយល្បឿន $v$ តាមបណ្តោយអ័ក្ស $\overrightarrow{ox}$។ គេដឹងថាក្នុងផ្ទៃ $1mm^{2}$ និងក្នុង $1s$ មានផង់ចំនួន $10^{15}$ \\ទៅទង្គិចនឹងផ្ទៃនោះ។
	ចូររកសម្ពាធរបស់ផង់លើផ្ទៃប៉ះ។\\
	គេឲ្យ $m_{0}=9.1\times10^{-31}kg$ និង $v=8\times10^{7}m/s$។ គេសន្មត ទង្គិចរវាងផង់ និងផ្ទៃប៉ះជាទង្គិចស្ទក់។
	\item គេបាញ់ផង់ឲ្យផ្លាស់ទីតាមបណ្តោយអ័ក្ស $\overrightarrow{ox}$ ដែលកែងនឹងផ្ទៃរបស់អេក្រង់មួយ។ គេដឹងថា ផង់នីមួយៗមានម៉ាស $m_{0}$ និងល្បឿន $v_{0}$។ គេដឹងថាក្នុង $1.25mm^{2}$ ផ្ទៃរបស់អេក្រង់មានផង់ចំនួន $4\times10^{14}$ ទៅទង្គិចរៀងរាល់វិនាទី។ គេសន្មតថា ទង្គិចនោះជាទង្គិចស្ទក់។\\ គណនាល្បឿនរបស់ផង់ដែលផ្លាស់ទីតាមអ័ក្ស $\overrightarrow{ox}$។\\ បើគេដឹងថា សម្ពាធដែលកើតឡើងដោយសារការទង្គិចរបស់ផង់លើផ្ទៃអេក្រង់គឺ $P=3.64\times10^{-3}N/m^{2},~m_{0}=9.1\times10^{-31}kg$។
	\item ផង់នីមួយមានម៉ាស $m_{0}$ នឹងផ្លាស់ទីដោយល្បឿន $v$ តាមបណ្តោយអ័ក្ស $\overrightarrow{ox}$។ គេដឹងថាក្នុងផ្ទៃ $2mm^{2}$ និងក្នុងមួយវិនាទីមានផង់ចំនួន $2\times10^{15}$ ទៅទង្គិចនឹងផ្ទៃនោះ។ គេឲ្យៈ $m_{0}=9.1\times10^{-31}kg$ និង $v=5\times10^{7}m/s$។ គេសន្មតថា ទង្គិចរវាងផង់ និងផ្ទៃប៉ះជាទង្គិចស្ទក់។
	\begin{enumerate}[k,2]
		\item គណនាកម្លាំងសរុបដែលផង់មានអំពើលើផ្ទៃប៉ះ។
		\item គណនាសម្ពាធសរុបរបស់ផង់លើផ្ទៃប៉ះ។
	\end{enumerate}
	\item ប្រូតុងមួយមានម៉ាស $m_{p}=1.67\times10^{-27}kg$ ផ្លាស់ទីដោយល្បឿន $v$ តាមបណ្តោយអ័ក្ស $\overrightarrow{ox}$ ក្នុងមាឌមួយមានរាងជាគូបដែលទ្រនុងនីមួយៗមានរង្វាស់ $3mm$ ប្រូតុងផ្លាស់ពីផ្ទៃម្ខាងទៀតក្នុង $2ns$។ គេសន្មត់ថា ទង្គិចរវាងប្រូតុង និងផ្ទៃខាងនៃគូបជាទង្គិចស្ទក់។
	\begin{enumerate}[k]
		\item រកល្បឿនដើមប្រូតុង នៅខណៈវាចាប់ផ្តើមចេញពីផ្ទៃខាងនៃគូប។
		\item រកសម្ពាធរបស់ប្រូតុងលើផ្ទៃខាងនៃគូប។
		\item គេដឹងថាក្នុងរយៈពេល $2ns$ មានចំនួនប្រូតុង $2\times10^{6}$ ទៅទង្គិចនឹងផ្ទៃខាងនៃគូប។ រកសម្ពាធសរុបរបស់ប្រូតុងលើផ្ទៃខាងនៃគូប។
	\end{enumerate}
	\item អេឡិចត្រុងមួយមានម៉ាស $m_{e}=9.1\times10^{-31}kg$ ផ្លាស់ទីដោយល្បឿន $v$ តាមបណ្តោយអ័ក្ស $\overrightarrow{ox}$ ក្នុងមាឌមួយមានរាងជាគូបដែលទ្រនុងនីមួយៗមានរង្វាស់ $5mm$ ប្រូតុងផ្លាស់ពីផ្ទៃម្ខាងទៀតក្នុង $25ns$។ គេសន្មត់ថា ទង្គិចរវាងប្រូតុង និងផ្ទៃខាងនៃគូបជាទង្គិចស្ទក់។
	\begin{enumerate}[k]
		\item រកល្បឿនដើមអេឡិចត្រុង នៅខណៈវាចាប់ផ្តើមចេញពីផ្ទៃខាងនៃគូប។
		\item រកសម្ពាធរបស់អេឡិចត្រុងលើផ្ទៃខាងនៃគូប។
		\item គេដឹងថាក្នុងរយៈពេល $25ns$ មានចំនួនអេឡិចត្រុង $2\times10^{10}$ ទៅទង្គិចនឹងផ្ទៃខាងនៃគូប។\\ រកសម្ពាធសរុបរបស់អេឡិចត្រុងមានលើផ្ទៃខាងនៃគូប។
	\end{enumerate}
	\item អេឡិចត្រុងមួយមានម៉ាស $m_{e}=9.1\times10^{-31}kg$ ផ្លាស់ទីដោយល្បឿន $v$ តាមបណ្តោយអ័ក្ស $\overrightarrow{ox}$ ក្នុងមាឌមួយមានរាងជាគូបដែលទ្រនុងនីមួយៗមានរង្វាស់ $2mm$ ប្រូតុងផ្លាស់ពីផ្ទៃម្ខាងទៀតក្នុង $25ns$។ គេសន្មត់ថា ទង្គិចរវាងប្រូតុង និងផ្ទៃខាងនៃគូបជាទង្គិចខ្ទាត។
	\begin{enumerate}[k]
		\item រកល្បឿនដើមអេឡិចត្រុង នៅខណៈវាចាប់ផ្តើមចេញពីផ្ទៃខាងនៃគូប។
	\item រកសម្ពាធរបស់អេឡិចត្រុងលើផ្ទៃខាងនៃគូប។
		\item គេដឹងថាក្នុងរយៈពេល $25ns$ មានចំនួនអេឡិចត្រុង $25\times10^{6}$ ទៅទង្គិចនឹងផ្ទៃខាងនៃគូប។\\ រកសម្ពាធសរុបរបស់អេឡិចត្រុងមានលើផ្ទៃខាងនៃគូប។
	\end{enumerate}
	\item អាតូមអុីដ្រូសែនមួយមានម៉ាស $m$ ផ្លាស់ទីដោយល្បឿន $v=1500km/s$ តាមបណ្តោយអ័ក្ស $\overrightarrow{ox}$ ក្នុងមាឌមួយមានរាងគូបដែលទ្រនុងនីមួយមានរង្វាស់ $3mm$។ អុីដ្រូសែន ផ្លាស់ទីពីផ្ទៃម្ខាងទៅម្ខាងទៀត។ គេសន្មតថាសន្មត់ថា ទង្គិចរវាងអុីដ្រូសែន និងផ្ទៃខាងនៃគូបជាទង្គិចខ្នាត។
	\begin{enumerate}[k]
		\item រករយៈពេលដែលអាតូមអុីដ្រូសែនទៅប៉ះនឹងផ្ទៃម្ខាងទៀតនៃគូប។
		\item គេដឹងថាក្នុងរយៈពេល $2ns$ មានចំនួនអាតូមអុីដ្រូសែន $2\times10^{6}$ ទៅទង្គិចនឹងផ្ទៃខាងនៃគូបហើយផ្ទៃខាងរងនៅសម្ពាធសរុប\\ $27.83\times10^{-2}N/m^{2}$។ រកម៉ាសអាតូមអុីដ្រូសែនមួយ។
	\end{enumerate}
	\item ឧស្ម័នបរិសុទ្ធមួយមានមាឌ $V=100cm^{3}$ ស្ថិតក្រោមសម្ពាធ $2.00\times10^{5}Pa$ នៅសីតុណ្ហភាព $20^\circ C$។ តើឧស្ម័ននោះមានប៉ុន្មានម៉ូល? $\left(R=8.31J/mol\cdot K\right)$
	\item ឧស្ម័នបរិសុទ្ធមួយមាន $n=0.08\times10^{-1}mol$ មានសម្ពាធ $P=5.00\times10^{5}Pa$ នៅសីតុណ្ហភាព $60^\circ C$។ តើឧស្ម័ននោះមានមាឌប៉ុន្មាន?
	\item នៅសីតុណ្ហភាព $293K$ និងសម្ពាធ $5atm$ មេតាន $1kmol$ មានម៉ាស $16.0kg$។ គណនាម៉ាសមាឌនៃមេតានក្នុងលក្ខខណ្ឌខាងលើ។
	\item នៅក្នុងបំពង់បិទជិតដែលមានមាឌ $20mL$ នៅសីតុណ្ហភាពកំណត់មួយយ៉ាងទាបមានតំណក់នីត្រូសែនរាវមានម៉ាស $50mg$។\\ គណនាសម្ពាធនីត្រូសែននៅក្នុងបំពង់នោះ កាលណាបំពង់នោះមានសីតុណ្ហភាព $300K$ ដោយសន្មតថានីត្រូសែននេះជាឧស្ម័នបរិសុទ្ធ។\\ គេឲ្យៈ $R=8.31J/mol\cdot K$។
	\item ធុងមួយមានផ្ទុកអេល្យូម $2.00mol$ នៅសីតុណ្ហភាព $27^\circ C$។ គេសន្មតថាអេល្យូមជាឧស្ម័នបរិសុទ្ធ។
	\begin{enumerate}[k]
		\item គណនាតម្លៃមធ្យមនៃថាមពលសុីនេទិចរបស់ម៉ូលេគុលនីមួយៗ
		\item គណនាថាមពលសុីនេទិចសរុបរបស់ម៉ូលេគុលទាំងអស់។\\
		គេឲ្យៈ $k_{B}=1.38\times10^{-23}J/K,~R=8.31J/mol\cdot K$។
	\end{enumerate}
	\item នៅក្នុងធុងមួយដែលមានមាឌ $2.00mL$ មានឧស្ម័នដែលមានម៉ាស $50mg$ និងសម្ពាធ $100kPa$។\\ ម៉ាសរបស់មូលេគុលនៃឧស្ម័ននីមួយៗគឺ $8.0\times10^{-26}kg$។
	\begin{enumerate}[k]
		\item រកចំនួនម៉ូលេគុលនៃឧស្ម័ននោះ។
		\item រកតម្លៃមធ្យមនៃថាមពលសុីនេទិចរបស់ម៉ូលេគុលនីមួយៗ។\\គេឲ្យៈ $k=1.38\times10^{-23}J/K$
	\end{enumerate}
	\item ចូរគណនាប្ញសការេនៃការេល្បឿនមធ្យមរបស់អាតូមអេល្យូមនៅសីតុណ្ហភាព $20.0^\circ C$។ ម៉ាសម៉ូលអេល្យូមគឺ $4.00\times10^{-3}kg/mol$។ \\គេឲ្យៈ $R=8.31J/mol\cdot K$។
	\item រកប្ញសការេនៃការេល្បឿនមធ្យមរបស់ម៉ូលេគុលអុកសុីសែននៅសីតុណ្ហភាព $200^\circ C$។ \\ម៉ាសម៉ូលអុកសុីសែន $32\times10^{-3}kg/mol$ និង $R=8.31J/mol\cdot K$។
	\item \begin{enumerate}[k]
		\item គណនាម៉ាសម៉ូលេគុលនៃអុីដ្រូសែន។ គេឲ្យម៉ាសម៉ូលគឺ $M=2.00\times10^{-3}kg/mol$ និងចំនួនអាវ៉ូកាដ្រូ $N_{A}=6.02\times10^{23}/mol$។
		\item គណនាតម្លៃប្ញសការេនៃការេល្បឿនមធ្យមរបស់ឧស្ម័នអុីដ្រូសែននៅសីតុណ្ហភាព $100^\circ C$។
		\item គណនាតម្លៃមធ្យមនៃថាមពលសុីនេទិចរបស់ម៉ូលេគុលនៃឧស្ម័នអុីដ្រូសែននីមួយៗនៅសីតុណ្ហភាព $100^\circ C$។ គេឲ្យៈ $k=1.38\times10^{-23}$។
	\end{enumerate}
	\item ដោយប្រើតម្លៃលេខ $1,3,7$ និង $8$ ចូរបង្ហាញថា ប្ញសការេនៃការេល្បឿនមធ្យម $v_{rms}$ ខុសគ្នាពីតម្លៃមធ្យម $v_{av}$ របស់វា។
	\item ចូរកំណត់រកល្បឿន $v_{rms}$ របស់ម៉ូលេគុលឧស្ម័នអុកសុីសែន $\left(O_{2}\right)$ និងអាសូត $\left(N_2\right)$ ក្នុងបន្ទប់មួយដែលមានសីតុណ្ហភាព $20^\circ C$។
	\item \begin{enumerate}[k]
		\item បង្ហាញថាល្បឿន $v_{rsm}$ នៃឧស្ម័នបរិសុទ្ធ អាចសរសេរជាទម្រង់មួយទៀតគឺ $v_{rms}=\sqrt{\frac{3P}{\rho}}$ ដែល $\rho$ ជាដង់សុីតេ ឬហៅថាម៉ាសមាឌ ហើយ $P$ ជាសម្ពាធ។
		\item ល្បឿន $v_{rms}$ របស់ម៉ូលេគុលឧស្ម័នមួយប្រភេទស្មើ $450m/s$។\\ ប្រសិនបើវាស្ថិតនៅសម្ពាធបរិយាកាស តើដងសុីតេរបស់ឧស្ម័ននោះស្មើប៉ុន្មាន?
	\end{enumerate}
	\item កែវបាឡុងមួយចំណុះ $1L$ មានអុកសុីសែនជាឧស្ម័នបរិសុទ្ធដែលមានសីតុណ្ហភាព $27^\circ C$ ក្រោមសម្ពាធ $2atm$។\\
	គណនាម៉ាសអុកសុីសែន។ គេឲ្យៈ $O=16$
	\item គេមានខ្យល់មានមាឌ $1m^3$ នៅសីតុណ្ហភាព $18^\circ C$ ក្នុងសម្ពាធបរិយាកាស $P_{1}=1atm$ ទៅបណ្ណែននៅសីតុណ្ហភាពដដែល តែក្នុងសម្ពាធបរិយាកាស $P_{2}=3.5atm$។ គណនាមាឌស្រេចនៃខ្យល់។
	\item ដបមួយផ្ទុកឧស្ម័នមានសម្ពាធ $P_{0}=1.0atm$ នៅសីតុណ្ហភាព $17^\circ C$។\\
	តើគេត្រូវកម្តៅឪ្យឧស្ម័ននេះដល់សីតុណ្ហភាពប៉ុន្មាន ដើម្បីសម្ពាធកើនឡើងដល់ $1.5atm$?
	\item គេយកបំពង់អុកសុីសែនមានចំណុះ $20L$ ក្រោមសម្ពាធ $P_{1}=200atm$ នៅសីតុណ្ហភាព $20^\circ C$ ទៅដាក់ក្នុងបាឡុង កៅស៊ូស្តើងមួយ។ \\គណនាមាឌបាឡុង បើឧស្ម័នក្នុងបាឡុងមានសម្ពាធ $P_{2}=1atm$ និងសីតុណ្ហភាព $9^\circ C$។
	
\end{enumerate}
\end{document}