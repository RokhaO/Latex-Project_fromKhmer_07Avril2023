\documentclass{officialexam} 
\usepackage{chemfig}
\usepackage{tikz}
\usepackage{circuitikz}
\usepackage{graphicx}
\graphicspath{ {./images/} }
\usepackage[version=4]{mhchem}
\definecolor{ffqqqq}{rgb}{1,0,0}
\definecolor{sqsqsq}{rgb}{0.13,0.13,0.13}
\definecolor{ffffff}{rgb}{1,1,1}
\definecolor{uququq}{rgb}{0.25,0.25,0.25}
\everymath{\color{blue}}
\begin{document}
	\maketitle\\
	\centering{\kml \color{blue}\underline{ប្រធាន:}}
	\begin{enumerate}[I]
		\item (៤ ពិន្ទុ) តើទ្រឹស្តីសុីនេទិចនៃឧស្ម័នសិក្សាអំពីអ្វី?
		\item (៦ ពិន្ទុ) គេផ្ទុកកុងដង់សាទ័រមួយដែលមានកាប៉ាសុីតេ $C=2.0\mu F$ ក្រោមតង់ស្យុង $V=5.0V$។\\ គណនាថាមពលអគ្គិសនីដែលផ្ទុកក្នុងកុងដង់សាទ័រ។
		\item (១០ ពិន្ទុ) គណនាមាឌឧស្ម័នអុកសុីសែន $6.4g$ ដែលផ្ទុកក្នុងធុងក្រោមសម្ពាធ $1.0\times10^{5}Pa$ និងសីតុណ្ហភាព $27^\circ C$ ។ \\គេឱ្យ $R=8.31J/mol\cdot K$ និងម៉ាសម៉ូលេគុលឧស្ម័នអុកសុីសែន $M(O_2)=32g/mol$
		\item (១០ ពិន្ទុ) គេធ្វើកម្មន្ត $20kJ$ លើប្រព័ន្ធឧស្ម័នបិទជិតមួយ។ ក្រោយមកកម្តៅ $4190J$ បានភាយចេញពីប្រព័ន្ធ។ \\គណនាបម្រែបម្រួលថាមពលក្នុងនៃប្រព័ន្ធ។
		\item (១០ ពិន្ទុ) ម៉ាសុីនមួយមានទិន្នផលកម្តៅ $0.40$ គណនា៖
		\begin{enumerate}[k]
			\item កម្មន្តដែលបានធ្វើ ប្រសិនបើវាស្រូបកម្ដៅ $4000J$ ពីធុងក្តៅ។
			\item កម្តៅភាយចេញពីធុងត្រជាក់។
		\end{enumerate} 
		\item (១០ ពិន្ទុ) សូលេណូអុីតមួយមានប្រវែង $l=40.0cm$ និងមានកាំ $R=2.0cm$ ត្រូវបានរុំជា​​ស្ពៀ​​​​ជាប់ៗ​​គ្នាចំនួន $2000$ ស្ពៀ។
		\begin{enumerate}[k]
			\item គណនាអាំងឌុចតង់នៃសូលេណូអុីតនេះ។
			\item គណនាថាមពលម៉ាញេទិច បើមានចរន្តប្រែប្រួល $i$ ឆ្លងកាត់បូប៊ីនមានតម្លៃ $i=2.0A$ ។\\ គេឱ្យ $~\pi^2=10$ និងជំរាបដែនម៉ាញេទិចក្នុងសុញ្ញាកាស $\mu_0=4\pi\times10^{-7}T\cdot m/A$
		\end{enumerate} 
	\end{enumerate}
{\color{white}.}\dotfill\\
{\color{white}.}\dotfill\\
{\color{white}.}\dotfill
\\
{\color{white}.}\dotfill\\
{\color{white}.}\dotfill\\
{\color{white}.}\dotfill
\\
{\color{white}.}\dotfill\\
{\color{white}.}\dotfill\\
{\color{white}.}\dotfill
\\
{\color{white}.}\dotfill\\
{\color{white}.}\dotfill\\
{\color{white}.}\dotfill
\\
{\color{white}.}\dotfill\\
{\color{white}.}\dotfill\\
{\color{white}.}\dotfill
\\
{\color{white}.}\dotfill\\
{\color{white}.}\dotfill
\end{document}