\documentclass{officialexam} 
\begin{document}
	{\maketitle}
	\borderline{ប្រធាន ០១}
	\begin{enumerate}[I]
		\item {\color{khtug}(១៥ ពិន្ទុ)} គណនាលីមីត៖
		\begin{enumerate}[k,4]
			\item $\lim_{x\to1}\frac{x^2-4x+3}{1-x^2}$
			\item $\lim_{x\to0}\frac{-2\sin4x}{\sqrt{4-x}-\sqrt{4+x}}$
			\item $\lim_{x\to3}\frac{x^3-27}{3x^2-9x}$
			\item $\lim_{x\to1}\frac{x^3-x^2+x-1}{1-x^2}$
		\end{enumerate}
		\item {\color{khtug}(១៥ ពិន្ទុ)} គេមានចំនួនកុំផ្លិច $z_1=1+i\sqrt{3}$ និង $z_2=1-i\sqrt{3}$
		\begin{enumerate}[k]
			\item គណនា $z_1+z_2, z_1-z_2, z_1\times z_2$ និង $\frac{z_1}{z_2}$។
			\item ចូរសរសេរ $z_1, z_2$ និង $z_1\times z_2$ ជាចំនួនកុំផ្លិចទម្រង់ត្រីកោណមាត្រ ។ 
			\item បង្ហាញថា $z_1$ ជាប្ញសនៃសមីការ $z^3+8=0$ ។ 
		\end{enumerate}
		\item {\color{khtug}(១៥ ពិន្ទុ)} ក្នុងប្រអប់មួយមានប៊ូល ៥ ដោយក្នុងនោះមានប៊ូលពណ៌ខ្មៅ ៣ ត្រូវបានគេចុះលេខពី ១ ដល់ ៣ និងប៊ូលពណ៌ស ២ ត្រូវបានគេចុះលេខពី ១ ដល់ ២ ។ គេចាប់យកប៊ូល ២ ព្រមគ្នាក្នុងពេលតែមួយដោយចៃដន្យចេញពីក្នុងប្រអប់នោះ ។ គណនាប្រូបាបនៃព្រឹត្តិការណ៍ដូចខាងក្រោម៖ 
		\begin{enumerate}[k]
			\item $A$ $ :"$គេចាប់បានប៊ូលមានពណ៌ដូចគ្នា$"$
			\item $B$ $ :"$គេចាប់បានប៊ូលដែលមានផលបូកលេខស្មើ ៣$"$ 
			\item $C$ $ :"$គេចាប់បានប៊ូលមានពណ៌ខុសគ្នា$"$ 
		\end{enumerate}
		\item \begin{enumerate}[1]
			\item {\color{khtug}(១០ ពិន្ទុ)} គណនាអាំងតេក្រាល៖ $\mathrm{I}=\int_{1}^{2}\left(\frac{x^2}{2}+x-3\right)dx$ ~~និង $\mathrm{J}=\int_{0}^{\frac{\pi}{4}}\left(\sin x+\cos2x\right)dx$ ។
			\item {\color{khtug}(៥ ពិន្ទុ)} គេមានអនុគមន៍ $f(x)=-\frac{4-x}{\left(x-3\right)^2}$ កំណត់ចំពោះគ្រប់ $x\neq3$ បង្ហាញថា $f(x)=\frac{1}{x-3}-\frac{1}{\left(x-3\right)^2}$ ។\\ គណនា $\mathrm{K}=\int_{0}^{2}f(x)dx$ ។ 
		\end{enumerate}
		\item \begin{enumerate}[k]
			\item{\color{khtug}(៥ ពិន្ទុ)} ដោះស្រាយសមីការឌីផេរ៉ង់ស្យែល $(E): y''-3y'+2y=0$ ។
			\item{\color{khtug}(៥ ពិន្ទុ)} រកចម្លើយពិសេសមួយនៃ $(E)$ បើ $y(0)=1$ និង $y'(0)=0$ ។
		\end{enumerate}
		\item \begin{enumerate}[1]
			\item{\color{khtug}(១០ ពិន្ទុ)} ក.គេឲ្យខ្សែកោង $(E): \frac{\left(x-4\right)^2}{25}+\frac{y^2}{9}=1$ ។ បញ្ជាក់ប្រភេទនៃខ្សែកោង $(E)$ ។\\
			ខ. កំណត់កូអរដោនេ ផ្ចិត កំពូល កំណុំ ប្រវែងអ័ក្សធំ និងប្រវែងអ័ក្សតូចនៃ $(E)$ រូចសង់ខ្សែកោង $(E)$ ។
			\item {\color{khtug}(១០ ពិន្ទុ)} នៅក្នុងតម្រុយអរតូណរម៉ាល់មានទិសទៅវិជ្ជមាន $\left(O,\vec{i}, \vec{j}, \vec{k}\right)$ គេមានចំណុចបី $A(1,2,1),B(4,2,4),C(5,3,0)$ ។
			\begin{enumerate}[k]
				\item រកប្រវែង $AB, AC, BC$ រួចធ្វើការសន្និដ្ឋាននៃប្រភេទត្រីកោណ $ABC$ ។
				\item គណនាផលគុណ $\overrightarrow{AB}\times\overrightarrow{AC}$ រួចគណនាផ្ទៃក្រឡានៃត្រីកោណ $ABC$ ។
			\end{enumerate}
		\end{enumerate}
		\item {\color{khtug}(៣៥ ពិន្ទុ) \sffamily ផ្នែក $A$} គេមានអនុគមន៍ $g$ កំណត់លើចន្លោះ $(0, +\infty)$ ដោយ $g(x)=x^2+2\ln x$ ។
		\begin{enumerate}[1]
			\item \begin{enumerate}[k]
				\item បង្ហាញថា $g$ ជាអនុគមន៍កើនដាច់ខាត់លើ $(0, +\infty)$ ។
				\item គណនា $g(1)$ ។
			\end{enumerate}
			\item \begin{enumerate}[k]
				\item ទាញលទ្ធផលពីសំនួរទី $1$ បញ្ជាក់ថា បើ $x\geq1$ នោះ $x^2+2\ln x\geq1$ និងបើ $0<x\leq1$ នោះ $x^2+2\ln x\leq1$ ។ 
				\item កំណត់សញ្ញានៃកន្សោម $x^2-1+2\ln x$ កាលណា $x$ នៅចន្លោះ $(0,+\infty)$ ។
			\end{enumerate}
		\end{enumerate}
		{\color{khtug} \sffamily ផ្នែក $B$} គេមានអនុគមន៍ $f$ កំណត់លើ $(0,+\infty)$ ដោយ $f(x)=\frac{x^2-1-2\ln x}{x}$ មានក្រាប $C$ ។
		\begin{enumerate}[k]
			\item ចូររកលីមីតនៃ $f$ ត្រង់ $0$ និង $+\infty$ ។ ទាញបញ្ជាក់នៃសមីការអាស៊ីមតូតឈរនៃក្រាប $C$ ។
			\item ចូរស្រាយបញ្ញាក់ថាបន្ទាត់ពុះទីមួយ $\Delta : y=x$ នៃអ័ក្សកូអរដោនេជាអាស៊ីមតូតទ្រេតរបស់ក្រាប $C$ ខាង $+\infty$ ។ \\សិក្សាទីតាំងរវាងក្រាប $C$ ធៀបនឹងបន្ទាត់ $\Delta$ ។ រួចរកកូអរដោនេនៃចំណុចប្រសព្វរវាងក្រាប $C$ នឹង បន្ទាត់ $\Delta$ ។
			\item ចូរស្រាយបញ្ជាក់ថាគ្រប់ $x>0$ គេបាន $f'(x)=\frac{x^2-1+2\ln x}{x^2}$ ។
			\item ដោយប្រើលទ្ធផលផ្នែក $A$ សិក្សាសញ្ញានៃ $f'(x)$ និងសង់តារាងអថេរភាពនៃ $f$ លើ $(0, +\infty)$ ។
			\item ចូរគណនា $f\left(\frac{1}{e}\right)$ និង $f(e)$ រួចសង់បន្ទាត់ $\Delta$ និងក្រាប $C$ ក្នុងតម្រុយអរតូណរម៉ាល់ $(O, \vec{i}, \vec{j})$ ។\\
			គេឲ្យ៖ $e=2.7$ និង $ \frac{1}{e}=0.37$ ។
		\end{enumerate}
	\end{enumerate}
\borderline{\bigg[សូមសំណាងល្អគ្រប់ៗគ្នា!\bigg]}\\
{\color{white}.}\dotfill\\
{\color{white}.}\dotfill\\
{\color{white}.}\dotfill\\
{\color{white}.}\dotfill\\
{\color{white}.}\dotfill\\
{\color{white}.}\dotfill\\
{\color{white}.}\dotfill\\
{\color{white}.}\dotfill\\
{\color{white}.}\dotfill\\
{\color{white}.}\dotfill\\
{\color{white}.}\dotfill\\
{\color{white}.}\dotfill\\
{\color{white}.}\dotfill\\
{\color{white}.}\dotfill\\
{\color{white}.}\dotfill\\
{\color{white}.}\dotfill\\
{\color{white}.}\dotfill\\
{\color{white}.}\dotfill
\newpage
{\maketitle}
\borderline{ប្រធាន​ ០២}
\begin{enumerate}[I]
	\item គណនាលីមីត៖ 
	\begin{enumerate}[k,4]
		\item $\lim_{x\to3}\frac{x^3-27}{3x^2-9x}$
		\item $\lim_{x\to0}\frac{-2\sin4x}{\sqrt{4-x}-\sqrt{4+x}}$
		\item $\lim_{x\to1}\frac{x^3-x^2+x-1}{1-x^2}$
		\item $\lim_{x\to\frac{\pi}{3}}\frac{4\cos^2x+4\cos x-3}{2\cos x-1}$
	\end{enumerate}
	\item $Z$ ជាចំនួនកុំផ្លិចដែល $Z=\left(\sqrt{2}-i\sqrt{2}\right)\left(\cos\frac{\pi}{6}-\sin\frac{\pi}{6}\right)$ ។
	\begin{enumerate}[k]
		\item សរសេរ $Z$ ជាទម្រង់ពិជគណិត។
		\item សរសេរ $Z^2$ ជាទម្រង់ត្រីកោណមាត្រ។
		\item គណនា $\cos\frac{5\pi}{12}$ និង $\sin\frac{5\pi}{12}$។
	\end{enumerate}
	\item ក្នុងប្រអប់មួយមានប៊ិចខៀវចំនួន $7$ដើម និងប៊ិចក្រហមចំនួន $5$ដើម។ គេចាប់យកប៊ិច $4$ ដើមចេញពីប្រអប់ដោយចៃដន្យ។
	\begin{enumerate}[k]
		\item រកប្រូបាបដែលចាប់បានប៊ិចខៀវទាំង $4$។
		\item រកប្រូបាបដែលចាប់បានប៊ិចខៀវចំនួន $3$ដើម និងចាប់បានប៊ិចក្រហមចំនួន $1$ដើម។
		\item រកប្រូបាបដែលចាប់បានប៊ិចក្រហមយ៉ាងតិចមួយដើម។
	\end{enumerate}
	\item \begin{enumerate}[k]
		\item ដោះស្រាយសមីការ $2y''-3y'+y=0~~(E)$ ។
		\item រកចម្លើយមួយនៃ $(E)$ ដោយដឹងថាក្រាបនៃចម្លើយនោះប៉ះនឹងបន្ទាត់ $L: y=2x+1$ ត្រង់ចំណុច $A(0;1)$។
	\end{enumerate}
	\item $f$ ជាអនុគមន៍កំណត់ចំពោះគ្រប់ $x\in\mathbb{R}$ ដោយ $f(x)=x+2-\ln\left(1+e^{2x}\right)$ មានក្រាប $C$ ក្នុងតម្រុយ $\left(O, \vec{i}, \vec{j}\right)$ ។
	\begin{enumerate}[1]
		\item \begin{enumerate}[k]
			\item គណនាលីមីតនៃ $\ln\left(1+e^{2x}\right)$ កាលណា $x\to-\infty$ រួចទាញរកលីមីតនៃ $\lim_{x\to-\infty}f(x)$ ។
			\item បង្ហាញថា បន្ទាត់ $d_1$ ដែលមានសមីការ $y=x+2$ ជាសមីការអាស៊ីមតូតនៃក្រាប $C$ ។
		\end{enumerate}
		\item \begin{enumerate}[k]
			\item បង្ហាញថា ចំពោះគ្រប់ $x\in\mathbb{R};~ f(x)=2-x-\ln\left(1+e^{-2x}\right)$ ។ ទាញរកសមីការអាស៊ីមតូតទ្រេត $d_2$ នៃក្រាប $C$ ។
			\item សិក្សាទីតាំងរវាងក្រាប $C$ ធៀបនឹងអាស៊ីមតូត $d_1;~d_2$ ។
		\end{enumerate}
		\item \begin{enumerate}[k]
			\item ចូរបង្ហាញថា $\forall x\in\mathbb{R};~f'(x)=\frac{\left(1+e^x\right)\left(1-e^x\right)}{1+e^{2x}}$ ។
			\item ដោះស្រាយវិសមីការ $1-e^x>0$ ។ រួចសិក្សាទិសដៅអថេរភាពនៃ $f$ ។
		\end{enumerate}
		\item គណនាតម្លៃនៃ $f(-1)$ និង $f(1)$ ។ (យក $\ln2=0.7~;~\ln\left(1+e^2\right)=2.1$)
		\item សង់បន្ទាត់ $d_1;~d_2$ និងក្រាប $C$ ក្នុងតម្រុយតែមួយ ។
	\end{enumerate}
	\item {\color{khtug}\sffamily ផ្នែក $A$} គេមានអនុគមន៍ $g$ កំណត់លើ $(0,+\infty)$ ដោយ $g(x)=x^2+2-2\ln x$ ។
	\begin{enumerate}[k]
		\item គណនាដេរីវេនៃ $g$ រួចសិក្សាអថេរភាពនៃអនុគមន៍ $g$ លើ $(0, +\infty)$​។
		\item សង់តារាងអថេរភាពនៃ $g$ រួចទាញរកសញ្ញាសនៃ $g(x)$ ​​លើ $(0,+\infty)$ ។ (ដោយមិនចំាបាច់គណនាលីមីត)
	\end{enumerate}
	{\color{khtug} \sffamily ផ្នែក $B$} គេមានអនុគមន៍ $f$ កំណត់លើ $(0,+\infty)$ ដោយ $f(x)=x-1+2\frac{\ln x}{x}$ មានក្រាបតាង $C$ ក្នុងតម្រុយ អរតូណរម៉ាល់ $(O, \vec{i}, \vec{j})$។
	\begin{enumerate}[k]
		\item គណនាលីមីតនៃ $f$ ត្រង់ $0$ និង $+\infty$ រួចបកស្រាយតាមក្រាបនូវលទ្ធផលដែលទទួលបាននេះ ។
		\item បង្ហាញថា $\Delta: y=x-1$ ជាសមីការអាស៊ីមតូតទ្រេតនៃក្រាប $C$ ខាង $+\infty$ ។
		\item សិក្សាទីតាំងរវាងក្រាប $C$ ធៀបនឹងអាស៊ីមតូតទ្រេត $\Delta$ ។
		\item បង្ហាញថា $f'(x)$ មានសញ្ញាដូច $g(x)$ រួចទាញរកសញ្ញានៃ $f'(x)$ លើ $(0, +\infty)$ ។
		\item សង់តារាងអថេរភាពនៃ $f$ រួចសង់ក្រាប $C$ និងបន្ទាត់ $\Delta$ ក្នុងតម្រុយតែមួយ ។
	\end{enumerate}
	\item {\color{khtug} \sffamily ផ្នែក $A$} $g$ ជាអនុគមន៍កំណត់លើ $(0,+\infty)$ ដោយ $g(x)=-x+1-2\ln x$ ។
	\begin{enumerate}[k]
		\item បង្ហាញថា $g$ ជាអនុគមន៍ចុះលើ $(0,+\infty)$ ។
		\item គណនា $g(1)$ រួចកំណត់សញ្ញានៃ $g(x)$ លើ $(0, +\infty)$ ។
	\end{enumerate}
	{\color{khtug} \sffamily ផ្នែក $B$} $f$ ជាអនុគមន៍គ្រប់ $x>0$ ដែល $f(x)=\frac{x+\ln x}{x^2}$ ។​
	\begin{enumerate}[k]
		\item គណនាលីមីតនៃ $f$ ត្រង់ $0$ និង $+\infty$ ។
		\item រកសមីការអាស៊ីមតូតឈរ និងអាស៊ីមតូតដេកនៃក្រាប $C$ ។
		\item បង្ហាញថាអនុគមន៍ $f់'(x)$ យកសញ្ញាដូច $g(x)$ ។ កំណត់សញ្ញានៃ $f'(x)$ រួចទាញថា $f$ មានតម្លៃអតិបរមាត្រង់ $x=1$ ។ 
		\item គណនា $f(1)$ ។ សង់តារាងអថេរភាពនៃ $f$​។
		\item សង់ក្រាប $C$ និងអាស៊ីមតូតដែលមាន ក្នុងតម្រុយតែមួយ ។
	\end{enumerate}
	\item {\color{khtug}(៣៥ ពិន្ទុ) \sffamily ផ្នែក $A$} គេមានអនុគមន៍ $g$ កំណត់លើ $(0, +\infty)$ ដោយ $g(x)=-x^2+1-2\ln x$ ។
	\begin{enumerate}[k]
		\item ចូរគណនាដេរីវេ $g'(x)$ រួចទាញថាអនុគមន៍ $g$ ជាអនុគមន៍ចុះជានិច្ចលើចន្លោះ $(0,+\infty)$ ។ 
		\item ចូរគណនាតម្លៃ $g(1)$ ។ ចូរបញ្ជាក់សញ្ញានៃ $g(x)$ លើ $(0, +\infty)$ ។
	\end{enumerate}
	{\color{khtug} \sffamily ផ្នែក $B$} គេឲ្យអនុគមន៍ $f$ កំណត់លើចន្លោះ $(0, +\infty)$ ដោយ $f(x)=\frac{9x^2+6\ln x-1}{2x^3}$ មានក្រាបតាង $C$ ។
	\begin{enumerate}[k]
		\item ចូររកលីមីតនៃ $f$ ត្រង់ $0$ និង $+\infty$ ។ ទាញរកសមីការអាស៊ីមតូតឈរ និងដេកនៃក្រាប $C$ ។
		\item ចូរស្រាយបញ្ជាក់ថាចំពោះគ្រប់ $x>0$ គេបាន $f'(x)=kg(x)$ ដែល $k$ ជាចំនួនពិតត្រូវកំណត់ ។​\\ គូសតារាងអថេរភាពនៃ $f$ ដោយប្រើលទ្ធផលផ្នែក $A$ ។
		\item ចូរគណនា $f\left(\frac{1}{2}\right)$ រួចទាញថា $f(x)=0$ មានប្ញសតែមួយគត់ស្ថិតនៅចន្លោះ $\left[\frac{1}{2},1\right]$
		\item កំណត់សមីការបន្ទាត់ $T$ ប៉ះទៅនឹងក្រាប $C$ ត្រង់អាប់ស៊ីស $1$ ។
		\item គណនា $f(2)$ រួចសង់ក្រាប $C$ និងបន្ទាត់ $T$ ក្នុងតម្រុយអរតូណរម៉ាល់ $(O,\vec{i}, \vec{j})$ ។\\
		គេឲ្យ៖ $\ln2=0.7 $ និង $\ln\frac{1}{2}=-0.7$ ។
	\end{enumerate}
	\item {\color{khtug} \sffamily ផ្នែក $A$} គេមានអនុគមន៍ $g$ កំណត់លើចន្លោះ $(0, +\infty)$ ដោយ $g(x)=x^2+2\ln x$ ។
	\begin{enumerate}[1]
		\item \begin{enumerate}[k]
			\item បង្ហាញថា $g$ ជាអនុគមន៍កើនដាច់ខាត់លើ $(0, +\infty)$ ។
			\item គណនា $g(1)$ ។
		\end{enumerate}
		\item \begin{enumerate}[k]
			\item ទាញលទ្ធផលពីសំនួរទី $1$ បញ្ជាក់ថា បើ $x\geq1$ នោះ $x^2+2\ln x\geq1$ និងបើ $0<x\leq1$ នោះ $x^2+2\ln x\leq1$ ។ 
			\item កំណត់សញ្ញានៃកន្សោម $x^2-1+2\ln x$ កាលណា $x$ នៅចន្លោះ $(0,+\infty)$ ។
		\end{enumerate}
	\end{enumerate}
	{\color{khtug} \sffamily ផ្នែក $B$} គេមានអនុគមន៍ $f$ កំណត់លើ $(0,+\infty)$ ដោយ $f(x)=\frac{x^2-1-2\ln x}{x}$ មានក្រាប $C$ ។
	\begin{enumerate}[k]
		\item ចូររកលីមីតនៃ $f$ ត្រង់ $0$ និង $+\infty$ ។ ទាញបញ្ជាក់នៃសមីការអាស៊ីមតូតឈរនៃក្រាប $C$ ។
		\item ចូរស្រាយបញ្ញាក់ថាបន្ទាត់ពុះទីមួយ $\Delta : y=x$ នៃអ័ក្សកូអរដោនេជាអាស៊ីមតូតទ្រេតរបស់ក្រាប $C$ ខាង $+\infty$ ។ \\សិក្សាទីតាំងរវាងក្រាប $C$ ធៀបនឹងបន្ទាត់ $\Delta$ ។ រួចរកកូអរដោនេនៃចំណុចប្រសព្វរវាងក្រាប $C$ នឹង បន្ទាត់ $\Delta$ ។
		\item ចូរស្រាយបញ្ជាក់ថាគ្រប់ $x>0$ គេបាន $f'(x)=\frac{x^2-1+2\ln x}{x}$ ។
		\item ដោយប្រើលទ្ធផលផ្នែក $A$ សិក្សាសញ្ញានៃ $f'(x)$ និងសង់តារាងអថេរភាពនៃ $f$ លើ $(0, +\infty)$ ។
		\item ចូរគណនា $f\left(\frac{1}{e}\right), f\left(\sqrt{e}\right), f(2)$ និង $f(e)$ រួចសង់បន្ទាត់ $\Delta$ និងក្រាប $C$ ក្នុងតម្រុយអរតូណរម៉ាល់ $(O, \vec{i}, \vec{j})$ ។\\
		គេឲ្យ៖ $e=2.7, \sqrt{e}=1.65, \frac{1}{e}=0.37$ និង $ \frac{1}{\sqrt{e}}=0.61$ ។
	\end{enumerate}
\end{enumerate}
\borderline{\bigg[ចម្លើយ\bigg]}\\
{\color{white}.}\dotfill\\
{\color{white}.}\dotfill\\
{\color{white}.}\dotfill\\
{\color{white}.}\dotfill\\
{\color{white}.}\dotfill\\
{\color{white}.}\dotfill\\
{\color{white}.}\dotfill\\
{\color{white}.}\dotfill\\
{\color{white}.}\dotfill\\
{\color{white}.}\dotfill
\end{document}