\documentclass[a4paper]{report}

%\usepackage[landscape,margin=1in]{geometry}
\usepackage[ top=0.5cm, left=1cm, bottom=1.5cm, right=1.5cm]{geometry}
\usepackage{mathtools}
\usepackage{mathpazo}% change math font
\usepackage{enumitem}% change list environment like enumerate, itemize and description
\usepackage{multicol}% multi columns
\usepackage{tikz}% graphic drawing
\usepackage[no-math]{fontspec}% font specfication
\setmainfont{Kantumruy}% set default font to Khmer OS
\usepackage{tcolorbox}
\usepackage{graphicx}
\usepackage{tikz}
\usepackage{wasysym}

%
\SetEnumitemKey{I}{%
	leftmargin=*,
	label={\protect\tikz[baseline=-0.9ex]\protect\node[draw=gray,thick,circle,minimum height=.65cm,inner sep=1pt,text=black,fill=cyan!20!white]{\Roman*};},%
	font=\small\sffamily\bfseries,%
	labelsep=1ex,%
	topsep=0pt}
%
\SetEnumitemKey{a}{%
	leftmargin=*,%
	label={\protect\tikz[baseline=-0.9ex]\protect\node[draw=gray,thick,circle,minimum height=.5cm,inner sep=1pt,text=blue,fill=magenta!5!white]{\alph*};},%
	font=\small\sffamily\bfseries,%
	labelsep=1ex,%
	topsep=0pt}
%
\SetEnumitemKey{1}{leftmargin=*,%
	label={\protect\tikz[baseline=-0.9ex]\protect\node[draw=gray,thick,circle,minimum height=.5cm,inner sep=1pt,text=black,fill=blue!20!white]{\arabic*};},%
	font=\small\sffamily\bfseries,%
	labelsep=1ex,%
	topsep=0pt}
%
\def\hard{\leavevmode\makebox[0pt][r]{\large\ensuremath{\star}\hspace{2em}}}
%
\def\hhard{\leavevmode\makebox[0pt][r]{\large\ensuremath{\star\star}\hspace{2em}}}
%
\everymath{\protect\displaystyle\protect\color{black}}
%
\pagecolor{cyan!1!white}
%
\newcommand{\heart}{\ensuremath\heartsuit}
\newcommand{\butt}{\rotatebox[origin=c]{180}{\heart}}
\newcommand*\circled[1]{\tikz[baseline=(char.base)]{
		\node[shape=circle,draw,inner sep=2pt] (char) {#1};}}
\setsansfont[Ligatures=TeX,AutoFakeBold=0,AutoFakeSlant=0.25]{Khmer OS Muol Light}% sans serif font
\begin{document}
	\begin{center}
		\sffamily\color{black}
		\circled{០១}\\
		\heart ទ្រឹស្តីអាស៊ីត-បាស និង ប្រតិម្មអាស៊ីត-បាស\heart \\
		រៀបរៀង និងបង្រៀនដោយៈ ស៊ុំ សំអុន\\
		\phone ទូរស័ព្ទៈ ០៩៦ ៩៤០	៥៨៤០\phone
	\end{center}
	\begin{enumerate}[1]
		\item ចូរជ្រើសរើសអាស៊ីតខាងក្រោមនេះដោយដាក់តាមប្រភេទ ម៉ូណូប្រូទិច ឌីប្រូទិច និងទ្រីប្រូូទិច~។
		\begin{multicols}{4}
			\begin{enumerate}[a]
				\item $HCl$
				\item $HClO_4$
				\item $H_2SO_3$
				\item $H_2SO_4$
				\item $CH_3COOH$
				\item $HNO_3$
				\item $H_3PO_2$
				\item $H_3PO_4$
			\end{enumerate}
		\end{multicols}
		\item ចូរចង្អុលបង្ហាញសារធាតុខាងក្រោមនេះ ណាខ្លះជាអាស៊ីត បាស និងមិនមែន៖
		\begin{multicols}{5}
			\begin{enumerate}[a]
				\item $Na_2SO_3$
				\item $HCl$
				\item $NaCl$
				\item $Ca(OH)_2$
				\item $H_2CO_3$
				\item $CH_4$
				\item $CaCO_3$
				\item $NH_4Cl$
				\item $NaOH$
				\item $CH_3COOH$
			\end{enumerate}
		\end{multicols}
		\item ចូរបំពេញ និងថ្លឹងសមីការអាស៊ីត-បាសខាងក្រោម៖
		\begin{enumerate}[a]
			\item $H_2CO_3~+~Sr(OH)_2 \rightarrow~\cdots~+~\cdots$
			\item $HBr~+~Ba(OH)_2~\rightarrow~\cdots~+~\cdots$
			\item $NaBr~+~H_2SO_4~\rightarrow~\cdots~+~\cdots$
		\end{enumerate}
		\item ចូរសរសេរសមីការអ៊ីយ៉ុងកម្មនៃអាស៊ីតនៅក្នុងទឹក៖
		\begin{multicols}{3}
			\begin{enumerate}[a]
				\item $HNO_3$
				\item $HClO_4$
				\item $H_2SO_4$
				\item $HF$
				\item $HCN$
				\item $HCOOH$
			\end{enumerate}
		\end{multicols}
		\item តើអ្វីទៅដែលហៅថាអាស៊ីតខ្លាំង? អាស៊ីតខ្សោយ?
		\item ចូរសរសេរសមីការអ៊ីយ៉ុងកម្មនៃបាសនៅក្នុងទឹក៖
		\begin{multicols}{3}
			\begin{enumerate}[a]
				\item $NaOH$
				\item $Ca(OH)_2$
				\item $Sr(OH)_2$
				\item $NH_3$
				\item $C_6H_5NH_2$
				\item $CH_3COO^-$
			\end{enumerate}
		\end{multicols}
		\item តើអ្វីទៅដែលហៅថាបាសខ្លាំង? បាសខ្សោយ?
		\item ចូរសរសេររូបមន្តបាសឆ្លាស់របស់អាស៊ីតដូចខាងក្រោម៖
		\begin{multicols}{5}
			\begin{enumerate}[a]
				\item $HCl$
				\item $HCO_3^{-}$
				\item $H_2SO_4$
				\item $N_2H_5^{+}$
				\item $CH_3NH_3^{+}$
			\end{enumerate}
		\end{multicols}
		\item ចូរសរសេររូបមន្តអាស៊ីតឆ្លាស់របស់បាសដូចខាងក្រោម៖
		\begin{multicols}{5}
			\begin{enumerate}[a]
				\item $NO_3^{-}$
				\item $OH^{-}$
				\item $C_2H_5NH_2$
				\item $CH_3COOH$
				\item $H_3O^{+}$
			\end{enumerate}
		\end{multicols}
		\item ចូរកំណត់គូអាស៊ីតបាសឆ្លាស់នៅក្នុងសមីការខាងក្រោម៖
		\begin{enumerate}[a]
			\item $HS_{(aq)}^{-}~+~H_2O_{(l)}~\xrightleftharpoons~H_2S_{(aq)}~+~H_3O_{(aq)}^{+}$
			\item $O_{(aq)}^{2-}~+~H_2O_{(l)}~\xrightleftharpoons~2OH_{(aq)}^{-}$
			\item $H_2S_{(aq)}~+~NH_{3(aq)}~\xrightleftharpoons~NH_{4(aq)}^+~+~HS_{(aq)}^-$
			\item $H_2SO_{4(aq)}~+~H_2O_{(l)}~\rightarrow~H_3O_{(aq)}^+~+~HSO_{4(aq)}^-$
		\end{enumerate}
		\newpage
		\item គេដាក់ស័ង្កសី $(Zn)$ ឲ្យមានប្រតិកម្មជាមួយនឹងសូលុយស្យុង $H_2SO_4$ ចំនួន $100ml$ កំហាប់ $0.5M$ ។ ចូរគណនា៖
		\begin{enumerate}[a]
			\item ម៉ាសស័ង្កសីស៊ុលផាតដែលទទួលបាន ។
			\item មាឌអ៊ីដ្រូសែនដែលភាយចេញនៅ $STP$ ។ (ឧស្ម័ន $1mol$ នៅ $STP$ មានមាឌ $22.4L$) 
		\end{enumerate}
		\item សំបកខ្យងមួយផ្សំពី $CaCO_3$ មានប្រតិកម្មជាមួយនឹងសូលុយស្យុង $HCl$ គេទទួលបាន $1.50L$ ឧស្ម័ន $CO_2$ នៅសីតុណ្ហភាព $STP$ ។ ចូរគណនា៖ 
		\begin{enumerate}[a]
			\item បរិមាណ $CaCO_3$ ដែលចូររួមប្រតិកម្ម ។
			\item មាឌសូលុយស្យុុង $HCl$ នៅ $0.25M$ ដែលប្រើក្នុងប្រតិកម្មនេះ ។\\
			($Ca:40~;~O:16~;~C:12~;~H:1$) 
		\end{enumerate}
		\item ដូចម្តេចដែលហៅថាសមាសធាតុអំផូទែ?
		\item គេយកសូលុយស្យុងអាស៊ីតក្លរីឌ្រិច $20.0mL$ ទៅធ្វើប្រតិកម្មបន្សាបជាមួយនឹង $18.5mL$ នៃសូលុយស្យុង $Ba(OH)_2$ កំហាប់ $0.04M$ ។
		\begin{enumerate}[a]
			\item ចូរសរសេរសមីការតាងប្រតិកម្មបន្សាបនេះ ។
			\item រកកំហាប់ម៉ូឡារីតេនៃសូលុយស្យុង $HCl$ ដែលត្រូវប្រើ ។
			\item គណនាម៉ាសអំបិលដែលបានបង្កើតឡើង ។ ($Cl:35.5~;~ Ba:137$)
		\end{enumerate}
		\item ចូរប្រៀបធៀបពីលក្ខណៈរបស់អាស៊ីតទៅ នឹងលក្ខណៈរបស់បាស។
		\item ឲ្យនិយមន័យបាសតាម អារ៉េញ៉ុស និងតាមប្រុងស្ទែត-ឡូរី ។
		\item ឲ្យនិយមន័យអាស៊ីតតាម អារ៉េញ៉ុស និងតាមប្រុងស្ទែត-ឡូរី និងតាមឡឺវីស ។
		\item ចូររាប់ និងប្រាប់ឈ្មោះ អាស៊ីតខ្លាំង និងបាសខ្លាំងមួយប្រភេទៗឲ្យបានប្រាំ?
		\item ចូរញែកលក្ខណៈសម្គាល់រវាងម៉ូណូប្រូទិចអាស៊ីត និងប៉ូលីប្រូទិចអាស៊ីត ។
		\item ឲ្យឧទាហរណ៍ ម៉ូណូប្រូទិចអាស៊ីត ឌីប្រូទិចអាស៊ីត និងប៉ូលីប្រូទិចអាស៊ីតមួយប្រភេទៗឲ្យបាន $2$ ។
			\begin{center}
				\sffamily\color{black}
				សូមសំណាងល្អ!
			\end{center}\newpage
		\begin{center}
			\sffamily\color{black}
			\circled{០២}\\
			\heart ទ្រឹស្តីអាស៊ីត-បាស និង ប្រតិម្មអាស៊ីត-បាស\heart \\
			រៀបរៀង និងបង្រៀនដោយៈ ស៊ុំ សំអុន\\
			\phone ទូរស័ព្ទៈ ០៩៦ ៩៤០	៥៨៤០\phone
		\end{center}
		\item ក្លរអាចត្រូវបានទង្វើ ដោយប្រតិកម្ម $HCl$ ជាមួយនឹង $MnO_2$ ។ ប្រតិកម្មត្រូវបានបង្ហាញដោយសមីការតុល្យការៈ $MnO_{2(g)}~+~4HCl_{(aq)}~\rightarrow~Cl_{2(g)}~+~MgCl_{2(g)}~+~2H_2O_{2(l)}$~ សន្មតថា ប្រតិកម្មប្រព្រឹត្តិទៅសព្វ ។
		\begin{enumerate}[a]
			\item តើម៉ាសនៃសូលុយស្យុង $HCl$ ខាប់($36.0\%~ HCl$ ជាម៉ាស) ស្មើប៉ុន្មានដែលត្រូវការចាំបាច់ដើម្បីផលិត $2.50g$ នៃ $Cl_2$ ?
			\item គណនាមាឌសូលុយស្យុង $HCl$ នៅ $0.5M$ ដែលប្រើក្នុងប្រតិកម្មនេះ ។
			($H=1~;~Cl=35.5$)
		\end{enumerate}
		\item សូលុយស្យុងមួយមាន $5.0\%$ នៃអាស៊ីតអេតាណូអ៊ិច ($HC_2H_2O_2$) ជាម៉ាស និងដង់ស៊ីតេរបស់វាស្មើនឹង $0.96g/moL$ ។
		\begin{enumerate}[a]
			\item សរសេរសមីការអ៊ីយ៉ុងកម្មនៃអាស៊ីតនេះក្នុងទឹក ។
			\item តើកំហាប់ជាម៉ូលនៃអាស៊ីតអេតាណូអ៊ិចនៅក្នុងសូលុយស្យុងស្មើប៉ុន្មាន?
		\end{enumerate}
		\item ភាគសំណាក $0.35g$ នៃអាស៊ីត $HX$ មួយត្រូវការ $25.4mL$ នៃ $NaOH_{(aq)}$ កំហាប់ $0.14molL^{-1}$ សម្រាប់ធ្វើប្រតិកម្មសព្វ។ ចូរគណនាម៉ាសម៉ូលនៃអាស៊ីត $HX$ នេះ ។
		\item នៅពេលភាគសំណាក $1.25g$ នៃថ្មកំបោរត្រូវរំលាយទៅក្នុងអាស៊ីត $0.44g$ នៃ $CO_2$ ត្រូវបានបង្កើតឡើង ។ ប្រសិនបើ ដុំថ្មបានផ្ទុកគ្មានកាបូណាតផ្សេងទៀតក្រៅពី $CaCO_3$ ។\\
		តើភាគរយជាម៉ាសនៃ $CaCO_3$ នៅក្នុងថ្មកំបោរស្មើប៉ុន្មាន?
		\item បរិមាណ $500cm^3$ នៃសូលុយស្យុងមួយមាន $H_2SO_4$ រលាយចូរ $0.20mol$ ។\\
		គណនាកំហាប់ជាម៉ូលនៃសូលុយស្យុង $H_2SO_4$ គិតជា $mol/dm^3$ ។
		\item សិស្សម្នាក់បានយកសូលុស្យុងអាស៊ីតក្លរីឌ្រិច $100mL$ នៅកំហាប់ $2.0mol\cdot L^{-1}$ ទៅធ្វើប្រតិកម្មជាមួយនឹង\\ សូដ្យូមប៊ីកាបូណាត ($NaHCO_3$) នៅក្នុងបន្ទប់ពិសោធន៍ ។
		\begin{enumerate}[a]
			\item ចូរសរសេរសមីការតុល្យការតាងប្រតិកម្ម ។
			\item គណនាម៉ាស $NaHCO_3$ ដែលបានប្រើ ។\\
			គេឲ្យៈ ($H=1~;~C=12~;~O=16~;~Na=23$)
		\end{enumerate}
		\item នៅក្នុងបន្ទប់ពិសោធន៍គេយកបន្ទះទង់ដែង $12.8g$ ទៅរំលាយក្នុងសូលុយស្យុងអាស៊ីតនីទ្រិច ($HNO_3$)\\ ខាប់ដែលមានបរិមាណលើស ។
		សមីការប្រតិកម្មៈ $Cu{(s)}~+~4HNO_{3(aq)}~\rightarrow~Cu(NO_3)_{(aq)}~+~2NO_{2(g)}~+~2H_2O_{(l)}$
		\begin{enumerate}[a]
			\item គណនាមាឌឧស្ម័នដែលភាួចេញនៅ $STP$ ។
			\item រកម៉ាសទង់ដែង $II$ នីត្រាតដែលទទួលបាន ។
			\item ម៉ាសទង់ដែង $II$ នីត្រាតដែលទទួលបានតាមពិសោធន៍ស្មើនឹង $22.8g$ ។ គណនាទិន្នផលនៃប្រតិកម្មនេះ ។ ($Vm=22.4L/mol$)
		\end{enumerate}
		\item គេដាក់ស័ង្កសីឲ្យមានប្រតិកម្មជាមួយសូលុយស្យុង $H_2SO_4$ ចំនួន $100mL$ នៅកំហាប់ $6.00M$ ។ គណនា
		\begin{enumerate}[a]
			\item ម៉ាសស័ង្កសីស៊ុលផាតដែលទទួលបាន ។
			\item មាឌអ៊ីដ្រូសែនដែលភាយចេញនៅ $STP$ ។ (ឧស្ម័ន $1mol$ នៅ $STP$ មានមាឌ $22.4L$) 
		\end{enumerate}
		\item ប្រភេទខាងក្រោមនេះអាចចាត់ទុកជាអាស៊ីតផង និងបាសផងក្នុងគូពីរផ្សេងគ្នាៈ $HSO^-_4~;~HS^-~;~HCO^-_3~;~H_2O~;~HSO^-_3$ និង $NH_3$ ។
		\begin{enumerate}[a]
			\item តើគេអាចទុកប្រភេទទាំងពីនេះយ៉ាងដូចម្តេច?
			\item ចូរសរសេរគួទាំងពីររបស់ប្រភេទនីមួយៗ ។
			\item ចូរបញ្ជាក់ គួណាខ្លះជាអាស៊ីតខ្លាំង និងណាខ្លះជាបាសខ្លាំង ។
		\end{enumerate}
		\item ដូចម្តេចដែលហៅថាប្រតិកម្មបន្សាប?
		\item តើគូអាស៊ីត បាសឆ្លាស់របស់ទឹកមានប៉ុន្មាន?
		\begin{enumerate}[a]
			\item តើគូណាមួយដែលទឹកមាននាទីជាអាស៊ីត?
			\item តើគូមួយណាដែលទឹកមាននាទីជាបាស?
		\end{enumerate}
		\item ចូរចាត់ថ្នាក់ប្រភេទគីមីដូចខាងក្រោមនេះថាជាអាស៊ីត បាស ប្រុងស្ទែត ដើរទួនាទីទាំងពីរ៖
		\begin{multicols}{2}
			\begin{enumerate}[a]
				\item $H_2O$
				\item $OH^-$
				\item $H_3O^+$
				\item $NH_3$
				\item $NH_4^+$
				\item $NH^-_2$
				\item $NO^-_3$
				\item $CO^{2-}_3$
				\item $HBr$
				\item $HCN$ ។
			\end{enumerate}
		\end{multicols}
			\begin{center}
				\sffamily\color{black}
				សូមសំណាងល្អ!
			\end{center}\newpage
			\begin{center}
				\sffamily\color{black}
				\circled{០៣}\\
				\heart ជំរើសលំហាត់ សម្រាប់ត្រៀមប្រឡងឆមាសលើកទី ១\heart \\
				រៀបរៀង និងបង្រៀនដោយៈ ស៊ុំ សំអុន\\
				\phone ទូរស័ព្ទៈ ០៩៦ ៩៤០	៥៨៤០\phone
			\end{center}
		\item ដូចម្តេចដែលហៅថាសមាសធាតុអំផូលីត? ចូរឧទាហរណ៍បញ្ជាក់។
		\item 
		\begin{enumerate}[a]
			\item តើមាឌសូលុស្យុងរាវនៃប្រាក់នីត្រាត ដែលមានកំហាប់ $0.1M$ ត្រូវមានប៉ុន្មាន $cm^3$ បើគេចាក់វាទៅក្នុង $20cm^3$ នៃសូលុយស្យុងរាវ សូដ្យូមក្លរួដែលមានកំហាប់ $23.4g.L^{-1}$ ?ដើម្បីធ្វើឲ្យបាត់អ៊ីយ៉ុងក្លុរួ ($Cl^{-}$) អស់ ។
			\item គណនាម៉ាសកករដែលកើតឡើង ?\\
			គេឲ្យ៖ $Ag=108~;~Na=23~;~Cl=35.5$
		\end{enumerate}
		\item គេបន្តក់សូលុយស្យុងអាស៊ីតក្លរីឌ្រិច ទៅលើថ្មកំបោរ($CaCO_3$)គេទទួលបានឧស្ម័ន $44.8mL$ នៅលក្ខខណ្ឌធម្មតា ($S.T.P$) ។
		\begin{enumerate}[a]
			\item ចូរកំណត់ឈោះ និងរូបមន្តនៃឧស្ម័នដែលទទួលបាន ?
			\item ចូសរសេរសមីការគីមី អ៊ីយ៉ុងសព្វ និងអ៊ីយ៉ុងសម្រួល សម្រាប់ប្រតិកម្មខាងលើនេះ ?
			\item គណនាម៉ាសថ្មកំបោរចូរប្រតិកម្ម?\\
			គេឲ្យ៖ $C=12~;~O=16~;~Ca=40$
		\end{enumerate}
		\item គេចាក់បារ្យ៉ុមក្លរួ នៅកំហាប់ $0.2M~;~20mL$ ទៅក្នុងសូលុយស្យុងសូដ្យូមកាបូណាត នៅកំហាប់ $C_M$ មិនស្គាល់ និងមាឌ $40mL$ ។
		\begin{enumerate}[a]
			\item សរសេរសមីការ គីមី អ៊ីយ៉ុងសព្វ និងអ៊ីយ៉ុងសម្រួួលនៃប្រតិកម្មខាងលើ ។\\ តើអ៊ីយ៉ុងណាខ្លះដែលគ្មានការប្រែប្រួលក្នុងពេលប្រតិកម្ម ?
			\item គណនា $C_M$ កំហាប់ម៉ូលែនៃ $Na_2CO_3$ ?
			\item គណនាកំហាប់ជាម៉ូលនៃអ៊ីយ៉ុង $Na^+$ និង $Cl^-$ ដែលមានក្នុងសូលុយស្យុងក្រោយប្រតិកម្មចប់?
		\end{enumerate}
		\item គេបង់កំទេចដែក $16.8g$ ទៅក្នុងសូលុយស្យុង $AgNO_3$ នៅកំហាប់ $1M$ គេទទួលបានសូលុយស្យុង $A$ និងអង្គធាត់រឹង $B$ ។
		\begin{enumerate}[a]
			\item ឲ្យសមីការតាងប្រតិកម្មដែលកើតមាន ។
			\item កំណត់មាឌនៃសូលុយស្យុង $AgNO_3$ ដែលយកមកប្រើ ?
			\item គណនាម៉ាសអង្គធាតុរឹង $B$ ?
			គេឲ្យ៖ $Fe=56~;~Ag=108$
		\end{enumerate}
		\item គេលាយ $50cm^3$ នៃសូលុយស្យុង $NaOH$ កំហាប់ $C_B=1.4mol.L^{-1}$ និង $50cm^3$ នៃសូលុយស្យុងអាស៊ីត $HCl$ កំហាប់ $C_A=1mol.L^{-1}$ ។
		\begin{enumerate}[a]
			\item តើប្រតិកម្មអ្វីកើតឡើង? ចូរឲ្យសមីការតុល្យការ។
			\item តើសូលុយស្យុងដែលទទួលបានក្រោយប្រតិកម្មស្ថិតឈដ្ខក្នុងមជ្ានអ្វី?\\
			គណនា $pH$ សូលុយស្យុងដែលទទួលបាននេះ?
		\end{enumerate}
			\item \hard -ក្នុងកែវបេស៊ែរមួយមានសូលុយស្យុងអាស៊ីតក្លរីឌ្រិច($H_3O^+, Cl^-$) នៅកំហាប់ $C_A=1\times10^{-2}M$ និងមាឌ $V_A=20mL$ ។\\
			\hhard -ក្នុងប៊ុយរ៉ែតក្រិតមួយមានសូលុយស្យុង $NaOH$ កំហាប់ $C_B=1\times10^{-2}M$ និងមាឌ $V_B$ ។\\
			គេបានធ្វើការសំរក់សូលុយស្យុង $NaOH$ ខាងលើនេះទៅក្នុងកែវបេស៊ែរនោះ ។
			\begin{enumerate}[a]
				\item សរសេរសមីការតុល្យការតាងប្រតិកម្មដែលកើតមាន?
				\item គណនា $pH$ សូលុយស្យុងអាស៊ីត $HCl$ មុនពេលសំរក់សូលុយស្យុង $NaOH$ ចូរ?
				\item គណនា $pH$ នៃសូលុយស្យុងដែលទទួលបានក្រោយពេលសំរក់សូលុយស្យុង $NaOH~~10mL$ ។ 
			\end{enumerate}
			\item គេរំលាយឧស្ម័នអ៊ីដ្រូសែនក្លរួ ($HCl$) $1.12L$ ក្នុងទឹកសុទ្ធ$1L$ ។
			\begin{enumerate}[a]
				\item សរសេរសមីការអ៊ីយ៉ុងកម្មនៃ $HCl$ ក្នុងទឹក ។
				\item គណនា $C_A$ កំហាប់ជាម៉ូលនៃសូលុយស្យុងអាស៊ីត $HCl$ ដែលទទួលបាន ?
				\item គេយកសូលុយស្យុងអាស៊ីត $HCl$ នេះ $10mL$ ចាក់ទៅក្នុងសូលុយស្យុង $KOH$ កំហាប់ $C_B=2\times10^{-2}M~;\\~V_B=25mL$ ។
				\begin{enumerate}[1]
					\item ឲ្យសមីការតុល្យការតាងប្រតិកម្មដែលកើតមានឡើង។
					\item តើសូលុយស្យុងដែលទទួលបានជា អាស៊ីត, បាស ឬណឺត?\\
					កំណតតម្លៃ $pH$ សូលុយស្យុងដែលទទួលបាន
				\end{enumerate}
			\end{enumerate}
			\item គេមានសូលុយស្យុង$HNO_3$ មួយនៅកំហាប់ $C_A=5\times10^{-2}M$ មាឌ $V_A=25cm^3$ ។\\
			តើគេត្រូវប្រើសូលុយស្យុង $KOH$ នៅកំហាប់ $C_B=2\times10^{-2}M$ ប៉ុន្មាន $cm^3$ ដើម្បីបន្សាបអាស៊ីត $HNO_3$ ខាងលើនេះ\\ឲ្យសាប់អស់? 
			\item សូ.អាស៊ីតក្លរីឌ្រិច ($HCl$) មួយមានកំហាប់ $C_A=5\times10^{-3}M$ ។\\
			គណនា $pH$ នៃសូ.នេះ? គេឲ្យ៖ $log5=0.7$\\
			\heart ចម្លើយ៖ $pH=2.3$
			\item 
	\end{enumerate}
	\begin{center}
		To be continued
	\end{center}
	\begin{center}
		\sffamily\color{black}
		សូមសំណាងល្អ!
	\end{center}\newpage
\end{document}