\documentclass[12pt,a4paper]{book}
\usepackage{lib/bookline}
\usepackage{amssymb}
\begin{document}
	\frontmatter
	\tableofcontents
	\clearpage
	\chapter{អារម្ភកថា}
	សរសេរអារម្ភកថាទីនេះ
	\chapter{អំណរគុណ}
	សរសេរអំណរគុណទីនេះ
	\mainmatter
	\chapter{សិក្សាអនុគមន៍​ និងខ្សែកោង}
	\section{អនុគមន៍សនិទាន}
	\subsection{គន្លឹះសិក្សាអនុគមន៍}
	\begin{itemize}
		\item \sffamily{ដែនកំណត់}
		\begin{itemize}
			\item \koc អនុគមន៍សនិទាន $y=\frac{f(x)}{g(x)}$ មានន័យកាលណា $g(x)\neq0$ ។ ដូច្នេះ $\mathbb{D}=\mathbb{R}-\left\lbrace g(x)=0\right\rbrace$
		\end{itemize}
	\end{itemize}
\begin{thm}
	ចំណុច
	
\end{thm}
\begin{proof}
	សរសេរទីនេះ
\end{proof}
\section*{លំហាត់}
\begin{enumerate}
	\item គណនាដេរីវេទី $ n $ នៃអនុគមន៍ខាងក្រោម៖
	\begin{tasks}[counter-format=tsk[k].](4)
		\task $ f(x)=e^x $
		\task $ f(x)=x e^x $
		\task $ f(x)=x^2 e^x $
		\task $ f(x)=x^3 e^x $
		\task $ f(x)=\cos x $
		\task $ f(x)=\sin x $
		\task $ f(x)=x\cos x $
		\task $ f(x)=x\sin x $
	\end{tasks}
\end{enumerate}
\appendix
\chapter{ចំណងជើងជំពូក}
\section{ចំណងជើងផ្នែក}
\subsection{ចំណងជើងផ្នែករង}
\backmatter
\begin{thebibliography}{3}
	
\end{thebibliography}
\end{document}