\documentclass[11pt,a4paper]{book}
\usepackage{lib/pinkbook}
\begin{document}
	\chapter{អនុគមន៏អ៊ិចស្ប៉ូណង់ស្យែល}
	\section{ស្វ័យគុណ}
	\begin{definition}{}{}
		សរសេរអត្ថបទនៅទីនេះ។
	\end{definition}
	\section{អនុគមន៏អ៊ិចស្ប៉ូណង់ស្យែលគោល $ a $}
	\begin{definition}{}{}
		សរសេរអត្ថបទនៅទីនេះ។
	\end{definition}
	\begin{property}{}{}
		សរសេរអត្ថបទនៅទីនេះ។
	\end{property}
	\section{ចំនួន $ e $}
	\begin{definition}{}{}
		សរសេរអត្ថបទនៅទីនេះ។
	\end{definition}
	\section{អនុគមន៏អ៊ិចស្ប៉ូណង់ស្យែលគោល $ e $}
	\begin{definition}{}{}
		សរសេរអត្ថបទនៅទីនេះ។
	\end{definition}
	\begin{property}{}{}
		សរសេរអត្ថបទនៅទីនេះ។
	\end{property}
	\section{លីមីតនៃអនុគមន៏អ៊ិចស្ប៉ូណង់ស្យែល}
	\begin{theorem}{}{}
		សរសេរអត្ថបទនៅទីនេះ។
	\end{theorem}
	\begin{property}{}{}
		សរសេរអត្ថបទនៅទីនេះ។
	\end{property}
	\section{ដេរីវេនៃអនុគមន៏អ៊ិចស្ប៉ូណង់ស្យែល}
	\begin{property}{}{}
		សរសេរអត្ថបទនៅទីនេះ។
	\end{property}
	\section{លំហាត់}
	\renewcommand{\baselinestretch}{}
	\begin{enumerate}
		\item សម្រួលកន្សោមខាងក្រោម៖
		\begin{enumerate}
			\begin{multicols}{2}
				\item លំហាត់
				\item លំហាត់
				\item លំហាត់
				\item លំហាត់
				\item លំហាត់
				\item លំហាត់
				\item លំហាត់
				\item លំហាត់
			\end{multicols}
		\end{enumerate}
		\item គណនាលីមីតនៃអនុគមន៏អ៊ិចស្ប៉ូណង់ស្យែលខាងក្រោម៖
		\begin{enumerate}
			\begin{multicols}{2}
				\item លំហាត់
				\item លំហាត់
				\item លំហាត់
				\item លំហាត់
				\item លំហាត់
				\item លំហាត់
				\item លំហាត់
				\item លំហាត់
			\end{multicols}
		\end{enumerate}
	\end{enumerate}
	%
	\section{ស្វ័យគុណ}
	\begin{definition}{}{}
		សរសេរអត្ថបទនៅទីនេះ។
	\end{definition}
	\section{អនុគមន៏អ៊ិចស្ប៉ូណង់ស្យែលគោល $ a $}
	\begin{definition}{}{}
		សរសេរអត្ថបទនៅទីនេះ។
	\end{definition}
	\begin{property}{}{}
		សរសេរអត្ថបទនៅទីនេះ។
	\end{property}
	\section{ចំនួន $ e $}
	\begin{definition}{}{}
		សរសេរអត្ថបទនៅទីនេះ។
	\end{definition}
	\section{អនុគមន៏អ៊ិចស្ប៉ូណង់ស្យែលគោល $ e $}
	\begin{definition}{}{}
		សរសេរអត្ថបទនៅទីនេះ។
	\end{definition}
	\begin{property}{}{}
		សរសេរអត្ថបទនៅទីនេះ។
	\end{property}
	\section{លីមីតនៃអនុគមន៏អ៊ិចស្ប៉ូណង់ស្យែល}
	\begin{theorem}{}{}
		សរសេរអត្ថបទនៅទីនេះ។
	\end{theorem}
	\begin{property}{}{}
		សរសេរអត្ថបទនៅទីនេះ។
	\end{property}
	\section{ដេរីវេនៃអនុគមន៏អ៊ិចស្ប៉ូណង់ស្យែល}
	\begin{property}{}{}
		សរសេរអត្ថបទនៅទីនេះ។
	\end{property}
	\section{លំហាត់}
	\renewcommand{\baselinestretch}{}
	\begin{enumerate}
		\item សម្រួលកន្សោមខាងក្រោម៖
		\begin{enumerate}
			\begin{multicols}{2}
				\item លំហាត់
				\item លំហាត់
				\item លំហាត់
				\item លំហាត់
				\item លំហាត់
				\item លំហាត់
				\item លំហាត់
				\item លំហាត់
			\end{multicols}
		\end{enumerate}
		\item គណនាលីមីតនៃអនុគមន៏អ៊ិចស្ប៉ូណង់ស្យែលខាងក្រោម៖
		\begin{enumerate}
			\begin{multicols}{2}
				\item លំហាត់
				\item លំហាត់
				\item លំហាត់
				\item លំហាត់
				\item លំហាត់
				\item លំហាត់
				\item លំហាត់
				\item លំហាត់
			\end{multicols}
		\end{enumerate}
	\end{enumerate}
\end{document}