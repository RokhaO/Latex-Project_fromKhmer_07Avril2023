\documentclass[12pt,xetex,serif]{beamer}
\mode<presentation>{}
\usepackage{beamerthemeBMlibrary}
\mode<presentation>
\usetheme{Madrid}
\usepackage{multicol}
\begin{document}
	\begin{frame}
		\titlepage
	\end{frame}
%	\begin{frame}
%		\contentsname
%		\tableofcontents
%	\end{frame}
	\begin{frame}{សេចក្តីផ្ដើម}
		\pause
		\centering \kml \huge \color{blue}សូមស្វាគមន៍!\\[.5ex]
		\color{magenta}{\tcfamily\scalebox{4}{6}}
	\end{frame}
	\section{អាំងតេក្រាលមិនកំណត់}
    \begin{frame}{លំហាតប្រឡងចូររៀនថ្នាក់វិស្វករ}
    	\pause
        \begin{example}
            \kb គេឲ្យ $E$ ជាសំណុំប្ញសទាំងអស់នៃសមីការ $x^2+5x+6=0$ ។ 
            \begin{multicols}{3}
            	\begin{enumerate}[a]
            		\item $E=\lbrace-2\rbrace$
            		\item $E=\lbrace-3\rbrace$
            		\item $E=\lbrace3,2\rbrace$
            		\item $E=\lbrace3,-2\rbrace$
            		\item $E=\lbrace-3,-2\rbrace$
            	\end{enumerate}
            \end{multicols}
        \end{example}
        \pause
        \centering \kml{\answername}\kb
        \pause
        \begin{center}
        	តាម  $Vieta's~Theorem$ គេមាន $X^2-SX+P=0$ ដែល $\alpha$ និង $\beta$ ជាប្ញសនៃសមីការនេះ គេបាន $\alpha+\beta=S$ និង $\alpha\cdot\beta=P$ \\
        	ដើម្បី ឲ្យបានសមីការមានទម្រង់ $x^2+5x+6=0$ លុះត្រាតែ ផលបូកប្ញស $\alpha+\beta=-5$ និង $\alpha\cdot\beta=6$\\
        	$\therefore \quad$ \kml ចម្លើយ \kbk ង
        \end{center}
        \pause
        \kml សម្គាល់ \kb យើងអាចដោះស្រាយតាមវីធីផ្សេងទៀតក៏បាន តែខ្លះអាចនឹងចំណាយពេលច្រើន
    \end{frame}
    \begin{frame}{លំហាតប្រឡងចូររៀនថ្នាក់វិស្វករ}
    	\pause
    	\begin{example}
    		\kb សំណុំ $I$ នៃប្ញសទាំងអស់របស់វិសមីការ $2^{2x}-4\ge0$ គឺ
    		\begin{multicols}{3}
    			\begin{enumerate}[a]
    				\item $I=\left(-\infty;1\right)$
    				\item $I=\left[1;+\infty\right)$
    				\item $I=\left(1;\infty\right)$
    				\item $I=\left(-\infty;1\right]$
    				\item ចម្លើយផ្សេង
    			\end{enumerate}
    		\end{multicols}
    	\end{example}
    	\pause
    	\centering \kml{\answername}\kb
    	\pause
    		\begin{center}
    			គេមាន $2^{2x}-4\ge0$ នោះ
    			\begin{align*}
    			2^{2x} \ge 2^{2}\\
    			\Leftrightarrow 2x \ge 2\\
    			\Rightarrow x\ge1
    			\end{align*}
    			$\therefore \quad$ \kml ចម្លើយ \kbk ខ
    		\end{center}
    \end{frame}
    \section{អាំងតេក្រាលកំណត់}
    \begin{frame}{អាំងតេក្រាលកំណត់}
        \begin{example}
        	\kb
            \begin{enumerate}[a]
            	\item ការសរសេរជាភាសាខ្មែរ
            	\item ការសរសេរជាភាសាខ្មែរ
            	$\int_{0}^{3}\frac{2x}{x^2+1}dx$
            \end{enumerate}
        \end{example}
    \end{frame}
    \begin{frame}{\bibname}
    	
    \end{frame}
\end{document}