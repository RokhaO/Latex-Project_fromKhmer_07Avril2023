\documentclass{officialexam}
  \begin{document}
  \maketitle
  \begin{center}
  	\kml{\underline{ប្រធានទី០១}}
  \end{center}
  \begin{enumerate}[I]
  	\item ក្នុងថតតុងមួយមានសៀវភៅគណិតវិទ្យា $7$ ក្បាល និងសៀវភៅភាសាខ្មែរ $5$ ក្បាល។ សិស្យម្នាក់បានយកសៀវភៅ $4$ ក្បាលព្រមគ្នា ចេញពីថតតុដោយចែដន្យ។
  	\begin{enumerate}[k]
  		\item រកប្រូបាបដែល "សិស្សយកបានសៀវភៅគណិតវិទ្យាទាំង $4$ ក្បាល"។
  		\item រកប្រូបាបដែល "សិស្សយកបានសៀវភៅភាសាខ្មែរ $1$  ក្បាល យ៉ាងតិច"។
  	\end{enumerate}
  	\item អេលីប $E$ មួយមានសមីការទូទៅ: $9x^2+4y^2+18x-24y+9=0$។
  	\begin{enumerate}[k]
  		\item រកសមីការស្តង់ដានៃអេលីប $E$។
  		\item រកប្រវែងអក្សធំ និងអក្សតូច ហើយរកកូអរដោនេនៃ ផ្ចិត កំពូល និងកំណុំនៃអេលីប $E$។
  	\end{enumerate}
  	\item អនុគមន៍ $g$ កំណត់ចំពោះ $x\ne -1$ ដោយ $g(x)=\frac{4x-1}{\left(x+1\right)^2}$។
  	\begin{enumerate}[k]
  		\item រកចំនួនពិត $a$ និង $b$ ដើម្បីឲ្យ $g(x)=\frac{a}{x+1}+\frac{b}{\left(x+1\right)^2}$ ចំពោះគ្រប់ $x\ne-1$។
  		\item ដោះស្រាយសមីការឌីផេរ៉ង់ស្យែល $\left(x+1\right)^2y'=4x-1$ ចំពោះ $x\ne -1$ ដោយដឹងថា $y(0)=2019$។
  	\end{enumerate}
  	\item គេមានអនុគមន៍ $f$ កំណត់ដោយ $f(x)=\frac{x^2+x+4}{x+1}$ ហើយមានក្រាប $C$។
  	\begin{enumerate}[k]
  		\item រកដែនកំណត់ និងសិក្សាសញ្ញាដេរីវេ​ $f'(x)$ នៃអនុគមន៍។
  		\item សរសេរសមីការអាសុីមតូតឈរ និងអាសុីមតូតទ្រេតនៃក្រាប $C$។
  		\item សង់តារាងអថេរភាព អាសុីមតូត និងក្រាប $C$ នៃអនុគមន៍ $f$។
  	\end{enumerate}
  \end{enumerate}
  \newpage
  \maketitle
  \begin{center}
  	\kml{\underline{ប្រធានទី០២}}
  \end{center}
  \begin{enumerate}[I]
  	\item \begin{enumerate}[k]
  		\item ដោះស្រាយសមីការឌីផេរ៉ង់ស្យែល $(E): y'-3y=0$។
  		\item រកចម្លើយ $y_p$ មួយនៃសមីការ $(E)$ ដើម្បីឲ្យក្រាបនៃចម្លើយកាត់តាមចំណុច $M\left(0,1\right)$។
  	\end{enumerate}
  	\item អេលីប $E$ មួយមានសមីការ $x^2+4y-2x+16y+13=0$។
  	\begin{enumerate}[k]
  		\item រកកូអរដោនេនៃផ្ចិត កំពូល និងកំណុំនៃអេលីប $E$។
  		\item រកកូអរដោនេនៃចំណុចប្រសព្វរវាងអេលីប $E$ និងអ័ក្សអរដោនេ $y'Oy$។
  	\end{enumerate}
  	\item គេឲ្យ $A(x)$=$\frac{x+1}{\left(x-1\right)^2}$ ចំពោះ $x\ne 1$។
  	\begin{enumerate}[k]
  		\item រកចំនួនពិត $a$ និង $b$ ដើម្បីឲ្យ $A(x)=\frac{a}{x-1}+\frac{b}{\left(x-1\right)^2}$ ចំពោះគ្រប់ $x\ne1$។
  		\item គណនា $I(x)=\int A(x)dx$។
  	\end{enumerate}
  	\item អនុគមន៍ $f$ កំណត់ចំពោះគ្រប់ $x$ ដោយ $y=f(x)=x+e^x$ ហើយមានក្រាប $C$ ។
  	\begin{enumerate}[k]
  		\item គណនា $\lim_{x \to +\infty } f(x)$ និង $\lim_{x \to -\infty } f(x)$។
  		\item បង្ហាញថាបន្ទាត់ $d:y=x$ ជាអាសុីមតូតទ្រេតនៃក្រាប $C$ កាលណា $x\to -\infty$ ។
  		\item គណនាដេរីវេ $f'(x)$ ហើយសង់តារាងអថេរភាពនៃ $f$។
  		\item គណនា $f(-1),~f(0),~f(1)$ ហើយសង់ក្រាប $C$ ក្នុងតម្រុយកូអរដោនេមួយ។ គេយក $e=2.7, e^{-1}=0.4$។
  	\end{enumerate}
  \end{enumerate}
  \newpage
  \maketitle
  \begin{center}
  	\kml{\underline{ប្រធានទី០៣}}
  \end{center}
  \begin{enumerate}[I]
  	\item គណនាលីមីតនៃអនុគមន៍ខាងក្រោម៖
  	\begin{enumerate}[k,3]
  		\item $\lim_{x\to3}\frac{x^4+6x+1}{x^2+1}$
  		\item $\lim_{x\to+\infty}\frac{x-1}{\left(x+1\right)^2}$
  		\item $\lim_{x\to+\infty}\left(x^2+2-\ln x\right)$
  	\end{enumerate}
  	\item ក្នុងថ្នាក់រៀនមួយមានសិស្ស $15$ នាក់ ក្នុងនោះសិស្យប្រុស $9$ នាក់ និងសិស្សស្រី $6$ នាក់ ។ \\ 
  	គេជ្រើសរើសសិស្ស $3$ នាក់ ដោយចៃដន្យជាតំណាងទៅសម្ភាសន៍ ។ គណនាប្រូបាបនៃព្រឹត្តិការណ៍ខាងក្រោម៖
  	\begin{enumerate}[A]
  		\item ក្រុមសិស្សទាំង $3$ នាក់ សុទ្ធតែជាសិស្សស្រី ។
  		\item ក្រុមសិស្សទាំង $3$ នាក់ សុទ្ធតែជាសិស្សប្រុស ។
  		\item ក្រុមសិស្សទាំង $3$ នាក់ មាន $2$ នាក់ជាសិស្សស្រី ។
  	\end{enumerate}
  	\item គណនាអាំងតេក្រាលខាងក្រោម៖ 
  	\begin{enumerate}[k,3]
  		\item $\mathrm{I}=\int_{1}^{2}\left(3x^2-2x+3\right)dx$
  		\item $\mathrm{J}=\int_{0}^{1}\left(e^{2x}-e^x+1\right)dx$
  		\item $\mathrm{K}=\int_{1}^{2}\left(\frac{1}{x+3}+\frac{1}{x^2}\right)dx$
  	\end{enumerate}
  	\item គេមានប៉ារ៉ាបូលមួយដែលមានកំពូលជាចំណុច $O\left(0,0\right)$ និងកំណុំ $F$ ស្ថិតនៅលើអ័ក្សអាប់ស៊ីស ។
  	\begin{enumerate}[k]
  		\item រកសមីការស្តង់ដានៃប៉ារ៉ាបូលនេះ បើគេដឹងថាវាកាត់តាមចំណុច $A\left(\frac{3}{2};-3\right)$ ។
  		\item រកកូអរដោនេរបស់កំណុំ សមីការបន្ទាត់ប្រាប់ទិស រួចសង់ប៉ារ៉ាបូលនេះ ។
  	\end{enumerate}
  	\item គេមានអនុគមន៍ $f$ កំណត់ដោយ $f(x)=\frac{2x^2-7x+5}{x^2-5x+7}$ ។ យើងតាងដោយក្រាប $C$ របស់វាលើតម្រុយអរតូណរម៉ាល់ $\left(O, \vec{i}, \vec{j}\right)$ ។
  	\begin{enumerate}[1]
  		\item រកដែនកំណត់ $\mathbb{D}$ នៃអនុគមន៍ $f$ ។
  		\item សិក្សាលីមីតនៃអនុគមន៍ $f(x)$ ត្រង់ $-\infty$ និងត្រង់ $+\infty$ ។ ទាញរកសមីការអាស៊ីមតូត $d$ ទៅនឹងក្រាប $C$ ត្រង់ $-\infty$ និង $+\infty$ ។
  		\item \begin{enumerate}[k]
  			\item ស្រាយបំភ្លឺថាគ្រប់ចំនួនពិត $x\in\mathbb{D}~,$ ដេរីវេ $f'(x)=\frac{-3\left(x^2-6x+8\right)}{\left(x^2-5x+7\right)}$ ។
  			\item សិក្សាអថេរភាពនៃអនុគមន៍ $f$ និងសង់តារាអថេរភាពនៃអនុគមន៍ $f$ ។
  			\item សង់ក្រាប $C$ នៃអនុគមន៍ $f$ ។
  		\end{enumerate}
  	\end{enumerate}
  \end{enumerate}
\newpage
 {\maketitle}
  \begin{center}
  	\kml{\underline{ប្រធានទី០៤}}
  \end{center}
  
  \begin{enumerate}[I]
  \item គណនាលីមីត៖
\begin{enumerate}[k,3]
\item $\lim_{x\to +\infty}\frac{x^2+x+1}{x^2+1}$
\item $\lim_{x\to 3}\frac{x^3-27}{\sqrt{x+6}-3}$
\item $\lim_{x\to 0}\frac{e^x+e^{-x}}{2}$
\end{enumerate}
\item ក្នុងថង់មួយមានប៊ូលពណ៌សចំនួន៣ និងប៊ូលពណ៌ក្រហមចំនួន៦។ គេចាប់យកប៊ូល៣ ក្នុងពេលតែមួយចេញពីថង់ដោយចៃដន្យ។ រកប្រូបាបនៃព្រឹត្តិការណ៍ខាងក្រោម៖
\begin{enumerate}[A]
\item : "ប៊ូលទាំងបីមានពណ៌ស"
\item : "ប៊ូលទាំងបីមានពណ៌ក្រហម"
\item : "មានប៊ូលមួយពណ៌ក្រហម និងពីរទៀតពណ៌ស"
\end{enumerate}
\item គណនាអាំងតេក្រាលខាងក្រោម៖
\begin{enumerate}[k,3]
		\item $\mathrm{I}=\int_1^3\left(3x^2+2x+1\right)dx$
		\item $\mathrm{J}=\int_0^1\left(2e^x-1\right)dx$
		\item $\mathrm{K}=\int_1^2\left(x+\frac{1}{x^2}\right)dx$
		\end{enumerate}
\item គេមានសមីការ $9x^2+25y^2=225$ ។
		\begin{enumerate}[k]
		\item បង្ហាញថាសមីការនេះជាសមីការអេលីប។ រកប្រវែងអ័ក្សតូច ប្រវែងអ័ក្សធំ និងកូអរដោនេនៃកំពូលទាំងពីរ។
		\item សង់អេលីបនេះ។
		\end{enumerate}
\item គេមានអនុគមន៍ $f$ កំណត់លើ $\mathbb{R}-\{2\}$ ដោយ $f(x)=\frac{x^2-x-1}{x-2}$ ។ យើងតាង $C$ ជាក្រាបរបស់វា លើតម្រុយអរតូណរម៉ាល់ $\left(0,\overrightarrow{i},\overrightarrow{j}\right)$ ។
\begin{enumerate}[1]
\item សិក្សាលីមីតនៃអនុគមន៍ $f$ ត្រង់ $-\infty$ និងត្រង់ $+\infty$ ។
\item សិក្សាអថេរភាព និងសង់តារាងអថេរភាពនៃអនុគមន៍ $f$ ។
\item \begin{enumerate}[a]
\item រកចំនួនពិត $a,b,c$ ដែលគ្រប់ $x\neq 2;\ \ \ f(x)=ax+b+\frac{c}{x-2}$ ។
\item គេតាង $\mathrm{d}$ ដែលមានសមីការ $y=x+1$។ បង្ហាញថា $d$ ជាអាស៊ីមតូតនៃ $C$ ត្រង់ $+\infty$ និង $-\infty$។ 
\\ សិក្សាទីតាំងនៃក្រាប $C$ ធៀបនឹងបន្ទាត់ $\mathrm{d}$ ។
\item សង់ក្រាប $C$ និង បន្ទាត់ $d$ ។
\end{enumerate}
\end{enumerate}
\end{enumerate}
\newpage
 {\maketitle}
\begin{center}
	\kml{\underline{ប្រធានទី០៥}}
\end{center}
\begin{enumerate}[I]
	\item {\color{khtug}(\kml{១០ ពិន្ទុ})} គណនាលីមីតនៃអនុគមន៍ខាងក្រោមៈ
	\begin{enumerate}[k,3]
		\item $\lim_{x\to +\infty}\frac{8x^2-x+1}{x^2+1}$
		\item $\lim_{x\to 0}\frac{\sqrt{x^2+4}-2}{x}$
		\item $\lim_{x \to -\infty }\frac{e^{x}-1}{e^{2x}+1}$
	\end{enumerate}
	\item {\color{khtug}(\kml{១០ ពិន្ទុ})} ក្នុងថង់មួយមានប៊ូលពណ៌សចំនួន $3$ និងប៊ូលពណ៌ក្រហមចំនួន $6$។ គេចាប់យកប៊ូល $3$ ក្នុងពេលតែមួយព្រមគ្នាចេញពីថង់ដោយចៃដន្យ។ រកប្រូបាបនៃព្រឹត្តិការណ៍ខាងក្រោមៈ\\
	{\color{khtug}ក.} $A:$ ប៊ូលទាំងបីមានពណ៌ស\quad {\color{khtug}ខ.} $B:$ ប៊ូលមួយពណ៌ក្រហម និងពីរទៀតពណ៌ស\quad
	{\color{khtug}គ.} $C:$ ប៊ូលពណ៌ក្រហមយ៉ាងតិចមួយ
	\item {\color{khtug}(\kml{១៥ ពិន្ទុ})} គណនាអាំងតេក្រាលៈ
	$I=\int_{1}^{2}\left(x^{2}-3x+2\right)dx\quad;\quad J=\int_{1}^{2}\left(\frac{x-1}{x+1}\right)dx\quad;\quad K=\int_{1}^{2}\left(\frac{2x^3-x^2-x}{x^2}\right)dx$។
	\item {\color{khtug}(\kml{១០ ពិន្ទុ})} គេមានប៉ារ៉ាបូលមួយមានកំពូលនៅត្រង់ចំណុច $O\left(0,0\right)$ និងកំណុំ $F$ ស្ថិតនៅលើអ័ក្សអរដោនេ។
	\begin{enumerate}[k]
		\item រកសមីការស្តង់ដារនៃប៉ារ៉ាបូល បើគេដឹងថាវាកាត់តាមចំណុច $A\left(2,6\right)$។
		\item រកតម្លៃនៃ $x$ បើ $B\left(x,\frac{3}{2}\right)$ ស្ថិតនៅលើប៉ារ៉ាបូលនេះ។ ចូរសង់ប៉ារ៉ាបូលនេះ។
	\end{enumerate}
	\item {\color{khtug}(\kml{៣០ ពិន្ទុ})} អនុគមន៍ $f$ កំណត់ដោយ $f(x)=\frac{-3x^2+18x-27}{x^2-7x+10}$ តាង $\left(C\right)$ ជាក្រាបរបស់វាលើតម្រុយអរតូណរម៉ាល់ $\left(o,\vec{i},\vec{j}\right)$។
	\begin{enumerate}[k]
		\item គណនា $\lim_{x\to 2}f(x),\lim_{x\to 5}f(x)$ និង $\lim_{x \to \pm\infty }f(x)$។ ទាញរកសមីការអាសុីមតូតឈរ និងដេកនៃក្រាប $\left(C\right)$។
		\item ចូរស្រាយបញ្ជាក់ថាគ្រប់ $x\ne2$ និង $x\ne5$ គេបាន $f'(x)=\frac{3\left(x^2-2x-3\right)}{\left(x^2-7x+10\right)^{2}}$ ។
		\item សិក្សាអថេរភាព និងសង់តារាងអថេរភាពនៃអនុគមន៍ $f$។
		\item គណនាកូអរដោនេនៃចំនុណ $A$ ជាចំណុចប្រសព្វរវាងអាសុីមតូតដេក និងក្រាប $\left(C\right)$។ រួចរកសមីការនៃបន្ទាត់ $\left(T\right)$ ដែលប៉ះនឹង $\left(C\right)$ ត្រង់ចំណុច $A$ នេះ។
		\item សង់ក្រាប $\left(C\right)$ អាសុីមតូត និងបន្ទាត់ប៉ះ $\left(T\right)$ ក្នុងតម្រុយអរតូណរម៉ាល់ $\left(o,\vec{i},\vec{j}\right)$។
	\end{enumerate}
		\begin{center}
		\sffamily\color{blue}
		សូមអានប្រធានលំហាត់ឲ្យបានច្បាស់មុនធ្វើលំហាត់!
	\end{center}
\end{enumerate}
\borderline{ចម្លើយ}\\
{\color{white}.}\dotfill
\\
{\color{white}.}\dotfill\\
{\color{white}.}\dotfill\\
{\color{white}.}\dotfill
\\
{\color{white}.}\dotfill\\
{\color{white}.}\dotfill\\
{\color{white}.}\dotfill
\\
{\color{white}.}\dotfill\\
{\color{white}.}\dotfill\\
{\color{white}.}\dotfill
\\
{\color{white}.}\dotfill\\
{\color{white}.}\dotfill\\
{\color{white}.}\dotfill
\\
{\color{white}.}\dotfill\\
{\color{white}.}\dotfill\\
{\color{white}.}\dotfill
\\
{\color{white}.}\dotfill\\
{\color{white}.}\dotfill\\
{\color{white}.}\dotfill
\\
{\color{white}.}\dotfill\\
{\color{white}.}\dotfill
\begin{center}
	\sffamily\color{blue}
	សូមសំណាងល្អ!
\end{center}\newpage
{\maketitle}

\begin{enumerate}[m]
	\item {\color{khtug}(\kml{១០ ពិន្ទុ})} គណនាលីមីតនៃអនុគមន៍ខាងក្រោមៈ
	\begin{enumerate}[k,3]
		\item $\lim_{x\to 1}\frac{x^4+x^3+x^2+x-1}{x^2+1}$
		\item $\lim_{x\to 3}\frac{3-\sqrt{x+6}}{x^2-9}$
		\item $\lim_{x \to +\infty }\left(x^2+3-\frac{\ln x}{x^2}\right)$
	\end{enumerate}
	\item {\color{khtug}(\kml{១០ ពិន្ទុ})} ក្នុងថង់មួយមានប៊ូល $12$ ក្នុងនោះមានប៊ូលសចំនួន $5$ និងប៊ូលក្រហមចំនួន $7$។ គេចាប់យកប៊ូល $4$ ក្នុងពេលតែមួយព្រមគ្នាដោយចៃដន្យ។ គណនាប្រូបាបនៃព្រឹត្តិការណ៍ៈ
	\begin{enumerate}[k]
		\item $A:$ ចាប់បានប៊ូលសទាំងអស់
		\item $B:$ ចាប់បានប៊ូលក្រហម $3$
		\item $C:$ ចាប់បានប៊ូលក្រហមយ៉ាងតិច $1$ ។
	\end{enumerate}
	\item {\color{khtug}(\kml{១៥ ពិន្ទុ})} {\color{khtug}\kml{១.}} គណនាអាំងតេក្រាលៈ $I=\int_{1}^{2}\left(\frac{x^2}{2}+\frac{x}{3}+\frac{1}{4}\right)dx$\quad និង $J=\int_{1}^{e}\left(2+\frac{4}{x}\right)dx$\\
	 {\color{khtug}\kml{២.}} គេមានអនុគមន៍ $f$ កំណត់ដោយ $f(x)=\frac{x^2-3x-4}{x^2-4}$ ចំពោះគ្រប់ $x\ne2$ និង $x\ne-2$។ \\ចូរបង្ហាញថាចំពោះគ្រប់ $x\ne2$ និង $x\ne-2$ គេបាន $f(x)=1-\frac{3x}{x^2-4}$ រួចទាញរក $K=\int_{1}^{3}f(x)dx$។
	\item {\color{khtug}(\kml{១០ ពិន្ទុ})}​គេមានសមីការ $\left(E\right): y^2=36-4x^2$។
	\begin{enumerate}[k]
		\item ចូរបង្ហាញថាសមីការនេះជាសមីការអេលីប។ រួចកំណត់ ប្រវែងអ័ក្សតូច អ័ក្សធំ និងកូអរដោនេ កំពូល កំណុំទាំងពីររបស់វា។
		\item សង់អេលីប $(E)$។
	\end{enumerate}
	\item {\color{khtug}(\kml{៣០ ពិន្ទុ})} គេមានអនុគមន៍ $f$ កំណត់ចំពោះគ្រប់ $x\ne-1$ និង $x\ne1$ ដែល $f(x)=\frac{x^2-9}{4\left(x^2-1\right)}$ មានក្រាបតាង $\left(C\right)$។
	\begin{enumerate}[k]
		\item គណនា $\lim_{x\to -1}f(x),\lim_{x\to 1}f(x)$ និង $\lim_{x \to \pm\infty }f(x)$។ ទាញរកសមីការអាសុីមតូតឈរ និងដេកនៃក្រាប $\left(C\right)$។
		\item ចូរស្រាយបញ្ជាក់ថាគ្រប់ $x\ne-1$ និង $x\ne1$ គេបាន $f'(x)=\frac{4x}{\left(x^2-1\right)^{2}}$ ។
		\item សិក្សាអថេរភាព និងសង់តារាងអថេរភាពនៃអនុគមន៍ $f$។
		\item សង់ក្រាប $\left(C\right)$ និងអាសុីមតូត  $\left(T\right)$ ក្នុងតម្រុយអរតូណរម៉ាល់ $\left(o,\vec{i},\vec{j}\right)$។
	\end{enumerate}
\end{enumerate}
\begin{center}
\sffamily\color{blue}
សូមអានប្រធានលំហាត់ឲ្យបានច្បាស់មុនធ្វើលំហាត់!
\end{center}
\newpage
{\maketitle}
\begin{enumerate}[m]
	\item {\color{khtug}(\kml{១០ ពិន្ទុ})} គណនាលីមីតៈ
	\begin{enumerate}[k,3]
		\item $\lim_{x\to 3}\frac{\sqrt{x+6}-3}{x^2-9}$
		\item $\lim_{x\to +\infty}\left(2+\ln x-x^2\right)$
		\item $\lim_{x\to +\infty}\ln\left(\frac{2x-1}{x+1}\right)$
	\end{enumerate}
	\item {\color{khtug}(\kml{១០ ពិន្ទុ})} ក្នុងកាបូបក្មេងប្រុសម្នាក់មានឃ្លីខៀវចំនួន $4$ ឃ្លីបៃតងចំនួន​ $2$ និងឃ្លីខៀវចំនួន $4$។ ក្មេងប្រុសម្នាក់នោះបានលូកចាប់យកឃ្លី $3$ ចេញពីកាបូបព្រមគ្នាក្នុងពេលតែមួយដោយចៃដន្យ។ គណនាប្រូបាបនៃព្រឹត្តិការណ៍ដែលក្មេងប្រុសនោះលូកចាប់បានៈ
	\begin{enumerate}[k,3]
		\item $A:$ ឃ្លីទាំង $3$ មានពណ៌ខៀវ។
		\item $B:$ ឃ្លីទាំង $3$ មានពណ៌ខុសៗគ្នា។
		\item $C:$ ឃ្លីទាំង $3$ មានពណ៌ក្រហមពីរ។
	\end{enumerate}
	\item {\color{khtug}(\kml{១៥ ពិន្ទុ})} {\color{khtug}\kml{១.}} គណនាអាំងតេក្រាលៈ $I=\int_{1}^{2}\left(1+2x-3x^2\right)dx$ និង $J=\int_{0}^{1}\left(e^{3x}-e^{x}+3\right)dx$\\
	{\color{khtug}\kml{២.}} ដោះស្រាយសមីការឌីផេរ៉ងស្យែល $\left(E\right):y'=2y$។ រួចកំណត់ចម្លើយពិសេសមួយនៃ $\left(E\right)$ បើ $y(0)=2019$។
	\item {\color{khtug}(\kml{១០ ពិន្ទុ})} គេមានសមីការ $\left(E\right):\left(3x+2y\right)^2$=$12\left(xy+3\right)$។
	\begin{enumerate}[k]
		\item បង្ហាញថាសមីការខាងលើនេះជាសមីការអេលីប​។ កំណត់កូអរដោនេនៃកំពូល និងកំណុំទាំងពីររបស់វា។
		\item សង់អេលីបនេះ។
	\end{enumerate}
	\item {\color{khtug}(\kml{៣០ ពិន្ទុ})} អនុគមន៍ $f$ កំណត់ដោយ $f(x)=\frac{x^2-2x+4}{x^2-2x+2}$ តាង $\left(C\right)$ ជាក្រាបរបស់វាលើតម្រុយអរតូណរម៉ាល់ $\left(o,\vec{i},\vec{j}\right)$។
	\begin{enumerate}[k]
		\item រកដែនកំណត់ $\mathbb{D}$ នៃអនុគមន៍ $f$។
		\item សិក្សាលីមីតនៃ $f$ ត្រង់ $-\infty$ និង $+\infty$។ ទាញរកសមីការអាសុីមតូតនៃក្រាប $\left(C\right)$។
		\item ចូរស្រាយបញ្ជាក់ថា $f'(x)=\frac{-4\left(x-1\right)}{\left(x^2-2x+2\right)^{2}}$ ចំពោះគ្រប់ $x\in\mathbb{D}$។
		\item សិក្សាអថេរភាព និងសង់តារាងអថេរភាពនៃអនុគមន៍ $f$។
		\item គណនា $f(0)$ និង $f(2)$។ រួចសង់ក្រាប $\left(C\right)$ និងអាសុីមតូត ក្នុងតម្រុយអរតូណរម៉ាល់ $\left(o,\vec{i},\vec{j}\right)$។
	\end{enumerate}
\end{enumerate}

\end{document}
