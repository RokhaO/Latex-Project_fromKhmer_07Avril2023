\documentclass{officialexam}
  \begin{document}
  \maketitle
  \begin{center}
  	\kml{\underline{ប្រធានទី០១}}
  \end{center}
  \begin{enumerate}[I]
  	\item ក្នុងថតតុងមួយមានសៀវភៅគណិតវិទ្យា $7$ ក្បាល និងសៀវភៅភាសាខ្មែរ $5$ ក្បាល។ សិស្យម្នាក់បានយកសៀវភៅ $4$ ក្បាលព្រមគ្នា ចេញពីថតតុដោយចែដន្យ។
  	\begin{enumerate}[k]
  		\item រកប្រូបាបដែល "សិស្សយកបានសៀវភៅគណិតវិទ្យាទាំង $4$ ក្បាល"។
  		\item រកប្រូបាបដែល "សិស្សយកបានសៀវភៅភាសាខ្មែរ $1$  ក្បាល យ៉ាងតិច"។
  	\end{enumerate}
  	\item អេលីប $E$ មួយមានសមីការទូទៅ: $9x^2+4y^2+18x-24y+9=0$។
  	\begin{enumerate}[k]
  		\item រកសមីការស្តង់ដានៃអេលីប $E$។
  		\item រកប្រវែងអក្សធំ និងអក្សតូច ហើយរកកូអរដោនេនៃ ផ្ចិត កំពូល និងកំណុំនៃអេលីប $E$។
  	\end{enumerate}
  	\item អនុគមន៍ $g$ កំណត់ចំពោះ $x\ne -1$ ដោយ $g(x)=\frac{4x-1}{\left(x+1\right)^2}$។
  	\begin{enumerate}[k]
  		\item រកចំនួនពិត $a$ និង $b$ ដើម្បីឲ្យ $g(x)=\frac{a}{x+1}+\frac{b}{\left(x+1\right)^2}$ ចំពោះគ្រប់ $x\ne-1$។
  		\item ដោះស្រាយសមីការឌីផេរ៉ង់ស្យែល $\left(x+1\right)^2y'=4x-1$ ចំពោះ $x\ne -1$ ដោយដឹងថា $y(0)=2019$។
  	\end{enumerate}
  	\item គេមានអនុគមន៍ $f$ កំណត់ដោយ $f(x)=\frac{x^2+x+4}{x+1}$ ហើយមានក្រាប $C$។
  	\begin{enumerate}[k]
  		\item រកដែនកំណត់ និងសិក្សាសញ្ញាដេរីវេ​ $f'(x)$ នៃអនុគមន៍។
  		\item សរសេរសមីការអាសុីមតូតឈរ និងអាសុីមតូតទ្រេតនៃក្រាប $C$។
  		\item សង់តារាងអថេរភាព អាសុីមតូត និងក្រាប $C$ នៃអនុគមន៍ $f$។
  	\end{enumerate}
  \end{enumerate}
  \newpage
  \maketitle
  \begin{center}
  	\kml{\underline{ប្រធានទី០២}}
  \end{center}
  \begin{enumerate}[I]
  	\item \begin{enumerate}[k]
  		\item ដោះស្រាយសមីការឌីផេរ៉ង់ស្យែល $(E): y'-3y=0$។
  		\item រកចម្លើយ $y_p$ មួយនៃសមីការ $(E)$ ដើម្បីឲ្យក្រាបនៃចម្លើយកាត់តាមចំណុច $M\left(0,1\right)$។
  	\end{enumerate}
  	\item អេលីប $E$ មួយមានសមីការ $x^2+4y-2x+16y+13=0$។
  	\begin{enumerate}[k]
  		\item រកកូអរដោនេនៃផ្ចិត កំពូល និងកំណុំនៃអេលីប $E$។
  		\item រកកូអរដោនេនៃចំណុចប្រសព្វរវាងអេលីប $E$ និងអ័ក្សអរដោនេ $y'Oy$។
  	\end{enumerate}
  	\item គេឲ្យ $A(x)$=$\frac{x+1}{\left(x-1\right)^2}$ ចំពោះ $x\ne 1$។
  	\begin{enumerate}[k]
  		\item រកចំនួនពិត $a$ និង $b$ ដើម្បីឲ្យ $A(x)=\frac{a}{x-1}+\frac{b}{\left(x-1\right)^2}$ ចំពោះគ្រប់ $x\ne1$។
  		\item គណនា $I(x)=\int A(x)dx$។
  	\end{enumerate}
  	\item អនុគមន៍ $f$ កំណត់ចំពោះគ្រប់ $x$ ដោយ $y=f(x)=x+e^x$ ហើយមានក្រាប $C$ ។
  	\begin{enumerate}[k]
  		\item គណនា $\lim_{x \to +\infty } f(x)$ និង $\lim_{x \to -\infty } f(x)$។
  		\item បង្ហាញថាបន្ទាត់ $d:y=x$ ជាអាសុីមតូតទ្រេតនៃក្រាប $C$ កាលណា $x\to -\infty$ ។
  		\item គណនាដេរីវេ $f'(x)$ ហើយសង់តារាងអថេរភាពនៃ $f$។
  		\item គណនា $f(-1),~f(0),~f(1)$ ហើយសង់ក្រាប $C$ ក្នុងតម្រុយកូអរដោនេមួយ។ គេយក $e=2.7, e^{-1}=0.4$។
  	\end{enumerate}
  \end{enumerate}
  \newpage
  \maketitle
  \begin{center}
  	\kml{\underline{ប្រធានទី០៣}}
  \end{center}
  \begin{enumerate}[I]
  	\item គណនាលីមីតនៃអនុគមន៍ខាងក្រោម៖
  	\begin{enumerate}[k,3]
  		\item $\lim_{x\to3}\frac{x^4+6x+1}{x^2+1}$
  		\item $\lim_{x\to+\infty}\frac{x-1}{\left(x+1\right)^2}$
  		\item $\lim_{x\to+\infty}\left(x^2+2-\ln x\right)$
  	\end{enumerate}
  	\item ក្នុងថ្នាក់រៀនមួយមានសិស្ស $15$ នាក់ ក្នុងនោះសិស្យប្រុស $9$ នាក់ និងសិស្សស្រី $6$ នាក់ ។ \\ 
  	គេជ្រើសរើសសិស្ស $3$ នាក់ ដោយចៃដន្យជាតំណាងទៅសម្ភាសន៍ ។ គណនាប្រូបាបនៃព្រឹត្តិការណ៍ខាងក្រោម៖
  	\begin{enumerate}[A]
  		\item ក្រុមសិស្សទាំង $3$ នាក់ សុទ្ធតែជាសិស្សស្រី ។
  		\item ក្រុមសិស្សទាំង $3$ នាក់ សុទ្ធតែជាសិស្សប្រុស ។
  		\item ក្រុមសិស្សទាំង $3$ នាក់ មាន $2$ នាក់ជាសិស្សស្រី ។
  	\end{enumerate}
  	\item គណនាអាំងតេក្រាលខាងក្រោម៖ 
  	\begin{enumerate}[k,3]
  		\item $\mathrm{I}=\int_{1}^{2}\left(3x^2-2x+3\right)dx$
  		\item $\mathrm{J}=\int_{0}^{1}\left(e^{2x}-e^x+1\right)dx$
  		\item $\mathrm{K}=\int_{1}^{2}\left(\frac{1}{x+3}+\frac{1}{x^2}\right)dx$
  	\end{enumerate}
  	\item គេមានប៉ារ៉ាបូលមួយដែលមានកំពូលជាចំណុច $O\left(0,0\right)$ និងកំណុំ $F$ ស្ថិតនៅលើអ័ក្សអាប់ស៊ីស ។
  	\begin{enumerate}[k]
  		\item រកសមីការស្តង់ដានៃប៉ារ៉ាបូលនេះ បើគេដឹងថាវាកាត់តាមចំណុច $A\left(\frac{3}{2};-3\right)$ ។
  		\item រកកូអរដោនេរបស់កំណុំ សមីការបន្ទាត់ប្រាប់ទិស រួចសង់ប៉ារ៉ាបូលនេះ ។
  	\end{enumerate}
  	\item គេមានអនុគមន៍ $f$ កំណត់ដោយ $f(x)=\frac{2x^2-7x+5}{x^2-5x+7}$ ។ យើងតាងដោយក្រាប $C$ របស់វាលើតម្រុយអរតូណរម៉ាល់ $\left(O, \vec{i}, \vec{j}\right)$ ។
  	\begin{enumerate}[1]
  		\item រកដែនកំណត់ $\mathbb{D}$ នៃអនុគមន៍ $f$ ។
  		\item សិក្សាលីមីតនៃអនុគមន៍ $f(x)$ ត្រង់ $-\infty$ និងត្រង់ $+\infty$ ។ ទាញរកសមីការអាស៊ីមតូត $d$ ទៅនឹងក្រាប $C$ ត្រង់ $-\infty$ និង $+\infty$ ។
  		\item \begin{enumerate}[k]
  			\item ស្រាយបំភ្លឺថាគ្រប់ចំនួនពិត $x\in\mathbb{D}~,$ ដេរីវេ $f'(x)=\frac{-3\left(x^2-6x+8\right)}{\left(x^2-5x+7\right)}$ ។
  			\item សិក្សាអថេរភាពនៃអនុគមន៍ $f$ និងសង់តារាអថេរភាពនៃអនុគមន៍ $f$ ។
  			\item សង់ក្រាប $C$ នៃអនុគមន៍ $f$ ។
  		\end{enumerate}
  	\end{enumerate}
  \end{enumerate}
\newpage
 {\maketitle}
  \begin{center}
  	\kml{\underline{ប្រធានទី០៤}}
  \end{center}
  
  \begin{enumerate}[I]
  \item គណនាលីមីត៖
\begin{enumerate}[k,3]
\item $\lim_{x\to +\infty}\frac{x^2+x+1}{x^2+1}$
\item $\lim_{x\to 3}\frac{x^3-27}{\sqrt{x+6}-3}$
\item $\lim_{x\to 0}\frac{e^x+e^{-x}}{2}$
\end{enumerate}
\item ក្នុងថង់មួយមានប៊ូលពណ៌សចំនួន៣ និងប៊ូលពណ៌ក្រហមចំនួន៦។ គេចាប់យកប៊ូល៣ ក្នុងពេលតែមួយចេញពីថង់ដោយចៃដន្យ។ រកប្រូបាបនៃព្រឹត្តិការណ៍ខាងក្រោម៖
\begin{enumerate}[A]
\item : "ប៊ូលទាំងបីមានពណ៌ស"
\item : "ប៊ូលទាំងបីមានពណ៌ក្រហម"
\item : "មានប៊ូលមួយពណ៌ក្រហម និងពីរទៀតពណ៌ស"
\end{enumerate}
\item គណនាអាំងតេក្រាលខាងក្រោម៖
\begin{enumerate}[k,3]
		\item $\mathrm{I}=\int_1^3\left(3x^2+2x+1\right)dx$
		\item $\mathrm{J}=\int_0^1\left(2e^x-1\right)dx$
		\item $\mathrm{K}=\int_1^2\left(x+\frac{1}{x^2}\right)dx$
		\end{enumerate}
\item គេមានសមីការ $9x^2+25y^2=225$ ។
		\begin{enumerate}[k]
		\item បង្ហាញថាសមីការនេះជាសមីការអេលីប។ រកប្រវែងអ័ក្សតូច ប្រវែងអ័ក្សធំ និងកូអរដោនេនៃកំពូលទាំងពីរ។
		\item សង់អេលីបនេះ។
		\end{enumerate}
\item គេមានអនុគមន៍ $f$ កំណត់លើ $\mathbb{R}-\{2\}$ ដោយ $f(x)=\frac{x^2-x-1}{x-2}$ ។ យើងតាង $C$ ជាក្រាបរបស់វា លើតម្រុយអរតូណរម៉ាល់ $\left(0,\overrightarrow{i},\overrightarrow{j}\right)$ ។
\begin{enumerate}[1]
\item សិក្សាលីមីតនៃអនុគមន៍ $f$ ត្រង់ $-\infty$ និងត្រង់ $+\infty$ ។
\item សិក្សាអថេរភាព និងសង់តារាងអថេរភាពនៃអនុគមន៍ $f$ ។
\item \begin{enumerate}[a]
\item រកចំនួនពិត $a,b,c$ ដែលគ្រប់ $x\neq 2;\ \ \ f(x)=ax+b+\frac{c}{x-2}$ ។
\item គេតាង $\mathrm{d}$ ដែលមានសមីការ $y=x+1$។ បង្ហាញថា $d$ ជាអាស៊ីមតូតនៃ $C$ ត្រង់ $+\infty$ និង $-\infty$។ 
\\ សិក្សាទីតាំងនៃក្រាប $C$ ធៀបនឹងបន្ទាត់ $\mathrm{d}$ ។
\item សង់ក្រាប $C$ និង បន្ទាត់ $d$ ។
\end{enumerate}
\end{enumerate}
\end{enumerate}

\end{document}
